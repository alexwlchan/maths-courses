\documentclass[twoside, numbers=noendperiod]{scrartcl}

\usepackage[blacklinks]{drangreport}

\usepackage{todonotes}
\usepackage{awlc-algebra}

\usetikzlibrary{calc,
				fadings,
				arrows,
				decorations.markings,
				decorations.pathreplacing,
        scopes}

\let\pp\relax
\let\tt\relax
\newcommand{\va}{\vec{a}}
\newcommand{\vb}{\vec{b}}
\newcommand{\vc}{\vec{c}}
\newcommand{\vv}{\vec{v}}
\newcommand{\ww}{\vec{w}}
\newcommand{\xx}{\vec{x}}
\newcommand{\yy}{\vec{y}}
\newcommand{\zz}{\vec{z}}
\newcommand{\pp}{\vec{p}}
\newcommand{\qq}{\vec{q}}
\newcommand{\rr}{\vec{r}}
\newcommand{\ee}{\vec{e}}
\newcommand{\tt}{\vec{t}}
\newcommand{\nn}{\vec{n}}

\newcommand{\ii}{\vec{i}}
\newcommand{\jj}{\vec{j}}
\newcommand{\kk}{\vec{k}}

\let\AA\relax
\newcommand{\AA}{\vec{A}}
\newcommand{\BB}{\vec{B}}
\newcommand{\CC}{\vec{C}}
\newcommand{\NN}{\vec{N}}

\newcommand{\HH}{\bb{H}}
\newcommand{\iH}{\in\HH}

\renewcommand{\vec}[1]{\mathbf{#1}}

\newcommand{\OO}{\vec{0}}

\newcommand{\zbar}{\overline{z}}
\newcommand{\wbar}{\overline{w}}

\DeclareMathOperator{\Area}{Area}
\DeclareMathOperator{\length}{length}
\DeclareMathOperator{\defect}{\delta}
\DeclareMathOperator{\Mobgp}{Mob}

\newcommand{\closure}[1]{\overline{#1}}
\newcommand{\boundary}[1]{\partial{#1}}

\renewcommand{\varphi}{\phi}

% Riemannian geometry
\newcommand{\ddif}[2]{\eval[0]{\dif{#1}}_{#2}}
\newcommand{\ssig}[1]{\eval[0]{\sigma_{#1}}_{\pp}}
\newcommand{\bsig}{\boldsymbol{\sigma}}

\renewcommand{\vec}[1]{#1}

\title{IB Geometry}

\newcommand\pgfmathsinandcos[3]{%
  \pgfmathsetmacro#1{sin(#3)}%
  \pgfmathsetmacro#2{cos(#3)}%
}
\newcommand\LongitudePlane[3][current plane]{%
  \pgfmathsinandcos\sinEl\cosEl{#2} % elevation
  \pgfmathsinandcos\sint\cost{#3} % azimuth
  \tikzset{#1/.estyle={cm={\cost,\sint*\sinEl,0,\cosEl,(0,0)}}}
}
\newcommand\LatitudePlane[3][current plane]{%
  \pgfmathsinandcos\sinEl\cosEl{#2} % elevation
  \pgfmathsinandcos\sint\cost{#3} % latitude
  \pgfmathsetmacro\yshift{\cosEl*\sint}
  \tikzset{#1/.estyle={cm={\cost,0,0,\cost*\sinEl,(0,\yshift)}}} %
}
\newcommand\DrawLongitudeCircle[2][1]{
  \LongitudePlane{\angEl}{#2}
  \tikzset{current plane/.prefix style={scale=#1}}
   % angle of "visibility"
  \pgfmathsetmacro\angVis{atan(sin(#2)*cos(\angEl)/sin(\angEl))} %
  \draw[current plane] (\angVis:1) arc (\angVis:\angVis+180:1);
  \draw[current plane,dashed] (\angVis-180:1) arc (\angVis-180:\angVis:1);
}
\newcommand\DrawLatitudeCircle[2][1]{
  \LatitudePlane{\angEl}{#2}
  \tikzset{current plane/.prefix style={scale=#1}}
  \pgfmathsetmacro\sinVis{sin(#2)/cos(#2)*sin(\angEl)/cos(\angEl)}
  % angle of "visibility"
  \pgfmathsetmacro\angVis{asin(min(1,max(\sinVis,-1)))}
  \draw[current plane] (\angVis:1) arc (\angVis:-\angVis-180:1);
  \draw[current plane,dashed] (180-\angVis:1) arc (180-\angVis:\angVis:1);
}

\tikzset{%
  % >=latex, % option for nice arrows
  % inner sep=0pt,%
  % outer sep=2pt,%
  mark coordinate/.style={inner sep=0pt,outer sep=0pt,minimum size=3pt,
    fill=black,circle}%
}

\begin{document}

\NotesTitle{IB}{Lent 2013}{Geometry}{Prof.~J.}{Rasmussen}
{
\section*{Course schedule}

Groups of rigid motions of Euclidean space. Rotation and reflection groups in two and three dimensions. Lengths of curves. \hfill [2]

Spherical geometry: spherical lines, spherical triangles and the Gauss-Bonnet theorem. Stereographic projection and Möbius transformations. \hfill [3]

Triangulations of the sphere and the torus, Euler number. \hfill [1]

Riemannian metrics on open subsets of the plane. The hyperbolic plane. Poincar\'{e} models and their metrics. The isometry group. Hyperbolic triangles and the Gauss-Bonnet theorem. The hyperboloid model. \hfill [4]

Embedded surfaces in $\R^3$. The first fundamental form. Length and area. Examples. \hfill [1]

Length and energy. Geodesics for general Riemannian metrics as stationary points of the energy. First variation of the energy and geodesics as solutions of the corresponding Euler-Lagrange equations. Geodesic polar coordinates (informal proof of existence). Surfaces of revolution. \hfill [2]

The second fundamental form and Gaussian curvature. For metrics of the form $\dif{u}^2 + G(u,v) \dif{v}^2$, expression of the curvature as $\sqrt{G_{uu}}/\sqrt{G}$\,. Abstract smooth surfaces and isometries. Euler numbers and statement of Gauss-Bonnet theorem, examples and applications. \hfill [3]

\subsection*{Appropriate books}

{\shortskip
P.M.H.~Wilson \emph{Curved Spaces}. CUP, January 2008 (£60 hardback, £24.99 paperback).

M.~Do Carmo \emph{Differential Geometry of Curves and Surfaces}. Prentice-Hall, Inc., Englewood Cliffs, N.J., 1976 (£42.99 hardback)

A.~Pressley \emph{Elementary Differential Geometry}. Springer Undergraduate Mathematics Series, Springer-Verlag London Ltd., 2001 (£19.00 paperback)

E.~Rees \emph{Notes on Geometry}. Springer, 1983 (£18.50 paperback)

M.~Reid and B.~Szendroi \emph{Geometry and Topology}. CUP, 2005 (£24.99 paperback)}}

\TableofContents

%!TEX root = geometry.tex
\stepcounter{lecture}
\setcounter{lecture}{1}
\sektion{Euclidean geometry}

\subsection[Geometry in $\Rn$]{Geometry in $\boldsymbol{\sf{R}^\sf{n}}$} % (fold)
\label{sub:geometry_in_rn}

\lecturemarker{1}{21 Jan}

For $\vv, \ww \iRn$, the \emph{dot product} is defined as
\begin{equation*}
	\vv \cdot \ww = \sum_{i=1}^n v_i \, w_i.
\end{equation*}
The \emph{norm} or ``length'' of a vector is
\begin{equation*}
	\left\vert \vv \right\vert = \sqrt{\vv \cdot \vv}
\end{equation*}
and this satisfies the triangle inequality:
\begin{equation*}
	\left\vert \vv + \ww \right\vert \leq \left\vert \vv \right\vert + \left\vert \ww \right\vert,
\end{equation*}
with equality if and only if $\vv = k\ww$ or $\ww=k\vv$, for some $k \geq 0$.

\subsubsection*{Distance} % (fold)
\label{ssub:distance}

For $\xx,\yy\iRn$, $d(\xx,\yy) = \left\vert \xx-\yy \right\vert$ defines the \emph{Euclidean metric} on $\Rn$. We call this the Euclidean metric because it satisfies:
\begin{enumerate}
	\shortskip
	\item $d(\xx, \yy) \geq 0$ for all $\xx, \yy \iRn$, with equality if and only if $\xx = \yy$; %
	\item $d(\xx,\yy) = |\xx - \yy| = d(\yy,\xx)$, so it is symmetric;
	\item $d(\xx,\yy) = \left\vert \xx-\yy \right\vert + \left\vert \yy-\zz \right\vert \geq \left\vert \xx-\zz \right\vert = d(\xx,\zz)$, the triangle inequality. %
\end{enumerate}
So it satisfies the axioms for a metric space.

% subsubsection distance (end)

\subsubsection*{Lines} % (fold)
\label{ssub:lines}

The line through $\xx$ with direction vector $\vv$ is the set
\begin{equation*}
	\left\{ \xx+t\vv \mid t\iR \right\}.
\end{equation*}
The \emph{ray} starting at $\xx$ with direction vector $\vv$ is the set
\begin{equation*}
	\left\{ \xx+t\vv \mid t\iR, t\geq 0 \right\}.
\end{equation*}
The line segment from $\xx$ to $\yy$ is the set
\begin{equation*}
	\left\{ \xx + t\left( \yy-\xx \right) \mid t\in[0,1] \right\}.
\end{equation*}
Two direction vectors determine the same line through $\xx$ if and only if they are scalar multiples of each other.

Two direction vectors determine the same ray through $x$ if they are positive scalar multiples of each other.

\vspace{3pt}

\begin{proposition}
	Two distinct points lie on a unique line. %
\end{proposition}

\begin{proof}
	If $\xx$ and $\yy$ are points, then $\yy = \xx + t\vv \implies t\vv = \yy - \xx$, and the direction vector is determined up to a scalar multiple. %
\end{proof}

% subsubsection lines (end)

\subsubsection*{Angles} % (fold)
\label{ssub:angles}

If $R_1$ and $R_2$ are rays starting at $\xx$ with direction vectors $\vv_1$, $\vv_2$, then the angle between $R_1$ and $R_2$ is $0\leq \theta \leq \pi$ satisfying
\begin{equation*}
	\cos \theta = \f{\vv_1 \cdot \vv_2}{\left\vert \vv_1 \right\vert \left\vert \vv_2 \right\vert}.
\end{equation*}

% subsubsection angles (end)

	\pagebreak

We\lecturemarker{2}{23 Jan} want to show that the shortest path between two points in Euclidean space is a line. To do this, we need to have a notion of the length of a path. Once we have this, then the result becomes pretty tautological.

\begin{definition}
	A \emph{path} in a metric space $X$ is a continuous map $\gamma:[0,1] \to X$. %
	
	\begin{center}
	\begin{tikzpicture}[scale=2]
		\draw [thick] plot [smooth] coordinates {(-0.8, -0.512) (-0.7, -0.343) (-0.6, -0.216) (-0.5, -0.125)
			(-0.4, -0.064) (-0.3, -0.027) (-0.2, -0.008) (-0.1, -0.001) (0,0)
			(0.1, 0.001) (0.2, 0.008) (0.3, 0.027) (0.4, 0.064) (0.5, 0.125)
			(0.6, 0.216) (0.7, 0.343) (0.8, 0.512)}; %
		
		\draw (0.8,0.512) node {$\bullet$};
		\draw (-0.8,-0.512) node {$\bullet$};
		
		\draw (0.8,0.512) node [above right] {$\gamma(1)$};
		\draw (-0.8,-0.512) node [below left] {$\gamma(0)$};
		
		\bigarrow (-0.01,0) -- (0.01,0);
	\end{tikzpicture}
	\end{center}
	
	If $f:[0,1] \to [0,1]$ is a continuous bijection (implying that $f$ is a homeomorphism, since $\Rn$ is compact, and so $f^{-1}$ is continuous), then we say that $\gamma\circ f$ is a \emph{reparametrisation} of $\gamma$. %
\end{definition}

Now we want to define a notion of distance along a path. Suppose we approximate our path by a series of line segments. Then it should be intuitive that the length of our path is at least as long as any such approximation.

\begin{center}
\begin{tikzpicture}[scale=2.5]
	\draw [thick] plot [smooth] coordinates {(-1,0) (-0.9, 0.171) (-0.8, 0.288) (-0.7, 0.357) (-0.6, 0.384) (-0.5, 0.375) (-0.4, 0.336) (-0.3, 0.273) (-0.2, 0.192) (-0.1, 0.099) (0,0) (0.1, -0.099) (0.2, -0.192) (0.3, -0.273) (0.4, -0.336) (0.5, -0.375) (0.6, -0.384) (0.7, -0.357) (0.8, -0.288) (0.9, -0.171) (1,0)}; %
	
	\foreach \s/\t/\u/\v in {-1/0/-0.4/0.336, -0.4/0.336/0.1/-0.099, 0.1/-0.099/0.7/-0.357, 0.7/-0.357/1/0}
	{
		\draw (\s,\t) -- (\u,\v);
		\draw (\s,\t) node {$\bullet$};
	}
	
	\draw (-1,0) node [below right] {$t_0$};
	\draw (-0.4,0.336) node [above right] {$t_1$};
	\draw (0.1,-0.099) node [below left] {$t_2$};
	\draw (0.7,-0.357) node [below right] {$t_3$};
	\draw (1,0) node [above right] {$t_4$};
	
	\draw (1,0) node {$\bullet$};
	
\end{tikzpicture}
\end{center}

Let's try to formalise this. Let $A = \left\{0 = t_0 < t_1 < \cdots < t_n = 1 \right\} \subset [0,1]$ be a finite subset. Now define
\begin{equation*}
	L_A(\gamma) = \sum_{i=1}^n d(\gamma(t_i), \gamma(t_{i-1})).
\end{equation*}
It should be clear that if we increase an extra point, $\overline{t_j}$, that this length increases.

\begin{center}
\begin{tikzpicture}[scale=2.5]
	\draw [thick] plot [smooth] coordinates {(-1,0) (-0.9, 0.171) (-0.8, 0.288) (-0.7, 0.357) (-0.6, 0.384) (-0.5, 0.375) (-0.4, 0.336) (-0.3, 0.273) (-0.2, 0.192) (-0.1, 0.099) (0,0) (0.1, -0.099) (0.2, -0.192) (0.3, -0.273) (0.4, -0.336) (0.5, -0.375) (0.6, -0.384) (0.7, -0.357) (0.8, -0.288) (0.9, -0.171) (1,0)}; %
	
	\foreach \s/\t/\u/\v in {-1/0/-0.7/0.357, -0.7/0.357/-0.4/0.336, -0.4/0.336/0.1/-0.099, 0.1/-0.099/0.7/-0.357, 0.7/-0.357/1/0}
	{
		\draw (\s,\t) -- (\u,\v);
		\draw (\s,\t) node {$\bullet$};
	}
	
	\draw (-1,0) node [below right] {$t_0$};
	\draw (-0.4,0.336) node [above right] {$t_1$};
	\draw (0.1,-0.099) node [below left] {$t_2$};
	\draw (0.7,-0.357) node [below right] {$t_3$};
	\draw (1,0) node [above right] {$t_4$};
	
	\draw (-0.7,0.357) node [above left] {$\overline{t_j}$};
	
	\draw (1,0) node {$\bullet$};
	
\end{tikzpicture}
\end{center}

If we let $A\p = \left\{ 0 = t_0 < t_1 < \cdots < t_{j-1} < \overline{t_j} < t_j < \cdots < t_n = 1 \right\}$, then by the triangle inequality, $L_{A\p}(\gamma) \geq L_A(\gamma)$. Thus:
\begin{equation*}
	L_A(\gamma) \geq L_{\{0,1\}}(\gamma) = d(\gamma(0),\gamma(1)).
\end{equation*}
Also we have that if $A \subset A\p$, then $L_A(\gamma) \leq L_{A\p}(\gamma)$.

	\pagebreak

With these thoughts, we're ready to define the length of a path, and the following definition seems natural:

\begin{definition}
	The \emph{length} of a path $\gamma$ is
	\begin{equation*}
		L(\gamma) = \sup_A L_A(\gamma)
	\end{equation*}
	as $A$ runs over the finite subsets of $[0,1]$. %
\end{definition}

\begin{example}
	The line segment from $\xx$ to $\yy$ is parameterised by $\gamma(t) = \xx + t(\yy - \xx)$.

	The triangle inequality in $\Rn$ states that
	\begin{equation*}
		\left\vert \xx - \yy \right\vert + \left\vert \yy - \zz \right\vert \geq \left\vert \xx - \zz \right\vert,
	\end{equation*}
	with equality if and only if any of the three equivalent conditions hold:
	\begin{itemize}
		\shortskip
		\item $\xx - \yy$ is a nonnegative multiple of $\yy - \zz$;
		\item $\xx - \zz$ is a $\geq 1$ multiple of $\yy - \zz$;
		\item $\yy$ lies on the line segment from $\xx$ to $\zz$.
	\end{itemize}
	In particular, this last condition means that if $\gamma$ is a line segment from $\xx$ to $\yy$, then $L_A(\gamma) = d(\xx,\yy)$ for all $A$, and so %
	\begin{equation*}
		L(\gamma) = \sup_A L_A(\gamma) = d(\xx,\yy).
	\end{equation*}
\end{example}

Now let's check that there isn't some other path with the same length as the line segment. First we need the following lemma:

\begin{lemma}
	If $\gamma\circ f$ is a reparameterisation of $\gamma$, then $L(\gamma \circ f) = L(\gamma)$. %
\end{lemma}

\begin{proof}
	We have $L_A(\gamma \circ f) = L_{f(A)}(\gamma)$. Also $L_A(\gamma \circ f) \leq \sup_A L_A(\gamma) = L(\gamma)$, and so $L(\gamma \circ f) \leq L(\gamma)$. %
	
	Similarly $\gamma= (\gamma \circ f) \circ f^{-1}$ implies $L(\gamma) = L(\gamma \circ f \circ f^{-1}) \leq L(\gamma \circ f)$. %
	
	Hence $L(\gamma) = L(\gamma \circ f)$.
\end{proof}

\begin{proposition}
	The line segment from $\xx$ to $\yy$ is the shortest path from $\xx$ to $\yy$. Precisely, if $\gamma_0$ is the line segment from $\xx$ to $\yy$ and $\gamma_1$ is a path from $\xx$ to $\yy$ with $L(\gamma_0) = L(\gamma_1) = d(\xx,\yy)$, then $\gamma_1 = \gamma_0 \circ f$, where $f:[0,1] \to [0,1]$ is continuous, and $t\geq s \implies f(t)\geq f(s)$. %
	
	% if $\gamma_0$ is the line segment and $\gamma$ is another path from $\xx$ to $\yy$, $L(\gamma_1) \geq L(\gamma_0)$ with equality if and only if $\gamma_1 = \gamma_0 \circ f$ is a reparameterisation of $\gamma_0$. %
\end{proposition}

\vspace{-6pt}

This does not imply that $f$ is invertible. We will call this property a \emph{weak reparameterisation}.

\begin{proof}
	We've already shown that $L(\gamma_1) \geq d(\xx,\yy) = L(\gamma_0)$. To have equality, we need equality everywhere in the triangle inequality. %
	
	That means that $\gamma_1(t)$ is on the line segment for all $t$. (Otherwise $L_{\{0,t,1\}}(\gamma_1) \geq d(\xx,\yy)$.

	Thus $\gamma_1(t)=\gamma_0(f(t))$ for some $f:[0,1] \to [0,1]$. Now $f(t) = \gamma_0^{-1} \circ \gamma_1(t)$ is cts.

	Suppose that $f(s) \geq f(t)$ for $t\geq s$. Then $L_{\{0,s,t,1\}}(\gamma_1) \geq d(\xx,\yy)$.

	So if $L(\gamma_1) = L(\gamma_0)$, then $t\geq s \implies f(t) \geq f(s)$, and $f$ is a bijection.
\end{proof}

	\pagebreak

\begin{proposition}
	Suppose $\gamma:[0,1] \to \Rn$ is continuously differentiable. Then %
	\begin{equation*}
		L(\gamma) = \int_0^1 \left\vert \gamma\p(t) \right\vert \dif{t}.
	\end{equation*}
\end{proposition}

\begin{proof}
	Write $\gamma\p(t) = (\gamma_1\p(t), \ldots, \gamma_n\p(t))$. Now $\gamma_i\p(t)$ is a continuous function on a compact set, so is uniformly continuous. That is, given $\epsilon>0$, there is some $\delta>0$ such that $\left\vert \gamma_i\p(t) - \gamma_i\p(s) \right\vert < \epsilon$ whenever $\left\vert t-s \right\vert<\delta$. %
	
	By the mean value theorem,
	\begin{equation*}
		\gamma_i(t) - \gamma_i(s) = \left( t-s \right) \gamma_i\p(t_i),
	\end{equation*}
	with $s\leq t_i \leq t$.  So if $\left\vert t-s \right\vert<\delta$, then
	\begin{equation*}
		\left\vert \gamma_i(t) - \gamma_i(s) - \left( t-s \right) \gamma_i\p(t) \right\vert \leq \left\vert t-s \right\vert \left\vert \gamma_i\p(t) - \gamma_i\p(t_i) \right\vert \leq \left\vert t-s \right\vert \epsilon. %
	\end{equation*}
	Then applying the triangle inequality repeatedly, we have
	\begin{equation*}
		\left\vert \gamma(t) - \gamma(s) - \left( t-s \right) \gamma\p(t) \right\vert \leq n\left\vert t-s \right\vert \epsilon. %
	\end{equation*}
	Now if $A\subset [0,1]$ satisfies $t_i - t_{i-1}<\delta$ for all $i$, then
	\begin{equation*}
		\left\vert \sum \left\vert \gamma(t_i) - \gamma(t_{i-1}) \right\vert - \sum \left( t_i-t_{i-1} \right) \left\vert \gamma\p(t_i) \right\vert \right\vert \leq ne \sum \left\vert t_i - t_{i-1} \right\vert \leq n \epsilon. %
	\end{equation*}
	Now $\sum \left( t_i-t_{i-1} \right) \left\vert \gamma\p(t_i) \right\vert$ is the right Riemann sum for $\int_0^1 \left\vert \gamma\p(t) \right\vert \dif{t}$, so if we take $A\p$ with $t_i-t_{i-1} < \delta\p < \delta$, then %
	\begin{equation*}
		\left\vert \sum \left( t_i-t_{i-1} \right) \left\vert \gamma\p(t_i) \right\vert - \int_0^1 \left\vert \gamma\p(t) \right\vert \dif{t} \right\vert \leq \epsilon. %
	\end{equation*}
	Thus we have
	\begin{equation*}
		\left\vert L_{A\p}(\gamma) - \int_0^1 \left\vert \gamma\p(t) \right\vert \dif{t} \right\vert < \left( n+1 \right) \epsilon %
	\end{equation*}
	whenever $t_i-t_{i-1} < \delta\p$ for all $i$.
	
	Given any $A$, pick $A\p$ satisfying the condiion above, then
	\begin{equation*}
		L_A(\gamma) \leq L_{A\p}(\gamma) \leq \int_0^1 \left\vert \gamma\p(t) \right\vert \dif{t} + \left( n+1 \right) \epsilon %
	\end{equation*}
	and
	\begin{equation*}
		L_{A\p}(\gamma) \geq \int_0^1 \left\vert \gamma\p(t) \right\vert \dif{t} - \left( n+1 \right)\epsilon. %
	\end{equation*}
	Combining these two, we have
	\begin{equation*}
		L(\gamma) = \sup_A L_A(\gamma) = \int_0^1 \left\vert \gamma\p(t) \right\vert \dif{t}. \qedhere %
	\end{equation*}
\end{proof}

% subsection geometry_in_rn (end)

	\pagebreak

\subsection[Isometries of $\Rn$]{Isometries of $\boldsymbol{\sf{R}^\sf{n}}$} % (fold)
\label{sub:isometries_of_rn}

\begin{definition}
	Let $(X,d_X)$ and $(Y,d_Y)$ be metric spaces. A bijection $\phi:X\to Y$ is an \emph{isometry} if it preserves distances; that is, %
	\begin{equation*}
		d_X(X_1,X_2) = d_Y(\phi(X_1), \phi(X_2))
	\end{equation*}
	for all $X_1, X_2 \in X$.
\end{definition}

An isometry is continuous: given $\epsilon>0$, $d_Y(\phi(X_1),\phi(X_2)) < \epsilon$ whenever $d_X(X_1,X_2) < \epsilon$.

\begin{lemma}
	The inverse of an isometry is an isometry. The composition of two isometries is an isometry. %
\end{lemma}

\begin{proof}
	Suppose $\phi:X \to Y$ is an isometry with $\phi(X_i) = Y_i$. Then $d_Y(Y_1, Y_2) = d_X(X_1,X_2)$, and so $d_Y(Y_1,Y_2) = d_X(\phi^{-1}(Y_1), \phi^{-1}(Y_2))$, which shows that $\phi^{-1}$ is an isometry. %
	
	If $\psi: Y \to Z$ is an isometry, then
	\begin{equation*}
		d_Z(\psi(\phi(X_1)), \psi(\phi(X_2)) = d_Y(\phi(X_1), \phi(X_2)) = d_X(X_1,X_2),
	\end{equation*}
	and so $\psi \circ \phi$ is an isometry.
\end{proof}

\begin{corollary}
	Let $\Isom(X)$ be the set of isometries:
	\begin{equation*}
		\Isom(X) = \left\{\phi:X \to X \mid \phi \text{ is an isometry}\right\}.
	\end{equation*}
	Then $\Isom(X)$ is a group under composition. %
\end{corollary}

\lecturemarker{3}{28 Jan}

\begin{examples}
\mbox{}
\begin{enumerate}
	\item \emph{Translations.} If $\vv\iRn$, define $T_{\vv}:\Rn\to\Rn$ by $T_{\vv}(\xx) = \xx + \vv$. Then %
	\begin{equation*}
		\left\vert T_{\vv}(\xx) - T_{\vv}(\yy) \right\vert
		= \left\vert \xx+\vv - \yy-\vv \right\vert
		= \left\vert \xx-\yy \right\vert.
	\end{equation*}
	It is clear that $T_{\vv}$ is bijective by $T_{\vv}^{-1} = T_{-\vv}$, and hence $T_{\vv}$ is an isometry. %
	
	\item \emph{Orthogonal transformations.} Recall that a \emph{linear} map $O:\Rn\to\Rn$ is orthogonal if
	\begin{equation*}
		O(\vv) \cdot O(\ww) = \vv \cdot \ww
	\end{equation*}
	for all $\vv,\ww \iRn$. (Or in matrix form, $OO^\Trans = I$.) The set of all such transformations is the \emph{orthogonal group}, $O(n)$. If $O\in O(n)$, then
	\begin{equation*}
		O\vv \cdot O\vv = \vv \cdot \vv \implies \left\vert O\vv \right\vert = \left\vert \vv \right\vert.
	\end{equation*}
	Then using the fact that $O\in O(n)$ is a linear map:
	\begin{equation*}
		\left\vert O\xx - O\yy \right\vert = \left\vert O\left( \xx-\yy \right) \right\vert = \left\vert \xx-\yy \right\vert, %
	\end{equation*}
	and so $O$ is an isometry.
	
	Consider the case $n=2$. $O\in O(2)$ looks like
	\begin{equation*}
		\mat{\cos\theta & -\sin\theta \\ \sin\theta & \cos\theta},
	\end{equation*}
	rotation by an angle $\theta$ around the origin.
	
		\pagebreak
	
	\begin{center}
	\begin{tikzpicture}[scale=2.5]
		
		\draw (0,-0.2) -- (0,1.2);
		\draw (-1.2,0) -- (1.2,0);
		
		\bigarrow (0,0) -- (1,0) node [below right] {$\ee_1$};
		\bigarrow (0,0) -- (0,1) node [right] {$\ee_2$};
		
		\bigarrow (0,0) -- (0.866,0.5) node [right] {$O(\ee_1) = (\cos\theta,\sin\theta)$};
		\bigarrow (0,0) -- (-0.5,0.866) node [left] {$=(\cos(\theta+\f{\pi}{2}),\sin(\theta+\f{\pi}{2}))$};
		\draw (-0.72,1.066) node [left] {$O(\ee_2) = (-\sin\theta,\cos\theta) \hspace{1cm}$};

		% 0.15
		
		\draw (0.0866,0.05) -- (0.0366,0.1366) -- (-0.05,0.0866);
		
		\draw (0.35,0) arc (0:30:0.35);
		\draw (0.33,0.11) node [right] {$\theta$};
		
	\end{tikzpicture}
	\end{center}
	or like
	\begin{equation*}
		\mat{\cos\theta & \phantom{-}\sin\theta \\ \sin\theta & -\cos\theta},
	\end{equation*}
	reflection in the line that makes angle $\theta/2$ with the $x$-axis.
	
	\emph{Proof.}
	\begin{equation*}
		\mat{\cos\theta & \sin\theta \\ \sin\theta & -\cos\theta}
		= \underset{\text{rotate by $\theta/2$}}{\mat{\cos\theta/2 & -\sin\theta/2 \\ \sin\theta/2 & \phantom{-}\cos\theta/2}} %
		  \underset{\substack{\text{reflect across} \\ \text{$x$-axis}}}{\mat{1 & 0 \\ 0 & -1}}
		  \underset{\text{rotate by }-\theta/2}{\mat{\phantom{-}\cos\theta/2 & \sin\theta/2 \\ -\sin\theta/2 & \cos\theta/2}}. %
	\end{equation*}
	How do we tell these two apart? We note that rotations have determinant $+1$, whereas reflections have determinant $-1$.
	
	\item \emph{Rotation by angle $\theta$ about some $\pp\iRn$.}
	Here we translate $\pp\iRn$ to the origin, perform our rotation, then undo the translation. That is, we have the composition $\phi=T_{\pp} \circ O_\theta \circ T_{-\pp}$, or %
	\begin{equation*}
		\phi(\xx) = \pp + O\left( \xx-\pp \right) = O\xx + \left( \pp-\theta\pp \right).
	\end{equation*}
\end{enumerate}
\end{examples}

It turns out that these examples are all we need to generate the orthogonal group, which is summarised by the following theorem:

\begin{theorem}
	Every $\phi\in\Isom(\Rn)$ can be written as $\phi=T_{\vv} \circ O$ for some $\vv\iRn$ and $O\in O(n)$; that is, $\phi(\xx)=O(\xx) + \vv$. %
\end{theorem}

We will prove this theorem through a series of lemmas.

\begin{lemma}
	\label{lem:orthog-1} If $\phi\in\Isom(\Rn)$ satisfies $\phi(\OO) = \OO$ and $\phi(\ee_i) = \ee_i$, where $\{\ee_i\}$ is the standard basis, then $\phi=\id_{\Rn}$. %
\end{lemma}

\begin{proof}
	Let $\phi(\xx)=\yy$. Then
	\begin{equation*}
		\vert \xx-\OO \vert^2
		= \vert \phi(\xx) - \phi(\OO) \vert^2
		= \vert \phi(\xx)-\OO \vert^2
		= \vert \yy \vert^2,
	\end{equation*}
	and so we have
	\begin{equation*}
		\textstyle\sum_{i=1}^n x_i^2 = \sum_{i=1}^n y_i^2. \tag{$*$}
		\label{eq:orthog-lem-11}
	\end{equation*}
	Similarly, for any basis vector $\ee_i$, we have
	\begin{equation*}
		\left\vert \xx-\ee_i \right\vert^2
		= \left\vert \phi(\xx) - \phi(\ee_i) \right\vert^2
		= \left\vert \yy - \ee_i \right\vert^2.
	\end{equation*}
	Hence we have
	\begin{equation*}
		x_1^2 + x_2^2 + \cdots + (x_i-1)^2 + \cdots + x_n^2 = y_1^2 + y_2^2 + \cdots + (y_i-1)^2 + \cdots + y_n^2. %
		\tag{$**$}
		\label{eq:orthog-lem-12}
	\end{equation*}
	Subtracting \eqref{eq:orthog-lem-11} from \eqref{eq:orthog-lem-12} gives $-2x_i+1 = -2y_i+1 \implies x_i=y_i$. Hence $\yy=\xx$, and $\phi=\id_{\Rn}$. %
\end{proof}

\begin{lemma}
	\label{lem:orthog-2} If $\phi\in\Isom(\Rn)$ satisfies $\phi(\OO) = \OO$, then $\phi(\xx) \cdot \phi(\yy) = \xx\cdot\yy$ for all $\xx,\yy\iRn$. %
\end{lemma}

\begin{proof}
	First we have
	\begin{equation*}
		\left\vert \phi(\xx) \right\vert^2
		= \left\vert \phi(\xx)-\phi(\OO) \right\vert^2
		= \left\vert \xx-\OO \right\vert^2
		= \left\vert \xx \right\vert^2.
		\tag{$*$}
		\label{eq:orthog-lem-21}
	\end{equation*}
	We also have
	\begin{equation*}
		\left\vert \phi(\xx)-\phi(\yy) \right\vert^2 = \left\vert \xx-\yy \right\vert^2.
	\end{equation*}
	This is also equal to
	\begin{equation*}
		\left\vert \phi(\xx) \right\vert^2 - 2\,\phi(\xx) \cdot \phi(\yy) + \left\vert \phi(\yy) \right\vert^2
		= \left\vert \xx \right\vert^2 - 2\,\xx\cdot\yy + \left\vert \yy \right\vert^2.
	\end{equation*}
	Finally, using \eqref{eq:orthog-lem-21}, we get $\phi(\xx)\cdot\phi(\yy) = \xx\cdot\yy$. %
\end{proof}

\begin{lemma}
	\label{lem:orthog-3} If $\phi\in\Isom(\Rn)$ with $\phi(\OO)=\OO$, then $\phi(\xx) = O\xx$ for some $O\in O(n)$. %
\end{lemma}

\begin{proof}
	Let $\vv_i=\phi(\ee_i)$. Then $\vv_i\cdot\vv_j = \phi(\ee_i) \cdot \phi(\ee_j) = \ee_i \cdot \ee_j = \delta_{ij}$ (lemma~\ref{lem:orthog-2}). %

	Thus $O=(\vv_1,\ldots,\vv_n) \in O(n)$, with $O(\ee_i)=\vv_i$.

	Then $O^{-1}\circ\phi\in\Isom(\Rn)$, and by lemma~\ref{lem:orthog-1},
	\begin{equation*}
		\begin{rcases}
			& O^{-1}\circ\phi(\ee_i) = \ee_i \\
			& O^{-1}\circ\phi(\OO) = \OO
		\end{rcases}
		\implies O^{-1}\circ \phi=\textstyle \id_{\Rn}
	\end{equation*}
	and so we have $\phi=O$. %
\end{proof}

\begin{proof}
	[Proof of theorem] Let $\vv = \phi(\OO)$. Then by lemma~\ref{lem:orthog-3},
	\begin{equation*}
		T_{\vv}^{-1} \circ \phi(\OO) = \OO, \qquad
		T_{\vv}^{-1} \circ \phi(\xx) = O\xx.
	\end{equation*}
	Thus $\phi(\xx)=O\xx+\vv$. %
\end{proof}

\begin{corollary}
	Isometries preserve angles. That is, if $\phi\in\Isom(\Rn)$, and $R_1,R_2$ are rays starting at $\xx$, then $\angle\phi(R_1),\phi(R_2) = \angle R_1,R_2$. %
\end{corollary}

\begin{proof}
	It suffices to check for $\phi=T_{\vv}$ and $\phi=O$. If $R_i$ has direction vector $\vv_i$, then
	\begin{equation*}
		T_{\vv}(R_i)
		= T_{\vv}\left( \left\{\xx+t\vv_i \mid t\geq 0 \right\} \right)
		= \left\{\vv_i+\xx+t\vv_i \mid t\geq 0 \right\}
	\end{equation*}
	which has direction vector $\vv_i$ also and so the angle is unchanged.

	Similarly $OR_i$ has direction vector $O\vv_i$, and we know that
	\begin{equation*}
		O\vv_1 \cdot O\vv_2 = \vv_1 \cdot \vv_2,
	\end{equation*}
	and so the angle is unchanged.
\end{proof}

	\pagebreak

\begin{definition}
	An \emph{orthogonal frame} at $\xx$ is an $n$-tuple of perpendicular rays, denoted $(R_1,\ldots,R_n)$, starting at $X$. %
	
	The \emph{standard frame} is $F_0=(X_1,\ldots,X_n)$, where $X_i$ is the positive $x_i$-axis. %
\end{definition}

\begin{corollary} % Corollary 2 
	If $F_1$ and $F_2$ are orthogonal frames, then there is a unique $\phi\in\Isom(\Rn)$ with $\phi(F_1)=F_2$. %
\end{corollary}

\begin{proof}
	Let $\vv_i^j$ be the direction vector for $R_i$. Then %
	\begin{equation*}
		O = \left( \f{\vv_1^j}{\Vert \vv_1^j \Vert}, \ldots, \f{\vv_n^j}{\Vert \vv_n^j \Vert} \right) \in O(n).
	\end{equation*}
	Let $\phi_j = T_{\xx_j} \circ O_j$ and $F_j = (R_1^j, \ldots, R_n^j)$.
	
	Then $\phi_j(F_0) = F_j$, and so $\phi=\phi_2\circ\phi^{-1}$ has
	\begin{equation*}
		\phi(F_1) =\phi_2(\phi_1^{-1}(F_1)) = \phi_2(F_0) = F_2.
	\end{equation*}
	That proves existence, now for uniqueness: if $\phi\p(F_1) = F_2$, then
	\begin{equation*}
		\phi_2^{-1} \circ \phi\p \circ \phi_1(F_0) = \phi_2^{-1}(\phi(F_1)) = \phi_2^{-1}(F_2) = F_0,
	\end{equation*}
	and $\phi_2^{-1} \circ \phi\p \circ \phi_1 = \id_{\Rn}$ by lemma~\ref{lem:orthog-1}. Thus $\phi\p = \phi_2 \circ \phi_1^{-1} = \phi$. %
\end{proof}

% subsection isometries_of_rn (end)

\subsection{The Euclidean plane} \lecturemarker{4}{30 Jan} % (fold)
\label{sub:the_euclidean_plane}

\begin{proposition}
	Two distinct lines in $\R^2$ intersect in at most one point. %
\end{proposition}

\begin{proof}
	The intersections are solutions of
	\begin{equation*}
		\xx+t\vv_1 = \yy+s\vv_2.
	\end{equation*}
	Rearranging this, we have
	\begin{equation*}
		t\vv_1 - s\vv_2 = \yy-\xx.
	\end{equation*}
	If $\vv_1$ and $\vv_2$ are linearly independent, then there is a unique solution. If they are linearly dependent:
	\begin{itemize}
		\shortskip
		\item either $\yy-\xx$ is in the span of $\vv_1$, and the lines are the same;
		\item or $\yy-\xx$ is not in the span of $\vv_1$, and there are no solutions. \qedhere
	\end{itemize}
\end{proof}

\begin{definition}
	Two distinct lines in $\R^2$ are \emph{parallel} if they do not intersect. %
\end{definition}

\begin{corollary}
	If $L$ is a line, and $p$ is a point not on $L$, then there is a unique line $L\p$, that passes through $p$ and is parallel to $L$. %
\end{corollary}

\begin{proof}
	The calculation above shows that the direction vector of $L\p$ is a scalar multiple of the direction vector of $L$.

	We saw before that there's a unique line passing through $p$ with a given diection (up to scalar multiple). %
\end{proof}

\begin{definition}
	The \emph{circle of radius $r$, centred at $\xx$} is given by
	\begin{equation*}
		\left\{ \yy : d(\xx,\yy) = r\right\}.
	\end{equation*}
\end{definition}

\begin{proposition}
	A line and a circle intersect in at most two points. %
\end{proposition}

\begin{proof}
	Suppose the circle is centred at $p$.

	We need to solve the two equations:
	\begin{gather*}
		ax_1 + bx_2 + c = 0, \tag{line} \\
		\left( x_1-p_1 \right)^2 + \left( x_2-p_2 \right)^2 = r^2. \tag{circle}
	\end{gather*}
	We can solve for $x_1$ in terms of $x_2$ (or vice versa, if $a=0$). Substitute to get a quadratic equation for $x_2$, and so there are at most two solutions. %
\end{proof}

% subsection the_euclidean_plane (end)		% Euclidean geometry
%!TEX root = geometry.tex
\stepcounter{lecture}
\setcounter{lecture}{2}
\sektion{Spherical geometry}

\subsection{Basics} % (fold)
\label{sub:basics}

\begin{definition}
	The {sphere $S^{\,2}$} is
	\begin{equation*}
		\left\{ (x,y,z) \subseteq \R^3 : x^2 + y^2 + z^2 = 1 \right\}.
	\end{equation*}
	The \emph{tangent space} to $S^{\,2}$ at $\pp\in S^{\,2}$ is
	\begin{equation*}
		T_{\pp}S^{\,2} = \pp^\perp \subseteq \R^3,
	\end{equation*}
	which is a vector space.
	% \missingfigure{Geo 4/1}
\end{definition}

The name tangent space is natural, because tangents to paths on the sphere naturally lie in this space:

\begin{proposition}
	If $\gamma:[0,1]\to S^{\,2}$ has $\gamma(t_0)=\pp$, then $\gamma\p(t_0) \in T_{\pp}S^{\,2}$ %
\end{proposition}

\begin{proof}
	We have $\gamma(t_0) \cdot \gamma(t_0) = 1$, so differentiating gives $2\,\gamma\p(t_0) \cdot \gamma(t_0) = 0$. Thus $\gamma\p(t_0) \perp \gamma(t_0)$. %
\end{proof}

\begin{definition}
	Points $\xx,-\xx \in S^{\,2}$ are called \emph{antipodal}. Antipodal points are diametrically opposite on the sphere. %
\end{definition}

We now consider some of the structures that we're used to in Euclidean geometry, and how they apply to the sphere. Lines are slightly different to those in $\R^3$:

\begin{definition}
	A line $L\subseteq S^{\,2}$ is $H\cap S^{\,2}$, where $H$ is a two-dimensional linear subspace (a plane) in $\R^3$ that passes through the origin.
\end{definition}

Some properties of lines on the sphere carry over nicely from Euclidean space. For example, the fact that (almost) any two points define a unique line:

\begin{proposition}
	There is a unique line through any two distinct, non antipodal points. %
\end{proposition}

\begin{proof}
	There's a unique plane in $\R^3$ containing any two linearly independent vectors. This generates our unique line. %
\end{proof}

We require that the two points not be antipodal, because otherwise we can define a family on lines of $S^{\,2}$, all from the family of planes in $\R^3$ that contain the line segment which joins them.

A concept that doesn't carry over from Euclidean geometry is that of parallel lines. In spherical geometry, these don't exist:

\begin{proposition}
	Any two distinct lines intersect in two antipodal points. %
\end{proposition}

\begin{proof}
	Any two distinct planes in $\R^3$ intersect in a one-dimensional linear subspace $\left\langle \vv \right\rangle$, which intersects $S^{\,2}$ in $\vv/\norm{\vv}$, $-\vv/\norm{\vv}$. %
\end{proof}

We can also think of spherical lines as circles in Euclidean space, centred at the origin, which have radius $1$.

	\pagebreak

Now we consider direction vectors on the sphere.

\begin{proposition}
	There exists a bijection %
	\begin{equation*}
		\left\{\text{lines $L$ passing through $\pp$}\right\}
		\longleftrightarrow
		\left\{ \vv \in T_{\pp} S^{\,2} : \vv \neq 0\right\} /
		\vv \sim \lambda \vv, \lambda\iR.
	\end{equation*}
\end{proposition}

\begin{proof}
	We construct our bijection as follows: %
	\begin{align*}
		L = H \cap S^{\,2} & \longrightarrow \pp^\perp \cap H = \left\langle \vv \right\rangle, \\
		\left\langle \vv,\pp \right\rangle & \longleftarrow \vv.
	\end{align*}
	This is a two-dimensional space, since $\vv\in\pp^\perp.$
\end{proof}

Our concepts of rays and line segments carry over nicely from Euclidean space:

\begin{definition}
	The \emph{ray} at $\xx$ = $(L,\vv)$ is one such that that $L$ is a line through $\xx$, with direction vector $\vv$ for $L$ at $\xx$ with $\norm{\vv}=1$.
	
	% \missingfigure{Geo 4/2}
	
	The \emph{line segment} from $\pp$ to $\qq$ is the shorter arc of the line joining $\pp$ and $\qq$.

	% \missingfigure{Geo 4/3}
	
	There is no unique line segment from $\pp$ to $\qq$ if $\pp$ and $\qq$ are antipodal.
\end{definition}

Similarly, if we think of angles as arising from our definition of scalar product, then our definition is the obvious one:

\begin{definition}
	If $(L_1,\vv_1)$ and $(L_2,\vv_2)$ are rays at $\xx$, then their \emph{angle} is the Euclidean angle
	\begin{equation*}
		\angle \vv_1,\vv_2 = \cos^{-1}\left( \f{\vv_1 \cdot \vv_2}{\norm{\vv_1} \norm{\vv_2}} \right).
	\end{equation*}
	% \missingfigure{Geo 4/4}
\end{definition}

Finally, we come to our notion of distance. We define it in the obvious way:

\begin{definition}
	If $\pp,\qq$ are non antipodal points on $S^{\,2}$, then the distance between them is given by
	\begin{equation*}
		d(\pp,\qq) = \text{length of line segment from $\pp$ to $\qq$} = \theta,
	\end{equation*}
	where $\theta=\angle \pp,\qq = \cos^{-1}(\pp\cdot\qq)$.

	If $\pp$ and $\qq$ are antipodal; that is, if $\qq=-\pp$, the $d(\pp,\qq) = \theta$.
	% \missingfigure{Geo 4/5}
\end{definition}

Now we need to show that this definition of distance turns the sphere into a metric space, because then a lot of nice properties follow easily.

We need to check the three conditions for a metric:
\begin{enumerate}
	\shortskip
	\item $d(\pp,\qq) = 0 \iff \pp=\qq$ (easy);
	\item $d(\pp,\qq) = d(\qq,\pp)$ (easy);
	\item The triangle inequality: $d(\pp,\qq) + d(\qq,\rr) \geq d(\pp,\rr)$.
\end{enumerate}
As is usually the case, checking the triangle inequality will be the hardest of the three. The best way to check this is to do some spherical trigonometry.

% subsection basics (end)

	\pagebreak

\subsection{Spherical trigonometry} % (fold)
\label{sub:spherical_trigonometry}

First we will need the following lemma:

\begin{lemma}
	If $\va,\vb,\vc \iR^3$, then \label{lem:spher-trig} %
	\begin{enumerate}
		\shortskip
		\item $\left( \va\cross\vc \right) \cdot \left( \vb\cross\vc \right) = \left( \vc\cdot\vc \right) \left( \va\cdot\vb \right) - \left( \va\cdot\vc \right)\left( \vb\cdot\vc \right)$; %
		\item $\left( \va\cross\vc \right) \cross \left( \vb\cross\vc \right) = \left( \left( \va\cross\vb \right) \cdot \vc \right) \vc$. %
	\end{enumerate}
\end{lemma}

\begin{proof}
	We can prove this in generality using suffix notation and the summation convention. Recall from \emph{Vectors \& Matrices} that
	\begin{equation*}
		\left( \va \cross \vb \right)_i = \epsilon_{ijk} a_j b_k
		\qquad \text{and} \qquad
		\epsilon_{ijk} \epsilon_{ilm} = \delta_{jl} \delta_{km} - \delta_{jm} \delta_{kl}.
	\end{equation*}
	With those in hand, we just expand the expressions accordingly:
	\begin{enumerate}
		\item \(\begin{aligned}[t]
			\left( \va \cross \vc \right) \cdot \left( \vb \cross \vc \right)
			&= \left( \va \cross \vc \right)_i \left( \vb \cross \vc \right)_i \\
			&= \epsilon_{ijk} a_j c_k \epsilon_{ilm} b_l c_m \\
			&= \left( \delta_{jl} \delta_{km} - \delta_{jm} \delta_{kl} \right) a_j c_k b_l c_m \\
			&= a_j c_k b_j c_k - a_j c_k b_k c_j \\
			&= \left( \vc\cdot\vc \right)\left( \va\cdot\vb \right) - \left( \va\cdot\vc \right)\left( \vb\cdot\vc \right).
		\end{aligned}\)
		\item \(\begin{aligned}[t]
			\left[ \left( \va\cross\vc \right) \cross \left( \vb\cross\vc \right) \right]_i
			&= \epsilon_{ijk} \left( \va\cross \vc \right)_j \left( \vb\cross\vc \right)_k \\
			&= \epsilon_{ijk} \epsilon_{jlm} a_l c_m \epsilon_{kpq} b_p c_q \\
			&= \epsilon_{kpq} \left( \delta_{kl} \delta_{im} - \delta_{km} \delta_{il} \right) a_l c_m b_p c_q \\
			&= \epsilon_{kpq} a_k c_i b_p c_q - \epsilon_{kpq} a_i c_k b_p c_q \\
			&= \left( \va\cross\vb \right)_q c_q c_i - \left( \vc\cross\vc \right)_p b_p a_i \\
			&= \left[ \left( \va\cross\vb \right) \cdot \vc \right] c_i.
		\end{aligned}\)
	\end{enumerate}
	This proves the lemma, and gives us a lot of the machinery that we need to do spherical geometry.
\end{proof}

% \begin{proof}
% 	[Sketch proof] Both identities are lines in $\va$ and $\vb$. Now, (i) is symmetric under switching $\va$ and $\vb$; (ii) is antisymmetric; and both are symmetric under the operation $(\ii,\jj,\kk) \mapsto (\kk,\ii,\jj)$. %
	
% 	Thus, it is sufficient to check for $(\va,\vb,\vc) = (\ii,\jj,\kk)$. For example, consider $\va=\ii,\vb=\ii$ and $\va=\ii,\vb=\jj$. %
	
% 	For $\va=\ii,\vb=\jj$ and $\vc=x\,\ii+ y\,\jj + z\,\kk$, (i) becomes
% 	\begin{equation*}
% 		\left( \ii\cross\vc \right) \cdot \left( \,\jj\cross \vc \right) \stackrel{?}{=} \left( \vc\cdot\vc \right) \left( \ii\cdot\jj \right) - \left( \ii\cdot\vc \right)\left( \,\jj\cdot\vc \right). %
% 	\end{equation*}
% 	This reduces to
% 	\begin{equation*}
% 		\left( y\,\kk-z\,\jj \right) \cdot \left( x\,\kk+z\,\ii \right) = -xy = 0 - xy,
% 	\end{equation*}
% 	so it holds in this case.
	
% 	For $\va=\ii$, $\vb=\ii$, (i) becomes
% 	\begin{equation*}
% 		\left( \ii\cross\vc \right) \cdot \left( \,\ii\cross \vc \right) \stackrel{?}{=} \left( \vc\cdot\vc \right) \left( \ii\cdot \ii \right) - \left( \ii\cdot\vc \right)^2, %
% 	\end{equation*}
% 	which reduces to
% 	\begin{equation*}
% 		\left\vert y\,\kk+z\,\jj \right\vert^2 = x^2 + y^2 + z^2 - x^2,
% 	\end{equation*}
% 	which is true. We can do similar calculations for (ii). We could alternatively prove both identities using suffix notation. %
% \end{proof}

\lecturemarker{5}{4 Feb}
To use this lemma properly, we need to make sure we know what scalar and vector products mean in $S^{\,2}$. The scalar product is the same as in $\R^3$, and the vector (or cross) product is only slightly different:

\begin{definition}
	Let $L\subset S^{\,2} \cap H$ be a ray passing through $\xx$ with unit direction vector $\tt$, with $\xx$ perpendicular to $\tt$. If $\xx,\tt\in H$, then the \emph{cross product} $\xx \cross \tt$ is the unit vector perpendicular to $H$.
\end{definition}

Now if we have two rays through $\xx$ with directions $\tt_1, \tt_2$, then we have already defined the angle $\theta$ between them to satisfy
\begin{equation*}
	\cos \theta = \tt_1 \cdot \tt_2.
\end{equation*}
Now let $\nn_i = \xx \cross \tt_i$ be the unit normal to $H_i$. Then
\begin{align*}
	\nn_1 \cdot \nn_2
	&= \left( \xx\cross\tt_1 \right) \cdot \left( \xx\cross\tt_2 \right) \\
	&= \left( -1 \right)^2 \left[ \left( \tt_1\cdot\tt_2 \right)\left( \xx\cdot\xx \right) - \left( \tt_1 \cdot \xx \right) \left( \tt_2\cdot\xx \right) \right] \tag{by (i) in the lemma} \\ %
	&= \left( \tt_1 \cdot \tt_2 \right)1 - 0 = \tt_1 \cdot \tt_2.
\end{align*}
Hence $\nn_1 \cdot \nn_2 = \cos\theta$.

This makes some sort of intuitive sense. If we consider the plane on the page, $T_{\xx} S^{\,2}$, then the unit normals are just rotations by $\pi/2$. Then clearly the angle $\theta$ is preserved.

% \missingfigure{Geo 5/1}

	\pagebreak

Next we need to consider what triangles mean on a sphere.

Now suppose we have a spherical triangle with vertices $\AA,\BB,\CC\in S^{\,2}$, with no two antipodal, sides of length $a,b,c$, and angles $\alpha,\beta,\gamma$.

Since drawing spherical triangles in three dimensions is often difficult without losing clarity, we often use two-dimensional representations of the form below. This captures much of the information about the triangle, but it significantly easier to draw and understand. The curved arcs represent the spherical lines that define the triangle.

\begin{center}
\begin{tikzpicture}[scale=3.5]
	
	\fill [fill=\shadeblack] (1,0) arc (60:0:1) arc (-60:-120:1) arc (180:120:1);

	\draw (0,0) arc (120:60:1)
					node [above=2pt] {$\CC$} arc (60:51:1)					% point C
					arc (-30:-150:0.15)
						arc (-150:-90:0.15) node [below] {$\gamma$}
						arc (-90:-30:0.15) arc (51:30:1)					% angle gamma
					node [above right] {$a$} arc (30:-60:1)					% distance a
				arc (240:180:1)
					node [below left=2pt] {$\AA$} arc (180:171:1)			% point A
					arc (90:-30:0.15)
						arc (-30:30:0.15) node [above right] {$\beta$}
						arc (30:90:0.15) arc (171:150:1)					% angle beta
					node [above left] {$b$} arc (150:60:1)					% distance b
				arc (0:-60:1)
					node [below right=2pt] {$\BB$} arc (-60:-69:1)			% point B
					arc (210:90:0.15)
						arc (90:150:0.15) node [above left] {$\alpha$}
						arc (150:210:0.15) arc (-69:-90:1)					% angle alpha
					node [below] {$c$} arc(-90:-180:1);						% distance x

	\draw [very thick] (1,0) arc (60:0:1) arc (-60:-120:1) arc (-180:-240:1);

\end{tikzpicture}
\end{center}

In particular, it's worth noting that $\alpha+\beta+\gamma>\pi$, as opposed to triangles in Euclidean space. We will explore the properties of angles of a spherical triangle in more detail later.

% \missingfigure{Geo 5/2}

Since the sides are given by arcs on a unit sphere, their lengths are just the angles that they span. Thus:
\begin{equation*}
	\cos a = \BB\cdot\CC, \qquad \cos b = \AA\cdot\CC, \qquad \cos c = \AA\cdot\BB.
\end{equation*}
Note that when we say $a$, $b$ and $c$ here, we really do mean the lengths, not the angles, since these lengths are actually angles.

% \missingfigure{Geo 5/3}

Now, if $\tt$ is the direction vector for the line segment pointing from $\AA$ to $\BB$, then
\begin{equation*}
	\AA\cross\BB = \sin c\,\nn_c = \sin c\left( \AA\cross\tt \right),
\end{equation*}
where $\nn_c$ is the unit normal to $\left\langle \AA,\BB \right\rangle$.

% \missingfigure{Geo 5/4}

	\pagebreak

\begin{proposition}
	Suppose we have a triangle on $S^{\,2}$ as described above. Then we have the following two rules, which are very similar to rules for triangles in $\Rn$. %
	\begin{enumerate}
		\item Cosine rule:
		\begin{equation*}
			\cos a = \cos b \cos c + \sin b \sin c \cos \alpha.
		\end{equation*}
		\item Sine rule:
		\begin{equation*}
			\f{\sin \alpha}{\sin a} = \f{\sin\beta}{\sin b} = \f{\sin\gamma}{\sin c}.
		\end{equation*}
	\end{enumerate}
\end{proposition}

\begin{proof}
	These follow very nicely from the machinery we derived in lemma~\ref{lem:spher-trig}. Consider:
	\begin{align*}
		\left( \AA\cross\BB \right) \cdot \left( \AA\cross\CC \right)
		&= \sin c \sin b \cos \alpha \\
		% \intertext{But this is also equal, by the lemma, to}
		&= \left( \BB\cdot \CC \right)\left( \AA\cdot\AA \right) - \left( \AA\cdot\BB \right)\left( \AA\cdot\CC \right) \\ %
		&= \left( \cos a \right)\cdot 1 - \cos b \cos c.
	\end{align*}
	Rearranging these gives the cosine rule.
	
	Now consider
	\begin{align*}
		\left( \AA\cross\BB \right) \cross \left( \AA\cross\CC \right)
		&= \sin c \sin b \left( \nn_c \cross \nn_b \right) \\
		&= \sin c \sin b \sin \alpha\,\AA. \\
		% \intertext{By the , this is also equal to}
		&= \left( \left( \AA\cross\BB \right)\cdot \CC \right) \AA \\
		&\Rightarrow \left( \AA\cross\BB \right)\cdot\CC = \sin c \sin b \sin\alpha.
	\end{align*}
	Now we know that this triple product is invariant under cyclic permutations, and so
	\begin{align*}
		\left( \AA\cross\BB \right)\cdot \CC &= \left( \CC\cross\AA \right)\cdot\BB \\
		\sin c\sin b\sin\alpha &= \sin b\sin a\sin\gamma.
	\end{align*}
	This second relation gives us
	\begin{equation*}
		\f{\sin\gamma}{\sin c} = \f{\sin\alpha}{\sin b},
	\end{equation*}
	and the rest of the rule follows by symmetry.
\end{proof}

	\vspace{3pt}

\begin{note}
	Suppose you're standing on the surface of the Earth. Technically, the Earth is approximately a sphere, but standing on its surface, the distances involved are so small that you might expect to be able to do plane geometry, and this turns out to be roughly right. We have $a,b,c\ll 1$. $\sin a \approx a$ and $\cos a \approx 1-a^2/2$. %

	The sine rule on spheres obviously reduces to the sine rule in the Euclidean plane. The cosine rule becomes
	\begin{equation*}
		(1-a^2/2) \approx (1-b^2/2) (1-c^2/2) + bc \cos \alpha,
	\end{equation*}
	which can be rearranged to give
	\begin{equation*}
		a^2 = b^2 + c^2 - 2bc \cos\alpha,
	\end{equation*}
	which is the cosine law in the plane.
\end{note}

% subsection spherical_trigonometry (end)

	\pagebreak

\subsection{Distance (again)} % (fold)
\label{sub:distance_again_}

Finally, we can return to where we started: trying to prove that our notion of distance defined a metric on the sphere, which required us to prove the triangle inequality. With a better understanding of spherical trigonometry, we can proceed.

\begin{corollary}
	[Triangle inequality] For points $\AA,\BB,\CC\in S^{\,2}$ and the distance function $d(\cdot,\cdot)$ as defined in section~\ref{sub:basics}, we have
	\begin{equation*}
		d(\BB,\AA) + d(\AA,\CC) \geq d(\BB,\CC),
	\end{equation*}
	with equality if and only if $\AA$ lies on the line segment $\BB\CC$ or $\BB$ and $\CC$ are antipodal. %
\end{corollary}

\begin{proof}
	Using the notation established in the previous section, we want to show that $c+b\geq a$. We know that
	\begin{align*}
		\cos\alpha
		&= \cos b \cos c + \sin b \sin c \cos \alpha \\
		&\geq \cos b \cos c - \sin b \sin c = \cos(b+c).
	\end{align*}
	Since $\cos$ is decreasing on $[0,\pi]$, we have $a\leq b+c$. %
\end{proof}

So now we have two metrics: the Euclidean metric $d_E$ on $\R^3$, and the spherical metric $d_S$ on $S^{\,2}$. It's natural to ask the following question:

If $\gamma:[0,1] \to S^{\,2} \subset \R^3$ is a path, and we define
\begin{itemize}
	\shortskip
	\item[] $L^E(\gamma) = \text{length of $\gamma$ with respect to the Euclidean metric on $\R^3$}$
	\item[] $L^S(\gamma) = \text{length of $\gamma$ with respect to the spherical metric}$
\end{itemize}
Are these two distances the same? It turns out that they are, which justifies our choice of spherical metric.

\begin{proposition}
	$L^E(\gamma) = L^S(\gamma)$. %
\end{proposition}

\begin{proof}
	Let $\pp$ and $\qq$ be two points on $S^{\,2}$, with an angle of $2\theta$ between their position vectors

	\begin{center}
		\begin{tikzpicture}[scale=2]
			\draw (0,0) circle (1);

			\draw (0,0) -- (-0.5,0.866) node [above left] {$\pp$};
			\draw (0,0) -- (0.5,0.866) node [above right] {$\qq$};

			\foreach \s in {-0.5, 0.5}
			{
				\draw (\s,0.866) node {$\bullet$};
			}

			\draw (0,0) node {$\bullet$};

			\draw [dashed] (0,0) -- (0,0.866);
			\draw [dashed] (-0.5,0.866) -- (0.5,0.866);
			\draw (0,0.3) arc (90:120:0.3);
			\draw (-0.12,0.28) node [above] {$\theta$};
		\end{tikzpicture}
	\end{center}

	Simple plane geometry tells us that $d_S(\pp,\qq) = 2\theta$ and $d_E(\pp,\qq) = 2\sin\theta$. Consider
	\begin{equation*}
		\lim_{\theta\to0} \f{d_S(\pp,\qq)}{d_E(\pp,\qq)} = \lim_{\theta \to 0} \f{\theta}{\sin\theta} = 1.
	\end{equation*}
	Given $\epsilon>0$, there is some $\delta_1>0$ such that if $d_E(\pp,\qq) < \delta_1$,
	\begin{equation*}
		d_E(\pp,\qq) \leq d_S(\pp,\qq) \leq \left( 1+\epsilon \right) d_E(\pp,\qq).
		\tag{$*$}
	\end{equation*}
	Since $\gamma$ is uniformly continuous, there is some $\delta_2>0$ such that
	\begin{equation*}
		d_E(\gamma(t),\gamma(s)) < \delta_1 \text{ whenever } \left\vert t-s \right\vert<\delta_2.
	\end{equation*}
	Now we consider the set of dissections
	\begin{equation*}
		D_\delta = \left\{A = \left\{0=t_0 < t_1 < \ldots < t_n = 1\right\} \mid t_i - t_{i-1} < \delta \;\forall i\right\} %
	\end{equation*}
	and this means we can write
	\begin{equation*}
		L(\gamma) = \sup_A L_A(\gamma) = \sup_A \left\{L_A(\gamma) \mid A\in D_\delta\right\}.
	\end{equation*}
	Then $(*)$ implies that if $A\in D_{\delta_2}$, then
	\begin{equation*}
		L_A^E(\gamma) \leq L_A^S(\gamma) \leq \left( 1+\epsilon \right) L_A^E(\gamma). %
	\end{equation*}
	Finally, for all $\epsilon>0$, we have
	\begin{equation*}
		L^E(\gamma) \leq L^S(\gamma) \leq \left( 1+\epsilon \right) L^E(\gamma),
	\end{equation*}
	and so $L^E(\gamma) = L^S(\gamma)$. %
\end{proof}

This is extremely useful, because it means we can use whichever metric is more convenient.

% subsection distance_again_ (end)

\subsection{Isometries} % (fold)
\label{sub:spherical_isometries}

Now we consider the isometries of the sphere. This will turn out to be easier than when we were working in $\Rn$. Let's start by considering orthogonal matrices:

\begin{example}
	Suppose $O\in O(3)$. If $\AA,\BB\in S^{\,2}$, then
	\begin{equation*}
		d(\AA,\BB) = \cos^{-1} (\AA\cdot\BB) = \cos^{-1}(O\AA \cdot O\BB) = d(O\AA, O\BB),
	\end{equation*}
	and so $O\in\Isom(S^{\,2})$. %
\end{example}

It turns out that these actually define all the isometries of the sphere:

\begin{theorem}
	$\Isom(S^{\,2}) = O(3)$. %
\end{theorem}

Remember how we showed this in the Euclidean case. We proved it with a series of lemmas. First we showed that if an isometry fixes the origin and the standard basis, then it is the identity. Again:

\begin{lemma}
	If $\phi\in\Isom(S^{\,2})$, $\phi(\ee_i) = \ee_i$ for $i=1,2,3$, then $\phi=\id_{S^{\,2}}$. %
\end{lemma}

\begin{proof}
	Let $\xx=(x_1,x_2,x_3)$, and $\phi(\xx)=\yy=(y_1,y_2,y_3)$. Then
	\begin{equation*}
		x_i
		= \xx\cdot\ee_i
		= \cos d(\xx,\ee_i)
		= \cos d(\phi(\xx), \phi(\ee_i)) = \cos(d(\yy,\ee_i))
		= \yy\cdot\ee_i = y_i.
	\end{equation*}
	Thus $x_i=y_i$ for $i=1,2,3$, and hence $\xx=\phi(\xx)$. %
\end{proof}

	\pagebreak

Notice that this was easier than the Euclidean case. Since we don't have translations on $S^{\,2}$, that's all we need to prove the theorem:

\begin{proof}
	[Proof of theorem] If $\phi\in\Isom(S^{\,2})$, then let $\vv_i = \phi(\ee_i)$. Then
	\begin{equation*}
		\vv_i \cdot \vv_j = \cos d(\vv_i,\vv_j) = \cos d(\ee_i,\ee_j) = \ee_i \cdot \ee_j = \delta_{ij}. %
	\end{equation*}
	Thus we can construct a matrix $O=(\vv_1,\vv_2,\vv_3) \in O(3)$. Thus $O\in\Isom(S^{\,2})$, with $O(\ee_i) = \vv_i$.

	Now $O^{-1}\circ\phi \in \Isom(S^{\,2})$, with $(O^{-1}\circ \phi)(\ee_i) = O^{-1}(\vv_i) = \ee_i$, and so $O^{-1} \circ \phi = \id_{S^{\,2}}$ by the lemma. Hence $\phi=O$. %
\end{proof}

\lecturemarker{6}{6 Feb}
Now we know what the isometries of $S^{\,2}$ are, let's consider how their properties relate to those in Euclidean space. Suppose $L=S^{\,2}\cap H$ is a line through $\xx$ with unit direction $\tt\in T_\xx S^{\,2}$.

If $O\in O(3)$, then $OL = S^{\,2} \cap OH$ is a line through $O\xx$ with unit direction $O\tt$, since $O\tt \in OH$ and $O\tt \cdot O\xx = \tt \cdot \xx = 0$. Thus $O\tt \in T_{O\xx} S^{\,2}$.

From this observation we draw the immediate corollary:

\vspace{2pt}

\begin{corollary}
	Isometries of $S^{\,2}$ preserve angles. %
\end{corollary}

\begin{proof}
	If $R_1,R_2$ are rays at $\xx$ with direction vectors $\tt_1, \tt_2$, then $OR_1, OR_2$ have direction vectors $O\tt_1, O\tt_2$, and %
	\begin{equation*}
		\cos \angle R_1, R_2 = \tt_1 \cdot \tt_2 = O\tt_1 \cdot O\tt_2 = \cos \angle OR_1, OR_2,
	\end{equation*}
	since $O\in O(3)$. Thus angles are preserved.
\end{proof}

Another concept we can bring over from our work in $\R^2$ is that of orthogonal frames:

\begin{definition}
	An \emph{orthogonal} frame at $\xx\in S^{\,2}$ is an ordered pair of unit tangent vectors $(\tt_1,\tt_2)$ with $\tt_i \in T_\xx S^{\,2}$ and $\tt_1 \perp \tt_2$. %
	
	% \missingfigure{Geo 6/1}

	The \emph{standard frame} $F_0$ at $(0,0,1)$ is $(\ee_1,\ee_2)$. %
\end{definition}

Our results from the Euclidean plane carry over naturally:

\begin{corollary}
	If $F_1=(\tt_1^1,\tt_2^1)$ is an orthogonal frame at $\xx_1$, and $F_2=(\tt_1^2, \tt_2^2)$ is an orthogonal frame at $\xx_2$, then there is a unique $O\in\Isom(S^{\,2})$ with $O(F_1) = F_2$. %
\end{corollary}

\begin{proof}
	Observe that $\xx_1, \tt_1, \tt_2$ is an orthonormal basis for $\R^3$, so we construct $O_1 = (\xx_1,\tt_1,\tt_2) \in O(3)$. %

	Then $O_1(F_0)=F_1$. Define $O_2$ similarly. Then $(O_2 \circ O_1^{-1})(F_1) = O_2 (F_0) = F_2$.

	Uniqueness is immediate, since an element of $O(3)$ is determined by its action on the basis $\xx_1, \tt_1, \tt_2$. %
\end{proof}

% This property is very important, and we call spaces that satisfy it \emph{homogeneous spaces}. Both the Euclidean plane and the sphere are homogeneous spaces.\todo{What is the exact defn of a homogneous space?}

% subsection spherical_isometries (end)

	\pagebreak

\subsection{Angle defect} % (fold)
\label{sub:angle_defect}

Now we come to the first beautiful theorem of the course, involving the previously discussed angle formula for triangles.

\begin{definition}
	If $\triangle ABC$ is a spherical triangle with angles $\alpha,\beta,\gamma$, then the \emph{angle defect} of $ABC$ is defined as
	\begin{equation*}
		\defect(ABC) = \alpha+\beta+\gamma - \pi.
	\end{equation*} %
\end{definition}

\begin{theorem}
	For a triangle as described above,
	\begin{equation*}
		\defect(ABC) = \Area(\triangle ABC).
	\end{equation*}
\end{theorem}

\begin{proof}
	First let's consider two lines on the sphere. A pair of lines divide $S^{\,2}$ into four spherical sectors, and without loss of generality, suppose they intersect at the poles. (Compare them to slices of an orange.) Looking downward: %
	
	\begin{center}
		\begin{tikzpicture} [scale=1.5]
			\fill [fill=\shadeblack] (0,0) -- (0.5,-0.866025404) arc (-60:-120:1) -- cycle;

			\draw (0,0) circle (1);

			\draw [thick] (-0.5,0.866025404) -- (0.5,-0.866025404);
			\draw [thick] (-0.5,-0.866025404) -- (0.5,0.866025404);

			\draw [very thick] (0,0) -- (0.5,-0.866025404) arc (-60:-120:1) -- cycle;

			\draw (0,0) -- (0.1,-0.173205081) arc (-60:-120:0.2) -- cycle;
			\draw (0,-0.17) node [below=3pt] {$\Theta$};
		\end{tikzpicture}
	\end{center}

	Let $S_\Theta$ be the sector subtended by angle $\Theta$. Now $\Area(S^{\,2}) = 4\pi$, so $\Area(S_\Theta) = 2\Theta$, either by considering it as a proportion of the surface area of the whole sphere, or by considering the area integral %
	\begin{equation*}
		\Area(S_\Theta) = \int_{\theta=0}^{\Theta} \int_{\phi=-\pi/2}^{\pi/2} \sin\phi \dif{\theta} \dif{\phi}.	
	\end{equation*}
	A third line divides $S^{\,2}$ into an ``octahedron'' (strictly speaking, the projection of an octahedron onto a sphere). %
	
	% \missingfigure{Geo 6/3 -- these are an 90 degree angles, but we could easily  move them around}

	Consider the triangle $\triangle ABC$. This allows us to divide $S^{\,2}$ into two regions, $R$ and $-R$, where $R$ is $\triangle ABC$ and the three faces adjacent to $ABC$. We note that $-R$ is the image of $R$ under $\xx\mapsto-\xx$, so $\Area(R)=\Area(-R)$. Thus $\Area(R)=2\pi$.
	
	We label the triangles as follows (letting $\triangle 1 = \triangle ABC$):

	\begin{center}
	\begin{tikzpicture}[scale=2.5]
		
		\fill [fill=\shadeblack] (1,0) arc (60:0:1) arc (-60:-120:1) arc (180:120:1);
	
		\draw (0,0) arc (120:60:1)
						node [above=2pt] {$C$} arc (60:51:1)					% point C
						arc (-30:-150:0.15)
							arc (-150:-90:0.15) node [below] {$\gamma$}
							arc (-90:-30:0.15) arc (51:30:1)					% angle gamma
						node [above right] {} arc (30:-60:1)
					arc (240:180:1)
						node [below left=2pt] {$A$} arc (180:171:1)				% point A
						arc (90:-30:0.15)
							arc (-30:30:0.15) node [above right] {$\beta$}
							arc (30:90:0.15) arc (171:150:1)					% angle beta
						node [above left] {} arc (150:60:1)
					arc (0:-60:1)
						node [below right=2pt] {$B$} arc (-60:-69:1)			% point B
						arc (210:90:0.15)
							arc (90:150:0.15) node [above left] {$\alpha$}
							arc (150:210:0.15) arc (-69:-90:1)					% angle alpha
						node [below] {} arc(-90:-180:1);
	
		\draw [very thick] (1,0) arc (60:0:1) arc (-60:-120:1) arc (-180:-240:1);
	
		\draw (0.433012702,-0.25) node {$\triangle 2$};
		\draw (1.566987298,-0.25) node {$\triangle 3$};
		\draw (1,-1.25) node {$\triangle 4$};

	\end{tikzpicture}
	\end{center}

	Now we notice that pairs of triangles form spherical sectors:
	\begin{align*}
		1\cup 4 &= \text{spherical sector with $\angle\gamma$}, \\
		1\cup 3 &= \text{spherical sector with $\angle\alpha$}, \\
		1\cup 2 &= \text{spherical sector with $\angle\beta$}.
	\end{align*}
	Then using the previously discussed formula for the area of a spherical sector, we have
	\begin{align*}
		\Area(1\cup 4) &= 2\gamma, \\
		\Area(1\cup 3) &= 2\alpha, \\
		\Area(1\cup 2) &= 2\beta.
	\end{align*}
	We've already seen that $\Area(R)=2\pi$, and $R=1 \cup 2 \cup 3 \cup 4$. Thus
	\begin{equation*}
		2\pi
		= \Area(1 \cup 2 \cup 3 \cup 4)
		= 2\gamma + 2\alpha + 2\beta - 2\Area(1).
	\end{equation*}
	But $\Area(1)=\Area(\triangle ABC)$, and so rearranging just gives
	\begin{equation*}
		\Area(\triangle ABC) = \alpha+\beta+\gamma-\pi = \defect(ABC). \qedhere
	\end{equation*}
\end{proof}

Now let's look at an application of this.

\begin{definition}
	A \emph{spherical polyhedron $P$} is
	\begin{enumerate}
		\shortskip
		\item A set of points (vertices) in $S^{\,2}$;
		\item A set of line segments (edges) in $S^{\,2}$ which are disjoint except at vertices;
		\item Faces of $P$, the connected components $S^{\,2}-\{\text{edges}\}$.
		\item Every vertex lies on an edge.
	\end{enumerate}
\end{definition}

\begin{theorem}
	[Euler's formula] If $P$ is a spherical polyhedron with $V$ vertices, $E$ edges and $F$ faces, then %
	\begin{equation*}
		V-E+F=2.
	\end{equation*}
\end{theorem}

\begin{proof}
	Suppose some face has more than three sides. Then we can subdivide the face to make a new $P\p$ with $V\p=V$ vertices, $E\p=E+1$ edges and $F\p=F+1$ faces. Thus $V\p-E\p+F\p = V-E+F$. %
	
	\begin{center}
		\begin{tikzpicture}
			\draw [thick] (0,0) -- (2,1) -- (3,-2) -- (-0.5,-3) -- (-1,-1.8) -- cycle;
			\draw [dashed] (0,0) -- (3,-2);
		\end{tikzpicture}
	\end{center}

	Thus, it is sufficient to prove Euler's formula for the subdivided shape. After repeated subdividing, we can assume that all faces are triangles. %

	Every triangle has three edges, and every edge borders two faces. Thus
	\begin{equation*}
		3F=2E \text{ or } E=\tf{3}{2}F. \tag{$*$}
	\end{equation*}
	Now consider the sum of the angles:
	\begin{align*}
		S
		&= \text{sum of every angle in every face of $P$} \\
		&= \text{sum of every angle at every vertex of $P$}.
	\end{align*}
	Working from the face-based definition, we have:
	\begin{equation*}
		S
		= \sum_{\text{faces $f$}} \sum_{\substack{\text{angles $\theta_i$} \\ \text{in $f$}}}
		= \sum_{\text{faces $f$}} \left[ \pi+\Area(f) \right]
		=\pi F + \Area(S^{\,2}) = \pi F + 4\pi.
	\end{equation*}
	Alternatively, using the vertex definition, we have
	\begin{equation*}
		S
		=\sum_{\text{vertices $v_i$}} \sum_{\substack{\text{angles $\theta_i$} \\ \text{at $v$}}}
		= \sum_v 2\pi = 2\pi V
	\end{equation*}
	Combining these, we have
	\begin{equation*}
		2\pi V = \pi F + 4\pi \implies F=2V-4. \tag{$**$}
	\end{equation*}
	Combining equations~($*$) and ($**$), we have
	\begin{equation*}
		V-E+F = V-\tf{1}{2}F = V-\tf{1}{2}\left( 2V-4 \right) = 2. \qedhere
	\end{equation*}
\end{proof}

Now let's recast this in a form we might be slightly more familiar with:

\begin{definition}
	A \emph{convex Euclidean polyhedron} is a convex bounded subset of $\R^3$ bounded by a finite number of planes. That is, %
	\begin{equation*}
		P = \bigcap_{i=1}^n X_i,
	\end{equation*}
	where $X_i = \left\{\xx\iR^3 : \xx\cdot \vv_i \leq c_i\right\}$.
\end{definition}

\begin{corollary}
	If $P$ is a convex Euclidean polyhedrom with $V$ vertices, $E$ edges and $F$ faces, then $V-E+F=2$. %
\end{corollary}

\begin{proof}
	After translation, we can assume that the origin is inside $P$. Then consider the map which projects $P$ on to the surface of the sphere.
	\begin{equation*}
		\fullfunction{\pi}{\R^3-O}{S^{\,2}}{\vv}{\vv/\norm{\vv}}
	\end{equation*}
	The image of $P$ is a spherical polyhedron $P\p$ with $V$ vertices, $E$ edges and $F$ faces. %

	Note that an edge of $P\p$ is a Euclidean line segment. It projects to the spherical line segment lying on $H$, where $H$ is the plane spanned by $O$ on $L$. %
\end{proof}

% subsection angle_defect (end)

	\pagebreak

\subsection{Topology of surfaces} % (fold)
\label{sub:topology_of_surfaces}

\newcommand{\biggarrow}{\draw [decoration={markings,mark=at position 1 with {\arrow[scale=2]{>>}}},
    		   postaction={decorate},
    		   shorten >=0.4pt]}

\lecturemarker{7}{11 Feb}
\begin{definition}
	A \emph{surface} is a metric space $S$ which is locally homeomorphic to $\R^2$; that is, for every $\xx\in S$, there's an open $U\ni x$ and a homeomorphism
	\begin{equation*}
		\phi_U: U\to B_0(1) = \left\{ \xx\iR^2 : \vvert{\xx} < 1 \right\}.
	\end{equation*}
\end{definition}

Notice that there's nothing special about $2$ in this definition. If we replace $2$ by $n$, then we recover the definition of an \emph{$n$-dimensional manifold}. We will study these objects further in Part~II. 

\begin{examples}
\mbox{}
\begin{enumerate}
	\item Trivially, $\R^2$.
	\item The sphere $S^{\,2}$. Given $\xx\in S^{\,2}$, take $U_\xx$ to be the open hemisphere which contains $\xx$. Let $\phi_{U_\xx}$ be the projection on to the plane which cuts out the hemisphere.

	\begin{center}
		\begin{tikzpicture} [scale=1.8]

			\fill [\shadeblack] (0,0) -- (50:1) arc (50:230:1) -- cycle;

			\draw (0,0) ++ (140:0.45) node {$U_\xx$};

			\draw [thick] (0,0) circle (1);
			\draw (0,0) node {$\bullet$};

			\draw (0,0) ++ (140:1) node {$\bullet$};
			\draw (0,0) ++ (140:1) node [above left] {$\xx$};

			\draw [dashed] (0,0) -- (50:1.5);
			\draw [dashed] (0,0) -- (-130:1.5);
		\end{tikzpicture}
	\end{center}

	In the diagram above, we consider a cross-section of the sphere. The hemisphere containing $\xx$ is shaded, and we project onto the dashed line (the plane which removes $U_\xx$ from $S^{\,2}$). We will see a form of this later, when we discuss \emph{stereographic projections.}
	\item The cylinder $S^{\,1} \times \R = (\R/\Z) \times \R = \R^2/\Z$, where $\Z\cong \left\langle \left( 1,0 \right) \right\rangle \subset \R^2$.

	\begin{center} % Geo 7/1 
		\begin{tikzpicture}

			\foreach \s in {0,1} {
				\bigarrow (-0.7+\s*1.4,-2) node [below=2pt] {$\s$} -- ++ (0,2.8);
				\draw (-0.7+\s*1.4,-2) -- ++ (0,4);
			}

			\draw [dashed] (-0.7,0) -- (0.7,0);
			\biggarrow (-0.2,0) -- (0,0);

			\foreach \s in {-1,1} {\draw (3.2,\s*1.8) circle (0.8 and 0.2);}

			\biggarrow (3.5,1.61) -- (3.6,1.615);

			\fill [white] (2,-1.8) rectangle (4.1,-1.5);
			\draw [dashed] (3.2,-1.8) circle (0.8 and 0.2);

			\draw (3.2,-2) -- (3.2,1.6);

			\foreach \s in {2.4,4} {\draw (\s,1.8) -- (\s,-1.8);}

		\end{tikzpicture}
	\end{center}

	The diagram above shows that the cylinder can also be thought of as $[0,1] \times \R$. We take a pair of infinite lines at $0$ and $1$ (left), and we wrap them around until they meet, and this is the infinite cylinder (right). The interval $[0,1]$ is mapped to a circle, which we recover by taking a cross-section of the cylinder.

		\pagebreak

	\item Möbius band, $M=[0,1] \times [-1/1] / \sim$, where $(0,y) \sim (1,-y)$.

	\vspace{6pt}

	% Preamble for the Möbius strip
	% Taken from Möbius.tex

	% one third of the Moebius Strip
	%: \strip{<angle>}
	\newcommand{\strip}[1]{%
	\shadedraw[thick,top color=white,bottom color=gray, rotate=#1]
	 (0:2.8453) ++ (-30:1.5359) arc (60:0:2)
	 -- ++  (90:5) arc (0:60:2) -- ++ (150:3) arc (60:120:2) 
	 -- ++ (210:5) arc (120:60:2) -- cycle;}

	%: \MoebiusStrip{<text1>}{<text2>}{<text3>}
	\newcommand{\MoebiusStrip}[3]{%
	\begin{scope} [transform shape]
		\strip{0}
		\strip{120}
		\strip{-120}
		\draw (-60:3.5) node[scale=6,rotate=30] {#1};
		\draw (180:3.5) node[scale=4,rotate=-90]{#3};
		% redraw the first strip after clipping
		\clip (-1.4,2.4)--(-.3,6.1)--(1.3,6.1)--(5.3,3.7)--(5.3,-2.7)--cycle;
		\strip{0}
		\draw (60:3.5) node [gray,xscale=-4,yscale=4,rotate=30]{#2};
	\end{scope}}

	% \include{Möbius.tex}
	\begin{minipage}{0.55\textwidth}
		\centering
		\begin{tikzpicture}[xscale=2,yscale=2]
			\draw (-1,0) rectangle (1,1);

			\foreach \s in {-1,1} {\bigarrow (\s,0) -- (\s,0.55);}

			\foreach \s/\t/\u in {-1/0/below left, -1/1/above left, 1/0/below right, 1/1/above right}
			{\draw (\s,\t) node [\u] {$(\s,\t)$};}

		\end{tikzpicture}
	\end{minipage}
	\hspace{0.3cm}
	\begin{minipage}{0.35\textwidth}
		\begin{tikzpicture} [rotate=22,scale=0.25]
			\MoebiusStrip{}{}{}
		\end{tikzpicture}
	\end{minipage}

	\vspace{6pt}

	Traditionally we make a Möbius band by taking a strip of paper, twisting one end and glueing the ends together. Geometrically, we take the rectangle $[0,1] \times [-1/1]$ with a specified orientation, and we join the ends together in such a way that preserves orientation.

	In the diagram above, we have included several arrows to better illustrate how orientation is preserved.

	\emph{The TikZ code for the shaded Möbius strip was written by Jacques~Duma and Gerard~Tisseau, published online at \href{http://math.et.info.free.fr/TikZ/index.html}{http://math.et.info.free.fr/TikZ/index.html}.}
	
	\item The torus $T^2 = S^{\,1} \times S^{\,1} = (\R/\Z) \times (\R/\Z) = \R^2/\Z^2$.

	Constructing a torus from elementary geometry is slightly, but not significantly, more difficult than anything we've done so far. First consider the rectangle $\left[ 0,1 \right]^2$, with a clockwise orientation:

	\begin{center} % Geo 7/3 
		\begin{tikzpicture}[scale=2.25]

			% Square

			\draw (0,0) rectangle (1,1);

			\foreach \s/\t/\u in {0/0/below left, 0/1/above left, 1/0/below right, 1/1/above right}
			{\draw (\s,\t) node [\u] {$(\s,\t)$};}

			\foreach \s in {0,1} {
				\bigarrow (0,\s) -- (0.5,\s);
				\biggarrow (\s,0) -- (\s,0.6);
			}

			% Cylinder

			\foreach \s in {-1,1} {\draw [thick] (3,0.5+\s*0.9) circle (0.4 and 0.1);}

			\bigarrow [thick] (3.15,1.31) -- (3.2,1.315);

			\fill [white] (2.5,-0.4) rectangle (3.5,-0.25);
			\draw [dashed, thick] (3,-0.4) circle (0.4 and 0.1);

			\draw (3,1.3) -- (3,-0.5);

			\foreach \s in {2.6,3.4} {\draw (\s,1.4) -- (\s,-0.4);}

			\bigarrow [thick] (2.85,-0.49) -- (2.8,-0.485);
		\end{tikzpicture}
	\end{center}

	We wrap this around to construct a cylinder, not unlike example~(iii). However, this is finite in both dimensions. Notice that the orientation in the two circular faces go in the opposite directions.

	% We say that $[0,1] \times [0,1]$ is a \emph{fundamental domain} factorisation of $\Z^2$; that is, it’s closed, and every point in $\R^2$ is equivalent to some point in it. % (Consider that we say in \emph{Numbers \& Sets} that $[0,1]$ and $\R$ have the same cardinality; it intuitively follows that there is an equivalence between $\left[ 0,1 \right]^2$ and $\R^2$.)

	\begin{center}
		\begin{tikzpicture} [scale=0.28, yscale=0.5]

			\draw [thick] (0,-4.8) circle (1 and 4.2);
			\fill [white] (0,-0) rectangle (-2,-9);
			\draw [thick, dashed] (0,-4.8) circle (1 and 4.2);

			\draw [thick] (0,0) circle (9.7 and 9);
			\draw (0,0.6) circle (8.0833 and 6);

			\draw (0,3.3) .. controls (3,3.3) and (5,2) .. (5.4,1.2);
			\draw (0,-0.5) .. controls (3,-0.5) and (5,0.5) .. (5.4,1.2);

			\draw (0,3.3) .. controls (-3,3.3) and (-5,2) .. (-5.4,1.2);
			\draw (0,-0.5) .. controls (-3,-0.5) and (-5,0.5) .. (-5.4,1.2);

			\draw [thin, gray] (5.4,1.2) .. controls (5.7,1.5) and (6.2,2.9) .. (7.5,3.3);
			\draw [thin, gray] (-5.4,1.2) .. controls (-5.7,1.5) and (-6.2,2.9) .. (-7.5,3.3);
		\end{tikzpicture}
	\end{center}

	Then we wrap the two ends of the cylinder around to make a torus, or a doughnut shape.

	There's nothing special about the $2$ in the definitions of $S^{\,2}$ and $T^{\,2}$. Both constructions are dimension independent; we can just as easily, for example, define $T^{\,7}$ embedded in $\R^7$.

	% 	\pagebreak

	% \item $\R P^2$, the real projective plane, which is given by $S^{\,2}/\sim$, with $\xx \sim -\xx$.

	% This is essentially the projection we saw in example~(i), with the point $\xx$ fixed at the north pole. To find the image $\pp\p$ of a point $\pp$, we draw the line

	% \begin{center}
	% 	\begin{tikzpicture} [scale=2]
	% 		\draw (0,0) circle (1);
	% 		\draw [dashed] (-2,0) -- (2,0);

	% 		\draw (0,1) node {$\bullet$};
	% 		\draw (0,1) ++ (-160:1) node {$\bullet$};

	% 		\draw (0,1) ++ (-160:1) node [below right] {$\pp$} -- (0,1) node [above=2pt] {$(0,0,1)$};
	% 	\end{tikzpicture}
	% \end{center}

	% \begin{remark}
	% 	There's \emph{no} subset of $\R^3$ homeomorphic to $\R P^2$. The fundamental domain is $\left\{(x,y,z) \in S^{\,2} : z\geq 0 \right\}$ is a closed hemisphere.

	% 	is $(x,y,z) \to (x,y)$.

	% 	$x^2+y^2+z^2=1$ so $z=\sqrt{1-x^2-y^2}$.

	% 	$(x,y,0) \sim (-x,-y,0)$

	% 	$\R P^2 = \overline{B}^{\,2} / \sim$. where $\xx\sim -\xx$ for $\xx\in S^{\,1}$.
	% 	\missingfigure{Geo 7/7}

	% 	$\R P^2 = [-1,1] \times [-1,1] / \backslash$
	% 	$(-1,y) \sim (1,-y)$
	% 	$(x,1) \sim (-x,-1)$

	% 	Another way to consider this, and to see that $\R P^2$ cannot be embedded in $\R^3$, is to consider that it is the Mobius band with its edge glued to itself.
	% 	\missingfigure{Geo 7/8}
	% 	This is clearly impossible.
	% \end{remark}
\end{enumerate}
\end{examples}

% \subsubsection*{Digression on progressive geometry} % (fold)
% \label{ssub:progression_on_projective_geometry}

% A line in $\R P^2$ is the image of a line in $S^{\,2}$.

% Now $L_1, L_2 \subset S^{\,2}$ intersect in two antipodal points.

% If $\pi:S^{\,2} \to \R P^2$. then $\pi(L_1), \pi(L_2)$ intersect in one point.

% There is a unique line between any two points.

% But: $\R P^2 - \pi(L)$ is $\stackrel{\circ}{B}^{\,2}$ (the open unit ball). For example, consider the $x,y$ plane intersected with $S^{\,2}$. This is connected. (Compare with the Euclidean plane; take away a line and get two different pieces.)

% subsubsection progression_on_projective_geometry (end)

% subsection topology_of_surfaces (end)

\subsection{Building surfaces} % (fold)
\label{sub:building_surfaces}

\begin{definition}
	Let $S_1$ and $S_2$ be surfaces, and $\phi_i:U_i \to B_0(1)$, where $U_i \subset S_i$.

	Then we define the \emph{connected sum of $S_1$ and $S_2$}, denoted $S_1 \# S_2$, to be
	\begin{equation*}
		S_1 \# S_2 = \left[ S_1 - \phi_1^{-1}(B_0(\tf{1}{2})) \right] \cup \left[ S_2 - \phi_2^{-1}(B_0(\tf{1}{2})) \right] / \sim,
	\end{equation*}
	where $\phi_1^{-1}(\tf{1}{2}z) \sim \phi_2^{-1}(\tf{1}{2}\overline{z})$ for $z\in S^{\,1}$.
\end{definition}

For example, we can connect two copies of $T^{\,2}$ in this way to construct a two-holed torus, which is a surface of genus $2$. Indeed, in general, a surface of genus $g$ is the connected sum of $g$ copies of $T^{\,2}$.

% In pictures:
% \missingfigure{Geo 7/9}
% This is a surface of genus 2.

% In general, a surface of genus $g$ is the connected sum of $g$ copies of $T^{\,2}$.
% \missingfigure{Geo 7/10}

\emph{Fact.} Every compact surface is one of
\begin{enumerate}
	\shortskip
	\item $S^{\,2}$;
	\item $\#^g T^{\,2}$ (a genus $g$ surface);
	\item $\#^n \R P^2$.
\end{enumerate}
We will study this further in Part~II \emph{Algebraic Geometry}.

This obviously doesn't work in higher dimensions. For example, there are infinitely many three-manifolds that are not decomposable as connected sums.

Now let's look at another way of building surfaces.

\begin{definition}
	A \emph{triangulated surface} is obtained by starting with a disjoint union of closed triangles and identifying pairs of edges. Each triangle will be embedded with an orientation, and we join them in such a way as to preserve orientation.

	\begin{minipage}{0.24\textwidth}
		\centering
		\begin{tikzpicture} [scale=2, rotate=22]
			\draw [thick] (0,0) -- (1,0) -- ++ (120:1) -- cycle;
			
			\bigarrow [thick] (0,0) -- (0.6,0) node [below=3pt] {$4$};
			\bigarrow [thick] (0,0) ++ (60:1) -- ++ (-60:0.6) node [above right=2pt] {$2$};
			\bigarrow [thick] (0,0) -- ++ (60:0.6) node [left=4pt] {$1$};
		\end{tikzpicture}
	\end{minipage}
	\begin{minipage}{0.24\textwidth}
		\centering
		\begin{tikzpicture} [scale=2, rotate=48]
			\draw [thick] (0,0) -- (1,0) -- ++ (120:1) -- cycle;
			
			\bigarrow [thick] (0,0) -- (0.6,0) node [below right=1pt] {$4$};
			\bigarrow [thick] (1,0) -- ++ (120:0.5) node [above right=3pt] {$5$};
			\bigarrow [thick] (0,0) -- ++ (60:0.6) node [below left=3pt] {$3$};
		\end{tikzpicture}
	\end{minipage}
	\begin{minipage}{0.24\textwidth}
		\centering
		\begin{tikzpicture} [scale=2, rotate=5]
			\draw [thick] (0,0) -- (1,0) -- ++ (120:1) -- cycle;
			
			\bigarrow [thick] (0,0) -- (0.6,0) node [below=3pt] {$5$};
			\bigarrow [thick] (1,0) -- ++ (120:0.5) node [above right=3pt] {$6$};
			\bigarrow [thick] (0,0) -- ++ (60:0.6) node [left=4pt] {$3$};
		\end{tikzpicture}
	\end{minipage}
	\begin{minipage}{0.24\textwidth}
		\centering
		\begin{tikzpicture} [scale=2, rotate=-8]
			\draw [thick] (0,0) -- (1,0) -- ++ (120:1) -- cycle;
			
			\bigarrow [thick] (0,0) -- (0.6,0) node [below left=3pt] {$6$};
			\bigarrow [thick] (1,0) -- ++ (120:0.5) node [right=4pt] {$1$};
			\bigarrow [thick] (0,0) -- ++ (60:0.6) node [left=5pt] {$2$};
		\end{tikzpicture}
	\end{minipage}

	The result is a compact surface. (This means that is is bounded bounded and closed.)

	If $S$ is a triangulated surface with $V$ vertices, $E$ edges and $F$ faces, then
	\begin{equation*}
		\chi(S) = V-E+F
	\end{equation*}
	is the \emph{Euler characteristic} of $S$.
\end{definition}

Let's consider how we might go about computing this Euler characteristic. Counting the number of faces and edges is easily found from the number of triangles which we start with, but counting the number of vertices is more difficult.

Let's consider the Euler characteristic of a connected sum. If $S_1,S_2$ are triangulated surfaces, then we can make a triangulated surface homeomorphic to $S_1 \# S_2$ by removing one face from each of $S_1$ and $S_2$, and identifying edges of those faces. Thus
\begin{equation*}
	F_\# = F_1 + F_2 - 2, \qquad E_\# = E_1 + E_2 - 3, \qquad V_\# = V_1 + V_2 - 3.
\end{equation*}
Substituting into the definition of Euler characteristic, we have
\begin{equation*}
	\chi(S_1 \# S_2) = \chi(S_1) + \chi(S_2) - 2,
\end{equation*}
which gives us a good way to compute the Euler characteristic for complicated surfaces. If we can represent it as the connected sum of simpler surfaces of which we already know the Euler characteristic, then we can compute its Euler characteristic using the formula above.

If $S_1$ and $S_2$ are homeomorphic to one another, and in turn homeomorphic to $S^{\,2}$, then $\chi(S_1) = \chi(S_2)$, and so the Euler characteristic is a topological invariant. The basic idea is to construct triangulated surfaces on $S^{\,2}$ that are homeomorphic to $S_1$ and $S_2$, then apply the formula we already know for convex spherical polyhedra to them.

\begin{example}
	If we take a triangulation of a torus $T^2$, then it has Euler characteristic $0$ (unproved). Thus the connected sum of $g$ tori, a surface of genus $g$, has
	\begin{equation*}
		\chi(\#^g T^2) = 0 - 2\left( g-1 \right) = 2-2g.
	\end{equation*}
\end{example}

Finally, this example should make us wonder whether the Euler characteristic is well-defined for general surfaces, and not just triangulated ones. This turns out to be the case, although we won't prove it here; instead, see \emph{Algebraic Topology}.

% The basic idea of the proof is that if $S_1 \cong S_2 \cong S^{\,2}$, then given a triangulation of $S^{\,2}$, we can show that there's a spherical polyhedron with $F$ faces, $E+n$ edges and $V+n$ vertices. This is homeomorphic to $S_1$ and $S_2$, and then we use $V-E+F=2$ on spherical polyhedron.

% \missingfigure Geo 7/13

% \begin{remark}
% 	We should consider whether this characteristic is well-defined for general surfaces, not just triangulated ones. We won't prove it here; instead see \emph{Algebraic Topology}.
% \end{remark}

% subsection building_surfaces (end)		% Spherical geometry
%!TEX root = geometry.tex
\stepcounter{lecture}
\setcounter{lecture}{3}
\sektion{Möbius transformations}
\label{sec:m_bius_transformations}

\subsection{Stereographic projection} % (fold)
\label{sub:stereographic_projection}

\lecturemarker{8}{13 Feb}
We've encountered Möbius transformations before, in \emph{Groups}. These are transformations of the \emph{extended complex plane}, $\C\cup\{\infty\} = \C_\infty$. We'd like to consider how they apply to the sphere, $S^{\,2}$. To do this, first we need to map the sphere to the complex plane, which we do using \emph{stereographic projections}.

\begin{definition}
	Let $\NN=(0,0,1) \in S^2$ (the ``north pole''). Then we define the \emph{stereographic projection map} $\pi:S^{\,2}\backslash\{\NN\} \to \R^2 = \C$ by
	\begin{equation*}
		\pi(\pp) = \text{intersection of the Euclidean ray $\pp-\NN$ with the $x,y$ plane.}
	\end{equation*}
	Let's consider a slightly flattened picture of this: if $\pp=(x,0,z)$:

	\begin{center}
		\begin{tikzpicture}[scale=2]

			\bigarrow (-1.3,0) -- (2.5,0) node [right=2pt] {$z$};
			\bigarrow (0,-1.3) -- (0,1.6) node [right] {$x$};

			\draw [thick] (0,0) circle (1);

			\draw (0,1) node [above left=2pt] {$\NN$} -- (1.732,0) node {$\bullet$};

			\draw (1.732,0) node [below=2pt] {$\pi(\pp)$};

			\draw (0,1) node {$\bullet$};
			\draw (30:1) node {$\bullet$};
			\draw (30:1) node [above right] {$\pp=(x,0,z)$};

			\draw [dashed] (30:1) -- ++ (0,-0.5);
		\end{tikzpicture}
	\end{center}
\end{definition}

This picture gives us a way to compute the value of $\pi(\pp)$ for a given point $\pp$. By considering the two similar triangles, we see that
\begin{equation*}
	\f{z}{1} = \f{\pi(\pp) - x}{\pi(\pp)} \implies \pi(\pp) = \f{x}{1-z},
\end{equation*}
and so the $x$-coordinate of $\pi(\pp)$ is $x/(1-z)$.

The projection is radially symmetric about the $z$-axis, and so by rotating, we have
\begin{equation*}
	\pi\left( (x,y,z) \right) = \f{x+iy}{1-z}.
\end{equation*}
So now we naturally ask how to invert the projection. In other words, given $w\iC$, can we find $\pp\in S^{\,2}$ with $\pi(\pp) = w$. We consider $w\iC$ with
\begin{equation*}
	w = \f{x+iy}{1-z}, \qquad x^2+y^2+z^2=1.
\end{equation*}
Squaring this equation gives
\begin{equation*}
	\left\vert w \right\vert^2
	= \f{x^2+y^2}{\left( 1-z^2 \right)^2}
	= \f{1-z^2}{\left( 1-z \right)^2}
	= \f{1+z}{1-z}.
\end{equation*}
We can solve this for $z$ in terms of $\left\vert w \right\vert$, which gives
\begin{equation*}
	z = \f{\left\vert w \right\vert^2-1}{\left\vert w \right\vert^2+1}
	\implies 1-z = \f{2}{\left\vert w \right\vert^2+1}.
\end{equation*}
Now we return to our definition of $w$. We have
\begin{equation*}
	x = \left( 1-z \right) \Re(w)
	\qqand
	y = \left( 1-z \right) \Im(w),
\end{equation*}
and thus we can write
\begin{equation*}
	\pi^{-1}(w) = \left( \f{2\,\Re(w)}{1+\left\vert w \right\vert^2}, \f{2\,\Im(w)}{1+\left\vert w \right\vert^2}, \f{\left\vert w \right\vert^2-1}{\left\vert w \right\vert^2+1} \right).
\end{equation*}
This will turn out to be a very useful formula.

This projection map does most of the work of identifying $\C_\infty$ to $S^{\,2}$. There are just two points left unaccounted for: $\infty\in\C_\infty$, and $\NN\in S^{\,2}$. It naturally follows that we can identify $\C_\infty$ and $S^{\,2}$ using the map
\begin{align*}
	w\iC & \longleftrightarrow \pi^{-1}(w) \in S^{\,2}, \\
	\infty & \longleftrightarrow \NN \in S^{\,2}.
\end{align*}
This tells us that $\{w_n\} \to \infty$ in $\C_\infty$ if and only if $\left\vert w_n \right\vert \to \infty$ also.

In this context, we call $\C_\infty$ the \emph{Riemann sphere}, and we will also encounter it in \emph{Complex Analysis} and \emph{Complex Methods}.

% We will encounter this in \emph{Complex Analysis} or \emph{Complex Methods}. A mereomophic fn is analogous to a holoc map $\C\to\C_\infty$.

% subsection stereographic_projection (end)

\subsection{Möbius group} % (fold)
\label{sub:m_bius_group}

Consider an invertible matrix
\begin{equation*}
	A=\mat{a & b \\c & d} \in \GL_2(\C).
\end{equation*}
This induces a Möbius map. We define
\begin{equation*}
	\fullfunction{\phi_A}{\C_\infty}{\C_\infty}{w}{\f{aw+b}{cw+d}},
\end{equation*}
with $\phi_A(-d/c)=\infty$ and $\phi_A(\infty) = a/c$.

% Thinking of Möbius maps as induced by matrices gives us some useful results:

\begin{lemma}
\mbox{}
\begin{enumerate}
	\shortskip
	\item $\phi_{\lambda A}(w)=\phi_A(w)$;
	\item $\phi_A(\phi_B(w)) = \phi_{AB}(w)$.
\end{enumerate}
\end{lemma}

\begin{proof}
	Part (i) is easy and left as an exercise. For (ii), define
	\begin{equation*}
		X = \left\{\ww\iC^2 : \ww\neq\bf{0}\right\}/\sim, \qquad \ww\sim\lambda\ww, \qquad \lambda\iC^*.
	\end{equation*}
	We can define a map $P:X\to\C_\infty$ by $P(w_1,w_2) = w_1/w_2$.

	Then $\GL_2(\C)$ acts on $X$ by $A\cdot\ww = A\ww$ (by matrix multiplication) and
	\begin{equation*}
		P(A\ww) = \phi_A(P(\ww)).
	\end{equation*}
	Then we have
	\begin{equation*}
		\phi_A(\phi_B(P(\ww)))
		= P(A \cdot (B \cdot \ww))
		= P(AB\ww)
		= \phi_{AB}(P(\ww)). \qedhere
	\end{equation*}
\end{proof}

It would have been easy to do this by simply plugging in matrices and turning the handle on some algebra, but this is a cleaner proof. It gives us some understanding of why the result is true.

	\pagebreak

\begin{corollary}
	We define the \emph{projective general linear group} as
	\begin{equation*}
		\PGL_2(\C) := \f{\GL_2(\C)}{\left\{\lambda I: \lambda\iC\right\}}.
	\end{equation*}
	This acts on $\C_\infty$.
\end{corollary}

\vspace{-3pt}

\begin{exercise}
	If $\SL_2(\C)$ is the special linear group, then show that
	\begin{equation*}
		\PGL_2(\C) = \f{\SL_2(\C)}{\left\{\pm I\right\}} =: \PSL_2(\C).
	\end{equation*}
\end{exercise}

\begin{definition}
	The \emph{Mobius group} is given by
	\begin{equation*}
		\Mobgp = \left\{\phi:\C_\infty \to \C_\infty: \phi(w)=\phi_A(w), A\in\GL_2(\C)\right\} \cong \PSL_2(\C).
	\end{equation*}
	Then $\phi\in\Mobgp$ is a \emph{Mobius transformation}. This is the group of all invertible holomorphic maps $\C_\infty \to \C_\infty$.
\end{definition}

\begin{lemma}
\mbox{}
\begin{enumerate}
	\item We can generate $\Mobgp$ with maps of the form
	\begin{itemize}
		\shortskip
		\item $z\mapsto az$, $a\iC^*$ (dilation);
		\item $z\mapsto z+b$, $b\iC$ (translation);
		\item $z\mapsto 1/z$ (inversion).
	\end{itemize}
	\item If $z_1,z_2,z_3$ and $w_1,w_2,w_3$ are two sets of distinct points in $\C_\infty$, then there is a unique $\phi\in\Mobgp$ with $\phi(z_i)=w_i$.
	\item \emph{Cross ratios.} If $z_1,z_2,z_3,z_4\iC_\infty$ are distinct and $\phi\in\Mobgp$ with $\phi(z_i)=w_i$, then cross ratios are preserved. That is,
	\begin{equation*}
		\f{\left( z_2-z_3 \right)\left( z_4-z_1 \right)}{\left( z_2-z_1 \right)\left( z_4-z_3 \right)}
		= \f{\left( w_2-w_3 \right)\left( w_4-w_1 \right)}{\left( w_2-w_1 \right)\left( w_4-w_3 \right)}.
	\end{equation*}
\end{enumerate}
\end{lemma}

\vspace{-6pt}

Proof is left as an exercise.

\begin{definition}
	Let $\cal{C}$ be the set of Euclidean lines and circles in $\C$.
\end{definition}

\begin{lemma}
	Let $S\subset\C$. Then $S\in\cal{C}$ if and only if $S$ satisfies an equation of the form \label{lem:eq-lines-circles}
	\begin{equation*}
		az\zbar + bz+\overline{bz}+c=0,
	\end{equation*}
	for $a,c\iR$, $b\iC$, and not all zero.
\end{lemma}

\begin{proof}
	A line satifies $\alpha x+\beta y = \gamma$, for $\alpha,\beta,\gamma\iR$, so
	\begin{equation*}
		\alpha\left( \f{z+\zbar}{2} \right) + \beta\left( \f{z-\zbar}{2i} \right) = \gamma.
	\end{equation*}
	Rearranging this gives
	\begin{equation*}
		\left( \f{\alpha-i\beta}{2} \right)z + \left( \f{\alpha+i\beta}{2} \right)\zbar = \gamma,
	\end{equation*}
	which is what we want.

	A circle satisfies $\left\vert z-p \right\vert^2=r^2$, so
	\begin{equation*}
		z\zbar-p\zbar-\overline{p}z+\left\vert p \right\vert^2 = r^2 \qquad \text{or} \qquad
		z\zbar-p\zbar-\overline{p}z+\left( \left\vert p \right\vert^2-r^2 \right) = 0,
	\end{equation*}
	which is again the desired form.

	The converse is very similar: if $a\neq 0$, then divide by $a$ and complete the square. If $a=0$, then we have a line. 
\end{proof}

\begin{corollary}
	If $S\in\cal{C}$, $\phi\in\Mobgp$, then $\phi(S)\in\cal{C}$. That is, Möbius maps takes lines and circles to lines and circles.
\end{corollary}

\begin{proof}
	It sufficies to check that this is true for the generators of $\Mobgp$.

	Take $w=\phi(z)=\alpha z$. If the equation of $S$ is $az\zbar + bz+b\zbar + c = 0$, then
	\begin{equation*}
		z = \alpha^{-1} w \implies
		\f{a}{\left\vert \alpha \right\vert^2}\,w\wbar + \f{b}{\alpha}\,w + \f{\overline{b}}{\overline{\alpha}}\,\wbar + c = 0.
	\end{equation*}
	This equation is of the same form, and so elements of $\cal{C}$ map to other elements of $\cal{C}$.

	The cases $z\mapsto z+b$ and $z\mapsto 1/z$ are similar. For the latter, the new equation is
	\begin{equation*}
		\f{a}{w\wbar} + \f{b}{w} + \f{\overline{b}}{\wbar} + c = 0 \implies a + b\wbar + cw\wbar = 0,
	\end{equation*}
	which is again of the same form.
\end{proof}

\begin{corollary}
	There's a unique element of $\cal{C}$ passing through any three distinct points $z_1,z_2,z_3\iC_\infty$.
\end{corollary}

\begin{proof}
	Choose $\phi\in\Mobgp$ with $\phi(z_1)=0$, $\phi(z_2)=1$ and $\phi(z_3)=2$. There's a unique line $S\in\cal{C}$ passing through $0,1,2$ in $\R$. So $C=\phi^{-1}(\R)$ is the set that we want.
\end{proof}

\begin{corollary}
	The group $\Mobgp$ acts transitively on $\cal{C}$.
\end{corollary}

\begin{proof}
	 Given $C_1,C_2$, pick $z_1,z_2,z_3$ on $C_1$, $w_1,w_2,w_3$ on $C_2$, and $\phi$ with $\phi(z_i) = w_i$. Then $\phi(C_1)$ passes though $w_1,w_2,w_3$, and so $\phi(C_1)=C_2$.
\end{proof}

\begin{examples}
\mbox{}
\begin{enumerate}
	\item If $\phi(z) = \df{z-i}{z+i}$, then $\phi(\R)=S^{\,1}\subset \C$.

	Consider: if $z\iR$, then $\left\vert z-i \right\vert = \left\vert z+i \right\vert = \sqrt{z^2+1}$.
	\item The stabiliser of the real line is given by
	\begin{equation*}
		A=\left\{\phi\in\Mobgp: \phi(\R)=\R\right\} = \left\{\phi_A: A\in\GL_2(\R)\right\}.
	\end{equation*}
	Similarly, the stabiliser of the circle has
	\begin{equation*}
		B
		= \left\{\phi\in\Mobgp: \phi(S^{\,1}) = S^{\,1}\right\}
		= \left\{\phi: \phi(z) = \lambda\,\f{z+\alpha}{\overline{\alpha}z+1}, \lambda\in S^{\,1}, \alpha\iC, |\alpha|^2\neq 1\right\}.
	\end{equation*}
	The idea of the proof is that $B=\phi A\phi^{-1}$, where $\phi$ is as in the previous example.
\end{enumerate}
\end{examples}

% subsection m_bius_group (end)		% Möbius transformations
%!TEX root = geometry.tex
\stepcounter{lecture}
\setcounter{lecture}{4}
\sektion{Riemannian geometry}

\lecturemarker{9}{18 Feb}
All the functions we will encounter in this chapter are smooth (that is, infinitely differentiable) unless otherwise stated.

\subsection{Parameterised spaces} % (fold)
\label{sub:parameterised_spaces}

\begin{definition}
	A \emph{parametrised surface} $S\subset\R^3$ is a map $\sigma:U\to\R^3$, where $U$ is an open subset of $\R^2$ such that
	\begin{enumerate}
		\shortskip
		\item $\sigma$ is injective and $\Image(\sigma)=\sigma$;
		\item For each $\pp\in U$, $\ddif{\sigma}{\pp}$ is injective.
	\end{enumerate}
	This condition is actually slightly more restrictive than it needs to be.

	If $S$ satisfies (ii), then we say that it is \emph{smoothly embedded}.
\end{definition}

Recall that if $\sigma=(\sigma_1,\sigma_2,\sigma_3)$, then $\ddif{\sigma}{\pp}:\R^2\to\R^3$ has matrix representation
$$\mat{\ssig{1x} & \ssig{1y} \\
\ssig{2x} & \ssig{2y} \\
\ssig{3x} & \ssig{3y}}, \qquad
\text{where $\sigma_{ix} = \dpd{\sigma_i}{x}$.}$$

\begin{example}
	Consider the following two parametrisations of $S^{\,2}$. First, spherical coordinates:
	\begin{equation*}
		\fullfunction{\sigma}{(0,2\pi) \times (0,\pi)}{\R^3}{(\theta,\phi)}{(\cos\theta \sin\phi, \sin\theta\sin\phi, \cos\phi)},
	\end{equation*}
	Alternatively, consider the inverse of stereographic projection:
	\begin{equation*}
		\fullfunction{\sigma}{\R^2}{\R^3}{(x,y)}{\left( \f{2x}{1+x^2+y^2}, \f{2y}{1+x^2+y^2}, \f{x^2+y^2-1}{1+x^2+y^2} \right)}.
	\end{equation*}
\end{example}

We can construct paths on parametrised surfaces in the obvious way: if $\gamma:[0,1] \to U$ is a path in $U$, then $\Gamma = \sigma \circ \gamma$ is a path in $S$.

The chain rule holds as we would expect:
\begin{equation*}
	\Gamma\p(t) = \ddif{\sigma}{\gamma(t)}[\gamma\p(t)].
\end{equation*}

\begin{definition}
	The \emph{tangent space} to $S$ at $\sigma(\pp)$ is
	\begin{equation*}
		T_{\sigma(\pp)} S := \Image \ddif{\sigma}{\pp},
	\end{equation*}
	which is a linear subspace of $\R^3$.
\end{definition}

As with spherical geometry, the derivative of a path at a point is in the tangent space:
\begin{equation*}
	\Gamma\p(t) = \ddif{\sigma}{\gamma(t)} [\gamma\p(t)] \in T_{\Gamma(t)}.
\end{equation*}
This leads to the following fact, which we shall return to later:
\begin{equation*}
	\left.\vert \Gamma\p(t) \vert\right.^2 = \ddif{\sigma}{\gamma(t)} [\gamma\p(t)] \cdot \ddif{\sigma}{\gamma(t)} [\gamma\p(t)].
\end{equation*}

% subsection parameterised_spaces (end)

\subsection{Riemannian metrics} % (fold)
\label{sub:riemannian_metrics}

\begin{definition}
	If $U\subset\R^2$ is open, then a \emph{Riemannian metric} $g$ on $U$ is a smooth map $g:U\to\Mat_2(\R)$ such that for each $\pp\in U$, $g_{\pp} := g(\pp)$ is symmetric and positive definite. That is,
	\begin{equation*}
		g_\pp = \mat{E(x,y) & F(x,y) \\ F(x,y) & G(x,y)} \text{ with } E(x,y)>0, EG-F^2>0.
	\end{equation*}
\end{definition}

We saw in \emph{Linear Algebra} that a symmetric, positive definite matrix is analogous to an inner product, and that's what we really care about.

For each $\pp\in U$, $g_\pp$ defines an inner product on $\R^2$ 
\begin{equation*}
	\left\langle \va,\vb \right\rangle = \va^\Trans \mat{E & F \\ F & G} \vb =: g_\pp(\va,\vb).
\end{equation*}
If $\sigma:U\to\R^3$ is a parameterised surface, then define $g$ by
\begin{equation*}
	g_\pp(\va,\vb) = \ddif{\sigma}{\pp}(\va) \cdot \ddif{\sigma}{\pp}(\vb),
\end{equation*}
which is the usual inner product on $\R^3$.

\begin{note}
	If $\Gamma=\sigma\circ\gamma$, then
	\begin{equation*}
		\Gamma\p(t) \cdot \Gamma\p(t) = g_{\gamma(t)}(\gamma\p(t),\gamma\p(t)).
	\end{equation*}
\end{note}

Now if we have
\begin{equation*}
	A
	= \ddif{\bsig}{\pp}
	= \mat{\sigma_{1x} & \sigma_{1y} \\ \sigma_{2x} & \sigma_{2y} \\ \sigma_{3x} & \sigma_{3y}}
	= \mat{\bsig_x & \bsig_y},
\end{equation*}
then we can write
\begin{equation*}
	\ddif{\sigma}{\pp}(\va) \cdot \ddif{\sigma}{\pp}(\vb) = A\va \cdot A\vb = \va^\Trans A^\Trans A \vb.
\end{equation*}
This gives us
\begin{equation*}
	g 
	= A^\Trans a
	= \mat{\bsig_x \\ \bsig_y} \mat{\bsig_x \bsig_y}
	= \mat{\bsig_x \cdot \bsig_x & \bsig_x \cdot \bsig_y \\ \bsig_y \cdot \bsig_x & \bsig_y \cdot \bsig_y}.
\end{equation*}
Now we must show that this really is a metric:

\begin{lemma}
	As defined above, $g$ is a Riemannian metric.
\end{lemma}

\begin{proof}
	As $\bsig_x \cdot \bsig_y = \bsig_y \cdot \bsig_x$, the matrix is symmetric.

	To show that it is positive definite, write
	\begin{equation*}
		g_\pp(\va,\va) = \ddif{\sigma}{\pp}(\va) \cdot \ddif{\sigma}{\pp}(\va)\geq 0,
	\end{equation*}
	with equality if and only if $\ddif{\sigma}{\pp}(\va)=0$, which is true if and only if $\va=0$, since $\dif{\sigma}$ is injective. (This is where our embedding hypothesis comes in.)
\end{proof}

\emph{Notation.} We don't usually write $g$ as a $2\times 2$ matrix; instead we write $g=E\dif{x^2} + 2F \dif{x} \dif{y} + G \dif{y^2}$, where
\begin{equation*}
	E = \bsig_x \cdot \bsig_x, \qquad F = \bsig_x \cdot \bsig_y, \qquad G = \bsig_y \cdot \bsig_y.
\end{equation*}
We say that this $g$ is the Riemannian metric on $U$ \emph{induced} by $\sigma$.

Note that not every Riemannian metric arises in this way.

	\pagebreak

\begin{example}
	The Euclidean metric on $\R^2$ is $\dif{x^2}+\dif{y^2}$.

	The Euclidean metric on $\R^3$ is $\dif{u_1^2} + \dif{u_2^2} + \dif{u_3^2}$, for coordinates $(u_1,u_2,u_3)$ on $\R^3$. We write
	\begin{equation*}
		\dif{u_i} = \pd{\sigma_i}{x} \dif{x} + \pd{\sigma_i}{y} \dif{y},
	\end{equation*}
	then the metric induced by $\sigma$ is $\dif{u_1^2} + \dif{u_2^2} + \dif{u_3^2}$.
\end{example}

Now let's look at a more complicated example:

\begin{example}
	Consider
	\begin{equation*}
		\sigma(x,y) = \left( \f{2x}{1+x^2+y^2}, \f{2y}{1+x^2+y^2}, \f{x^2+y^2-1}{1+x^2+y^2} \right), \qquad \alpha=1+x^2+y^2.
	\end{equation*}
	Then we have
	\begin{align*}
		\dif{\sigma_1}
		&= \left( \f{2}{\alpha} - \f{4x^2}{\alpha^2} \right) \dif{x} - \f{4xy}{\alpha^2} \dif{y}
		 = 2 \left( \f{1+y^2-x^2}{\alpha^2} \dif{x} - \f{2xy}{\alpha^2} \dif{y} \right) \\
		\dif{\sigma_2}
		&= 2 \left( \f{1+x^2-y^2}{\alpha^2} \dif{y} - \f{2xy}{\alpha^2} \dif{x} \right) \\
		\dif{\sigma_3}
		&= \f{4x}{\alpha^2} \dif{x} + \f{4y}{\alpha^2} \dif{y}.
	\end{align*}
	So we have
	\begin{align*}
		g
		&= \left( \dif{\sigma_1} \right)^2 + \left( \dif{\sigma_2} \right)^2 + \left( \dif{\sigma_3} \right)^2 \\
		&= \f{4}{\alpha^4} \left[ \left[ \left( 1+y^2-x^2 \right)^2 + 4x^2 y^2 + 4x^2 \right] \dif{x^2} \right. \\
		&  \qquad + \left[ -2xy-2xy+4xy \right] \dif{x} \dif{y} \\
		&  \qquad\qquad \left.+ \left[ \left( 1+x^2 - y^2 \right)^2 + 4x^2 y^2 + 4y^2 \right] \dif{y}^2 \right] \\
		&= \f{4}{\alpha^4} \left( \alpha^2 \dif{x^2} + \alpha^2 \dif{y^2} \right) \\
		&= \f{4\left( \dif{x^2} + \dif{y^2} \right)}{\left( 1+x^2+y^2 \right)^2}.
	\end{align*}
	Notice in particular that this is a function of the standard Euclidean metric.
\end{example}

% subsection riemannian_metrics (end)

	\pagebreak

\subsection{Geometry with the Riemannian metric} % (fold) % formerly Riemannian geometry
\label{sub:geometry_with_the_riemannian_metric}

Let $g$ be a Riemannian metric on $U\subset\R^2$.

\begin{definition}
	A path $\gamma:[0,1]\to\R^2$ is \emph{piecewise smooth} if it is continuous on $[0,1]$ and smooth except at finitely many points $0=t_0<t_1<t_2<\cdots<t_n=1$.

	This gives us a way to define \emph{length}. If $\gamma:[0,1]\to U$ is a piecewise smooth curve, then we define
	\begin{equation*}
		L_g(\gamma) = \sum_{i=0}^{n-1} \int_{t_i}^{t_i+1} \sqrt{g_{\gamma(t)} (\gamma\p(t), \gamma\p(t))} \dif{t}.
	\end{equation*}
\end{definition}

Now, if $g$ is induced by $\sigma$, and $\Gamma=\sigma\circ\gamma$, then
\begin{equation*}
	|\Gamma\p(t)| = \sqrt{g_{\gamma(t)} (\gamma\p(t),\gamma\p(t))},
\end{equation*}
and so $L_g(\gamma)=L(\Gamma)$, the Euclidean length. This feels intuitively correct.

Now we have a notion of length, we can define distance: if $\pp,\qq\in U$, then define
\begin{equation*}
	d(\pp,\qq) = \inf\left\{L_g(\gamma) \mid \gamma:[0,1] \to U \text{ piecewise smooth}, \gamma(0) = \pp, \gamma(1) = \qq\right\}.
\end{equation*}
Note, however, that the infinum need not be obtained by any $\gamma$. Consider:

\begin{example}
	Let $U=\R^2\backslash\{\vec{0}\}$. Take $g=\dif{x^2} + \dif{y^2}$, and $\pp=(-1,0)$, $\qq=(1,0)$.

	Then the infinum is the straight line between them, but this is disallowed since $\vec{0}$ has been excluded. Thus the infinum is never attained.
\end{example}

Once we have distance, then we can define surface area. If $A\subset U$, then define
\begin{equation*}
	\Area(A) = \iint_A \sqrt{EG-F^2} \dif{x} \dif{y} = \iint_A \sqrt{\det g} \dif{x} \dif{y},
\end{equation*}
if this integral is defined, and otherwise we say that the area of $A$ is undefined.

\lecturemarker{10}{20 Feb}
Now we want to show that our notion of distance really is a metric in Riemannian space.

\begin{proposition}
	As defined above, $d$ is a metric.
\end{proposition}

\begin{proof*}
	We must check that:
	\begin{enumerate}
		\shortskip
		\item $d(\pp,\qq) \geq 0$, with equality if and only if $\pp=\qq$;
		\item $d(\pp,\qq) = d(\qq,\pp)$;
		\item $d(\pp,\qq) + d(\qq,\rr) \geq d(\pp,\rr)$.
	\end{enumerate}
	Unlike previous metrics, it turns out that (i) will be the hardest condition to prove. We need a lemma:
\end{proof*}

\begin{lemma}
	Given $\pp\p\in U$ and $r>0$ so that $B_r(\pp\p)\in U$, there is $c>0$ such that if $\gamma:[0,1] \to B_r(\pp\p)$ is a path, then $L_g(\gamma) \geq c\left\vert \gamma(0)-\gamma(1) \right\vert$ (Euclidean distance).
\end{lemma}

\begin{proof}
	[Proof of lemma] Consider the general form of $g_\pp$:
	\begin{equation*}
		g_\pp = \mat{E & F \\ F & G}.
	\end{equation*}
	This is symmetric and positive definite, so it has strictly positive eigenvalues $\lambda_1(\pp), \lambda_2(\pp)$ and eigenvectors $\vv_1(\pp), \vv_2(\pp)$ which form an orthonormal basis of $\R^2$.

	If $\vv=a\vv_1(\pp) + b\vv_2(\pp)$, then
	\begin{align*}
		g_\pp(\vv,\vv)
		&= a^2 \lambda_1(p) + b^2 \lambda_2(\pp) \\
		&\geq  \min(\lambda_1,\lambda_2) \left.(a^2+b^2)\right. \\
		&= \min(\lambda_1,\lambda_2) \, \vv\cdot\vv.
		\tag{$*$}
	\end{align*}
	Now $\lambda_1,\lambda_2$ are continuous functions of $\pp$ and $\closure{B_r(\pp\p)}$ is compact, so there exists some $\qq_1 \in \closure{B_r(\pp\p)}$ with $\lambda_1(\qq_1) \leq \lambda_1(\rr)$ for all $\rr\in B_r(\pp)$. Similarly, there is some $\qq_2$ with $\lambda_2(\qq_2) \leq \lambda_2(\rr)$ for all $\rr\in B_r(\pp)$.

	So take $\lambda=\min(\lambda_1(\qq_1),\lambda_2(\qq_2))$, then $g_\pp(\vv,\vv)≥\lambda \vv\cdot\vv$ for all $\pp\in B_r(\pp)$ (from $(*)$). Then
	\begin{align*}
		L_g(\gamma)
		&= \int_0^1 \sqrt{g_{\gamma(t)}(\gamma\p(t),\gamma\p(t))} \dif{t} \\
		&\geq \int_0^1 \sqrt{\lambda \gamma\p(t) \cdot \gamma\p(t)} \dif{t} \\
		&= \sqrt{\lambda}\, L_\text{Euclidean}(\gamma) \\
		&\geq \sqrt{\lambda}\,\left\vert \gamma(0)-\gamma(1) \right\vert.
	\end{align*}
	So we take $c=\sqrt{\lambda}$.
\end{proof}

\begin{proof} [Proof of proposition]
\mbox{}
\begin{enumerate}
	\item For any $\gamma$, we have $L_\gamma(\gamma) \geq 0$, so
	\begin{equation*}
		d(\pp,\qq) = \inf_\gamma {L_g(\gamma)} \geq 0.
	\end{equation*}
	Pick $r>0$ with $\closure{B_r(\pp)} \subset U$, and choose $c>0$ as in the lemma.

	Spose $\qq \neq \pp$. If $\qq\in\closure{B_r(\pp)}$, $\gamma(0)=\pp$, $\gamma(1)=\qq$, then the lemma tells us that
	\begin{equation*}
		L_g(\gamma)
		\geq c \left\vert \gamma(0)-\gamma(1) \right\vert
		= c\,d_\text{Euclidean}(\pp,\qq)
	\end{equation*}
	and so we have
	\begin{equation*}
		d(\pp,\qq) = \int_\gamma {L_g(\gamma)} \geq c\,d_\text{Euclidean}(\pp,\qq) > 0,
	\end{equation*}
	since $\pp \neq \qq$.

	If $\qq\not\in \closure{B_r(\pp)}$, then by the intermediate value theorem, if $\gamma(0)=\pp$, $\gamma(1)=\qq$, then there exists some $t\in(0,1)$ with $\left\vert \pp-\gamma(t) \right\vert=\rr$. Then
	\begin{equation*}
		L_g(\gamma) \geq L_g(\gamma|_{[0,t]}) \geq c\left\vert \pp-\gamma(t) \right\vert = cr > 0,
	\end{equation*}
	so again $\inf L_g(\gamma) \geq cr > 0$.

	\item There's a bijection between
	\begin{equation*}
		\left\{ \text{paths from $\pp$ to $\qq$} \right\} \longleftrightarrow \left\{ \text{paths from $\qq$ to $\pp$} \right\}
	\end{equation*}
	taking $\gamma(t) \mapsto \gamma(1-t) = \gamma^{-1}(t)$; that is, just traversing the paths in the opposite directions. So we have
	\begin{equation*}
		L_g(\gamma) = L_g(\gamma^{-1}) \implies d(\pp,\qq) = d(\qq,\pp).
	\end{equation*}

		\pagebreak

	\item We want to show that $d(\pp,\qq)+d(\qq,\rr) \geq d(\pp,\rr)$. Pick:
	\begin{itemize}
		\shortskip
		\item $\gamma_1$ with $\gamma_1(0)=\pp$, $\gamma_1(1)=\qq$, and $L_g(\gamma_1) \leq d(\pp,\qq)+\epsilon$ ($\epsilon>0$), and;
		\item $\gamma_2$ with $\gamma_2(0)=\qq$, $\gamma_2(1)=\rr$, and $L_g(\gamma_2) \leq d(\qq,\rr)+\epsilon$.
	\end{itemize}
	% \missingfigure{Geo 10/1}
	Define $\gamma$ by
	\begin{equation*}
		\gamma(t) =
		\begin{cases}
			\gamma_1(2t) & \text{if } t \leq 1/2, \\
			\gamma_2(2t-1) & \text{if } t>1/2.
		\end{cases}
	\end{equation*}
	Then $\gamma$ is piecewise smooth and
	\begin{equation*}
		L_g(\gamma)
		= L_g(\gamma_1) + L_g(\gamma_2)
		= d(\pp,\qq) + d(\qq,\rr) + 2\epsilon
		\geq d(\pp,\rr),
	\end{equation*}
	and letting $\epsilon\to 0$ gives
	\begin{equation*}
		d(\pp,\qq) + d(\qq,\rr) \geq d(\pp,\rr). \qedhere
	\end{equation*}
\end{enumerate}
\end{proof}

So now this definitely defines a metric. Now we move to consider angles. If $\gamma_1,\gamma_2:[0,1]\to U$, with $\gamma_i(t_i) = \pp$, then let $\vv_i = \gamma\p_i(t_i)$.

\begin{center}
\begin{tikzpicture}[scale=2.5, rotate=22]
	\draw plot [smooth] coordinates {(-0.8, -0.512) (-0.7, -0.343) (-0.6, -0.216) (-0.5, -0.125)
		(-0.4, -0.064) (-0.3, -0.027) (-0.2, -0.008) (-0.1, -0.001) (0,0)
		(0.1, 0.001) (0.2, 0.008) (0.3, 0.027) (0.4, 0.064) (0.5, 0.125)
		(0.6, 0.216) (0.7, 0.343) (0.8, 0.512)}; %
	
	\draw (0.8,0.512) node [above right] {$\gamma_1$};

	\begin{scope} [rotate=-72]
		\draw plot [smooth] coordinates {(-0.8, -0.512) (-0.7, -0.343) (-0.6, -0.216) (-0.5, -0.125)
			(-0.4, -0.064) (-0.3, -0.027) (-0.2, -0.008) (-0.1, -0.001) (0,0)
			(0.1, 0.001) (0.2, 0.008) (0.3, 0.027) (0.4, 0.064) (0.5, 0.125)
			(0.6, 0.216) (0.7, 0.343) (0.8, 0.512)}; %

		\draw (0.8,0.512) node [right] {$\gamma_0$};
	\end{scope}

	\draw (0,0) node {$\bullet$};
	\draw (0.05,0) node [above=3pt] {$\pp$};

	\bigarrow [thick] (0,0) -- (0.5,0);
	\bigarrow [thick] (0,0) -- ++ (-72:0.5);

	\draw (0.2,0) arc (0:-72:0.2);
	\draw (0.16,-0.14) node [right] {$\theta$};
\end{tikzpicture}
\end{center}

% \missingfigure{Geo 10/2}

The angle $\theta$ between $\gamma_1$ and $\gamma_2$ at $\pp$ is defined to be
\begin{equation*}
	\cos\theta = \f{g_\pp(\vv_1,\vv_2)}{\left\vert \vv_1 \right\vert_g \left\vert \vv_2 \right\vert_g}, \text{ where } \left\vert \vv_i \right\vert_g = \sqrt{g_\pp(\vv_i,\vv_i)}.
\end{equation*}
If $g$ is induced by $\sigma:U\to\R^3$, then $\theta$ is the Euclidean angle between $\Gamma_1=\sigma\circ\gamma_1$ and $\Gamma_2=\sigma\circ\gamma_2$ at $\sigma(\pp)$.

% subsection geometry_with_the_riemannian_metric (end)

	\pagebreak

\subsection{Isometries} % (fold)
\label{sub:riemannian_isometries}

Let $U_i\subset\R^2$. Suppose $\varphi:U_1\to U_2$ is bijective, and that $\varphi,\varphi^{-1}$ are smooth.

If $g_2$ is an Riemannian metric on $U_2$, then there is an induced metric $g_2\p$ on $U_1$, given by
\begin{equation*}
	g\p_{2\pp}(\va,\vb) = g_{2\,\varphi(\pp)} (\ddif{\varphi}{\pp}(\va), \ddif{\varphi}{\pp}(\vb)).
\end{equation*}
In terms of matrices, we have
\begin{equation*}
	g_2\p = \dif{\varphi^\Trans} \mat{E_2 & F_2 \\ F_2 & G_2} \dif{\varphi},
\end{equation*}
where
\begin{equation*}
	g_2 = \mat{E_2 & F_2 \\ F_2 & G_2}.
\end{equation*}

\begin{definition}
	If $\varphi$ is as above and $g_i$ is an Riemannian metric on $U_i$, we say that $\varphi$ is a \emph{Riemannian isometry} if $g_1 = g_2\p$; that is,
	\begin{equation*}
		g_1(\va,\vb) = g_2(\dif{\varphi(\va)}, \dif{\varphi(\vb)}).
	\end{equation*}
\end{definition}

\begin{proposition}
	If  $\varphi:U_1 \to U_2$ is a Riemannian isometry, then
	\begin{enumerate}
		\shortskip
		\item $L_{g_1}(\gamma) = L_{g_2}(\varphi \circ \gamma)$.
		\item $d_1(\pp,\qq) = d_2(\varphi(\pp),\varphi(\qq))$, where $d_i$ is the metric induced by $g_i$.
		\item The angle between $\gamma_1$ and $\gamma_2$ at $\pp$ is the angle between $\varphi\circ \gamma_1$ and $\varphi\circ\gamma_2$ at $\varphi(\pp)$.
		\item If $A\subset U_1$, then $\Area_1(A) = \Area_2(\varphi(A))$.
	\end{enumerate}
\end{proposition}

\begin{proof}
\mbox{}
\begin{enumerate}
	\item First we have
	\begin{equation*}
		g_2((\varphi \circ \gamma)', (\varphi \circ \gamma)')
		= g_2(\dif{\varphi(\gamma\p)}, \dif{\varphi(\gamma\p)})
		= g_1(\gamma\p,\gamma\p).
	\end{equation*}
	Thus we have
	\begin{equation*}
		L_{g_2}(\varphi \circ \gamma)
		= \int_0^1 \sqrt{g_2((\varphi \circ \gamma)', (\varphi\circ\gamma)')} \dif{t}
		= \int_0^1 \sqrt{g_1(\gamma\p, \gamma\p)} \dif{t}
		= L_{g_1}(\gamma).
	\end{equation*}
	\item There's a bijection
	\begin{equation*}
		\left\{\text{paths from $p$ to $q$ in $U_1$}\right\} \longleftrightarrow \left\{\text{paths from $\varphi(\pp)$ to $\varphi(\qq)$ in $U_2$}\right\}
	\end{equation*}
	taking $\gamma \mapsto \varphi\circ\gamma$. Since $L_{g_1}(\gamma) = L_{g_2}(\varphi \circ \gamma)$, the infinima are the same.
	\item Similar to (i).
	\item In matrix form,
	\begin{equation*}
		g_1 = \dif{\varphi^\Trans} g_2 \dif{\varphi} \implies \det g_1 = (\det \dif{\varphi})^2 \det g_2.
	\end{equation*}
	With this in hand, we have
	\begin{align*}
		\Area_1(A)
		&= \iint_A \sqrt{\det g_1} \dif{A} \\
		&= \iint_A \left\vert \det \dif{\varphi} \right\vert \sqrt{\det g_2} \dif{A} \\
		&= \iint_{\varphi(A)} \sqrt{\det g_2} \dif{A} \\
		&= \Area_2(\varphi(A)). \qedhere
	\end{align*}
\end{enumerate}
\end{proof}

This naturally leads us to consider conformal maps. \lecturemarker{11}{25 Feb} % Hopefully these are familiar from \emph{Complex Analysis}, although we'll look at them slightly differently. 

\begin{definition}
	If $g_1,g_2$ are Riemannian metrics on $U\subset \R^2$, then we say that $g_1$ and $g_2$ are \emph{conformal} if
	\begin{equation*}
		g_{1\pp} = \lambda(\pp)\,g_{2\pp},
	\end{equation*}
	where $\lambda:U\to\R^+$.
\end{definition}

Notice that if $g_1,g_2$ are conformal, then
\begin{equation*}
	\f{g_{1\pp}(\va,\vb)}{\left\vert \va \right\vert_{g_{1\pp}} \left\vert \vb \right\vert_{g_{1\pp}}}
	= \f{\lambda(\pp)\,g_{2\pp}(\va,\vb)}{ \sqrt{\lambda(\pp)}\, \left\vert \va \right\vert_{g_{2\pp}} \sqrt{\lambda(\pp)}\,\left\vert \vb \right\vert_{g_{2\pp}}}
	= \f{g_{2\pp}(\va,\vb)}{\left\vert \va \right\vert_{g_{2\pp}} \left\vert \vb \right\vert_{g_{2\pp}}},
\end{equation*}
so the angle between $\va, \vb$ is the same under $g_1$ and $g_2$. Conformal maps preserve angles.

\begin{example}
	Consider the Euclidean metric $g^E$, and the spherical metric $g^{S^2}$. These are conformal:
	\begin{equation*}
		g^E = \dif{x^2} + \dif{y^2} \qqand
		g^{S^2} = \f{4\left( \dif{x^2} + \dif{y^2} \right)}{\left( 1+x^2+y^2 \right)^2}.
	\end{equation*}
\end{example}

There's more than one definition. If $g_i$ is a metric on $U_i$, and $\phi:U_1\to U_2$, then we say that $\phi$ is conformal if $g_2\p$ defined by
\begin{equation*}
	g_2\p(\va,\vb) = g_2(d\phi(\va), d\phi(\vb))
\end{equation*}
is conformal to $g_1$.

% \begin{example}
% 	Consider the stereographic projection $\pi:S^{\,2}\backslash\{\NN\} \to \C$. This defines a conformal map.
% \end{example}

\begin{proposition}
	If $f:U_1\to U_2$ is holomorphic with $f\p(w)\neq 0$ for all $w\in U_1$, then it is conformal to the Euclidean metric $g^E$ or the spherical metric $g^{S^2}$.
\end{proposition}

\begin{proof}
	Let $z=x+iy$. Then $\zbar=x-iy$, and we have
	\begin{equation*}
		\dif{x^2} + \dif{y^2} = \dif{z} \dif{\zbar}.
	\end{equation*}
	If $z=f(w)$, then
	\begin{equation*}
		dz = \pd{f}{w} \dif{w} = f\p(w) \dif{w},
	\end{equation*}
	% since the second term is zero by the Cauchy-Riemann equations.
	Similarly, we have
	\begin{equation*}
		\dif{\zbar} = f\p(\wbar) \dif{\wbar} = \overline{f\p(w)} \dif{\wbar}.
	\end{equation*}
	Combining these two results, we have
	\begin{equation*}
		\dif{z} \dif{\zbar} = \left.|f\p(w)|\right.^2 \dif{w} \dif{\wbar},
	\end{equation*}
	and so $\dif{z} \dif{\zbar}$ is conformal to $\dif w \dif{\wbar} = g^E$.
\end{proof}

\begin{corollary}
	Mobius transformations are conformal with respect to $g^E$.
\end{corollary}

\vspace{-6pt}

\emph{Converse.} If $f:U_1\to U_2$ is conformal, then either
\begin{enumerate}
	\shortskip
	\item $f$ is orientation preserving, and hence holomorphic, or;
	\item $f$ is orientiation reversing, and $f(z)=g(\zbar)$, where $g$ is holomorphic.
\end{enumerate}

\emph{Idea of proof.} Since $\pi$ is conformal, and isometries are conformal, we see that $\pi\circ A \circ\pi^{-1}$ is a conformal map $\C_\infty \to \C_\infty$. Every orientation preserving conformal map of $\C_\infty$ is a Mobius map, which is proved properly in \emph{Complex Analysis} or \emph{Complex Methods}.

% subsection riemannian_isometries (end)   % Riemannian geometry
%!TEX root = geometry.tex
\stepcounter{lecture}
\setcounter{lecture}{5}
\sektion{Hyperbolic geometry}

\subsection{Hyperboloid model} % (fold)
\label{sub:hyperboloid_model}

\begin{definition}
	We consider the surface $S \subset \R^3$ given by $x^2+y^2-z^2=-1$, $z<0$. This \emph{hyperboloid sheet} gives us another way to think of points. The sketch below illustrates the sheet: it asymptotically approaches the planes $x=y$, $y=z$ and $z=x$; we take a cross section view.

	\begin{center} % Geo 11/1 
		\begin{tikzpicture}[yscale=1.4, rotate=180]
			\bigarrow (2,0) -- (-2,0);
			\bigarrow (0,2.5) -- (0,-1);

			\begin{scope} [yscale=3, xscale=1.4]
				\draw [thick] (0,0.3) arc (-90:-55:1.8);
			\end{scope}

			\begin{scope} [yscale=3, rotate=180, xscale=1.4]
				\draw [thick] (0,-0.3) arc (90:55:1.8);
			\end{scope}

			\draw [thick] (-0.05,0.9) -- (0.05,0.9);

			\draw [dashed] (2,2) -- (0,0) -- (-2,2);
		\end{tikzpicture}
	\end{center}

	With this in mind, we give $\R^3$ the \emph{Minkowski metric}
	\begin{equation*}
		g^M = \dif{x^2} + \dif{y^2} - \dif{z^2}.
	\end{equation*}
	Formally, we've taken the surface $x^2+y^2+z^2=-1$, and replaced $z$ by $iz$.
\end{definition}

At first, these two definitions might seem unnatural, but in some sense, it's the most natural thing in the world. Note, however, that the Minkowski metric is \emph{not} a Riemannian metric. It is sometimes called the \emph{pseudo-Riemannian metric}.

As before, we consider the stereographic projection $\pi:S\to\C$. And as before, we have a ``north pole'' $\NN=(0,0,1)$, and for any $\pi\in S$ that is not $\NN$, we define $\pi(\pp)$ to be the intersection of $NP$ with the $xy$ plane. Thus
\begin{equation*}
	\pi\left( (x,y,z) \right) = \f{x+iy}{1-z} = w,
\end{equation*}
the same as the sphere. Inversion is similar; we first consider
\begin{equation*}
	\left\vert w \right\vert^2
	= \f{x^2+y^2}{\left( 1-z \right)^2}
	= \f{z^2-1}{\left( 1-z \right)^2}
	= \f{z+1}{z-1}
	\implies
	z = \f{\left\vert w \right\vert^2+1}{\left\vert w \right\vert^2-1}.
\end{equation*}
Now, if $z<0$, then $\left\vert w \right\vert^2<1$, and so
\begin{equation*}
	\Image(\pi) = D = \left\{w\iC: \left\vert w \right\vert<1\right\}.
\end{equation*}
So using the same process as the sphere, we have
\begin{equation*}
	1-z = \f{2}{1-\left\vert w \right\vert^2},
\end{equation*}
and so the inverse is given by
\begin{equation*}
	\pi^{-1}(w) = \left( \f{2\,\Re(w)}{1-\left\vert w \right\vert^2}, \f{2\,\Im(w)}{1-\left\vert w \right\vert^2}, \f{\left\vert w \right\vert^2+1}{\left\vert w \right\vert^2-1} \right).
\end{equation*}

% subsection hyperboloid_model (end)

\subsection{Unit disc model} % (fold)
\label{sub:unit_disc_model}

Let $g^D$ be the metric induced on $g$ using $g^M$, that is,
\begin{equation*}
	g^D = \dif{\sigma_1^2} + \dif{\sigma_2^2} - \dif{\sigma_3^2},
\end{equation*}
where
\begin{equation*}
	\sigma(x,y) = \left( \f{2x}{1-x^2-y^2}, \f{2y}{1-x^2-y^2}, \f{1+x^2+y^2}{1-x^2-y^2} \right).
\end{equation*}
Note that we've now switched to using $x$ and $y$ as coordinates on $D$, not on $\R^3$. This is essentially the same calculation as for $g^{S^2}$, and we obtain
\begin{equation*}
	g^D = \f{4\left( \dif{x^2} + \dif{y^2} \right)}{(1-x^2-y^2)^2}.
\end{equation*}
Again, this is conformal to $g^E$.

% subsection unit_disc_model (end)

\subsection{Upper half-plane model} % (fold)
\label{sub:upper_half_plane_model}

This gives us another way to do hyperbolic geometry. Let $H=\{z\iC: \Im(z)>0\}$, the upper half-plane. Define
\begin{equation*}
	\fullfunction{\varphi}{\C_\infty}{\C_\infty}{z}{(z-i)/(z+i)}
\end{equation*}
We saw in section~\ref{sec:m_bius_transformations} that $\phi(\R\cup\{\infty\}) = S^1$, and since $\phi(i)=0$, we have $\phi(H)=D$.

\begin{definition}
	Let $g^H$ be the Riemannian metric on $H$ induced from $g^D$ using $\phi$:
	\begin{equation*}
		g_\pp^H(\va,\vb) = g^D_{\phi(\pp)}(\ddif{\phi}{\pp}(\va), \ddif{\phi}{\pp}(\vb))
	\end{equation*}
	By definition, $\phi$ is a Riemann isometry from $g^H$ to $g^D$.
\end{definition}

To compute $g^H$, write $w=x+iy$. Then $\dif{x^2} \dif{y^2} = \dif{w} \dif{\wbar}$. For $w\in D$, we have
\begin{equation*}
	w=\phi(z) = \f{z-i}{z+i} = 1-\f{zi}{z+i}.
\end{equation*}
By careful consideration, this gives us
\begin{equation*}
	\dif{w} = \f{zi}{\left( z+i \right)^2} \qqand
	\dif{\wbar} = \f{-zi}{\left( \zbar-i \right)^2} \dif{\zbar}.
\end{equation*}
By substituting appropriately, and writing $z=u+iv$, we have
\begin{align*}
	g^H
	= \f{4\left( \df{zi \dif{z}}{\left( z+i \right)^2} \right) \left( \df{-zi \dif{\zbar}}{\left( \zbar-i \right)^2} \right)}{\left( 1-\df{\left( z-i \right)\left( \zbar+i \right)}{\left( z+i \right)\left( \zbar-i \right)} \right)^2}
	&= \f{16 \dif{z} \dif{\zbar}}{\left[ \left( z+i \right)\left( \zbar-i \right) - \left( z-i \right)\left( \zbar+i \right) \right]^2} \\
	&= \f{16 \dif{z} \dif{\zbar}}{\left[ zi\left( \zbar-z \right) \right]^2} \\
	&= \f{16 \dif{z} \dif{\zbar}}{\left( 4v \right)^2} \\
	&= \f{\dif{u^2} + \dif{v^2}}{v^2}.
\end{align*}

% $G^H = \PSL_2(\R)$ acts on $H$ $\leftrightarrow$ $G^D = {\phi: \phi(z) = e^{i\theta} \f{z-a}{az-1}, \theta\iR, a\in D}$
% $G^D = \phi G^H \phi^{-1}$.

% subsection upper_half_plane_model (end)

\subsection{Geometry of the hyperbolic plane} % (fold)
\label{sub:geometry_of_the_hyperbolic_plane}

\lecturemarker{12}{27 Feb}
We've now seen two models of the hyperbolic plane: the upper half-plane model $H$, and the unit disc model $D$.

In particular, recall that $G^H = \PSL_2(\R)$ acts on $H$, with
\begin{equation*}
	G^D = \left\{\phi: \phi(z) = e^{i\theta} \f{z-a}{az-a}, \theta\iR, a\in D\right\} = \phi G^H \phi^{-1}.
\end{equation*}
This $\phi$ gives us a way to compare boundaries: $\boundary{H} = \R\cup\infty=\R_\infty$, and $\boundary{D} =S^{\,1}$. Then $\phi(\R_\infty) = S^{\,1}$.

We may use both of these models, depending on which is more convenient at the time. We use $\HH$ to denote either one, without specifying which. With these two models in hand, we can discuss all the features of geometry that we've talked about before.

Angles are simple. Both $g^D$ and $g^H$ are conformal, so angles between curves in $g^D$ or $g^H$ is the same as the Euclidean angle.

Now let's consider isometries. Let $\Isom(H)$ be the group of Riemannian isometries of $(H,g^H)$, and similarly, $\Isom(D)$ be the group of Riemannian isometries of $(D,g^D)$. We come to our first result:

\begin{proposition}
	$G^H \subset \Isom(H)$.
\end{proposition}

\vspace{-8pt}

Note that $G^H$ isn't all isometries, as not all isometries are orientiation preserving.

\begin{proof}
	Recall that $G^H = \PSL_2(\R)$ is generated by three kinds of maps:
	\begin{enumerate}
		\shortskip
		\item $z\mapsto z+b$, $b\iR$;
		\item $z \mapsto az$, $a\iR$;
		\item $z\mapsto -1/z$.
	\end{enumerate}
	It suffices to check that (i), (ii) and (iii) are in $\Isom(H)$.
	\begin{enumerate}
		\item If $z=\phi(z\p) = z\p+b$, then we have
		\begin{equation*}
			\begin{array}{lcl}
				x=x\p + b & & \dif{x} = \dif{x\p} \\
				y=y\p & & \dif{y} = \dif{y\p}.
			\end{array}
		\end{equation*}
		Thus we have
		\begin{equation*}
			g^H
			= \f{\dif{x^2} + \dif{y^2}}{y^2}
			\xrightarrow[\text{induced by $\phi$}]{\text{metric}}
			\f{\left( \dif{x\p} \right)^2 + \left( \dif{y\p} \right)^2}{\left( y\p \right)^2}
			= g^H.
		\end{equation*}

		\item If $z=\phi(z\p)=az\p$, then $x=ax\p$, $y=ay\p$ and
		\begin{equation*}
			g^H
			= \f{\dif{x^2} + \dif{y^2}}{y^2}
			\longrightarrow
			\f{a^2\left( \dif{x\p} \right)^2 + a^2\left( \dif{y\p} \right)^2}{a^2\left( y\p \right)^2} = g^H.
		\end{equation*}

		\item If $z=\phi(w)=-1/w$, then
		\begin{equation*}
			\dif{z} = \f{\dif{w}}{w^2} \qqand
			\dif{\zbar} = \f{\dif{\wbar}}{\wbar^2}.
		\end{equation*}
		Then we have
		\begin{equation*}
			g^H
			= \f{\dif{x^2} + \dif{y^2}}{y^2}
			= \f{\dif z \dif{\zbar}}{\left( \f{1}{2}\left( \zbar-z \right) \right)^2}
			\longrightarrow
			\f{\left( -\f{\dif{w}}{w^2} \right)\left( -\f{\dif{\wbar}}{\wbar^2} \right)}{\left( \f{1}{2}\left( \f{1}{\wbar}-\f{1}{w} \right) \right)^2}
			= \f{\dif{w} \dif{\wbar}}{\left( \f{1}{2}\left( w-\wbar \right) \right)^2} = g^H. \qedhere
		\end{equation*}
	\end{enumerate}
\end{proof}

\begin{corollary}
	$G^D \subset \Isom(D)$
\end{corollary}

\begin{proof}
	If $\psi \in G^D$, then $\psi=\phi_0 \chi \phi_0^{-1}$, where $\chi\in G^H$. Thus $\phi_0, \chi, \phi_0^{-1}$ are all isometries, and the composition of isometries is an isometry. Thus $\psi \in \Isom(D)$.
\end{proof}

Now we consider hyperbolic lines. These are defined in a very similar way to spherical lines.

\begin{definition}
	A \emph{hyperbolic line} in $H$ is $L = H \cap C$, where $C$ is a Euclidean line or circle which is perpendicular to $\partial{H}$. A similar definition under the disc model comes by replacing $H$ by $D$.

	\begin{center}
		\begin{minipage}{0.4\textwidth} % Geo 12/1, 12/1b
		\centering
		\begin{tikzpicture}
			\draw (0,0) -- (5,0);
			\draw [very thick] (1,0) -- (1,2);
			\draw [very thick] (4,0) arc (0:180:1);
		\end{tikzpicture}
	\end{minipage}
	\hspace{0.2cm}
	\begin{minipage}{0.4\textwidth}
		\centering
		\begin{tikzpicture}[scale=0.6]
			\draw (2,2) circle (2);
			\clip (2,2) circle (2);
			\draw [very thick] (2,2) -- ++ (45:2) -- ++ (-135:4);
			\draw [very thick] (2,2) ++ (-45:2) circle (1);
		\end{tikzpicture}
	\end{minipage}

	\begin{minipage}{0.4\textwidth}
		\centering
		half-plane model
	\end{minipage}
	\hspace{0.2cm}
	\begin{minipage}{0.4\textwidth}
		\centering
		disc model
	\end{minipage}
	\end{center}
\end{definition}

For the rest of the chapter, when we say ``line'', we mean ``hyperbolic line'' unless otherwise specified.

Once we have lines, then it's natural to define rays:

\begin{definition}
	If $\gamma:\R\to H$ is a parameterisation of a line, then $R=\gamma([c,\infty))$ is a \emph{hyperbolic ray} starting at $\gamma(c)$ and with direction $\gamma\p(c)$.
\end{definition}

Now let's consider a few basic results involving lines:

\begin{lemma}
	$L$ is a line in $H$ if and only if $\phi_0(L)$ is a line in $D$.
\end{lemma}

\begin{proof}
	As $\phi_0\in\Mobgp$, it preserves angles, and it takes Euclidean lines and circles to Euclidean lines and circles. Also, $\phi_0(\boundary{H}) = \boundary{D}$. Thus, if $L$ is a line in $H$, then $\phi_0(L)$ is a line in $D$.

	The converse is similar.
\end{proof}

\begin{lemma}
	Given $\va \neq \vec{0}$, there is a unique hyperbolic line through $\vec{0} \in D$ which is tangent to $\va$ at $\vec{0}$.
\end{lemma}

\begin{proof}
	First we show that any line through $\vec{0}$ is a diameter of $D$. Suppose $C$ is a Euclidean circle passing through $\vec{0}$, perpendicular to $\boundary{D}$. Let $B$ be its centre.

	% \begin{center} % Geo 12/3
	% 	\begin{tikzpicture}
	% 		\draw [thick, red] (0,0) circle (1.4);
	% 		\draw [thick, blue] (2,0) circle (2);
	% 		\draw (0,0) node [below left] {$O$} -- (2,0) node [below right] {$B$} -- (0.49,1.31) -- cycle;
	% 		\draw (0.42,1.31) node [above=2pt] {$A$};

	% 		\foreach \s/\t in {0/0, 2/0, .49/1.31} {\draw (\s,\t) node {$\bullet$};}
	% 	\end{tikzpicture}
	% \end{center}

	Let $A$ be a point in $C \cap \boundary{D}$, then $\triangle OAB$ is isoceles. Thus $\angle OAB = \pi/2 = \angle AOB$, and so the sum of the angles is more than $\pi$. But this is ridiculous.

	The lemma now follows, sine there's a unique Euclidean line through $\vec{0}$ with direction vector $\va$.
\end{proof}

\begin{corollary}
	If $p\in\HH$ and $\va\neq 0$, then there's a unique ray stating at $\pp$ with direction vector $\va$.
\end{corollary}

\begin{proof*}
	[In $D$] Choose $\psi \in G^D$ with $\psi(\pp)=0$, such as $\psi(z) = \df{z-p}{\overline{p}z-1}$.

	Then there's a unique ray $R\p$ starting at $\vec{0}$ with direction $\dif{\psi_p(\va)} \neq 0$. Then the ray $R$ which we want is $R=\psi^{-1}(R\p)$. \qed
\end{proof*}

\begin{proposition}
	If $R_1,R_2$ are hyperbolic rays starting at $p_1,p_2 \iH$, then there is $\psi\in G$ with $\psi(p_1) = p_2$, $\psi(R_1)=R_2$.
\end{proposition}

\begin{proof*}
	[In $D$] Let $R_0$ be the positive real axis. Let
	\begin{equation*}
		\psi_1(z) = \f{z-p_1}{\overline{p_1}z-1}.
	\end{equation*}
	If $\psi_1(p_1)=0$, then $\psi_1(R_1)$ is a radius of $D$.
	\begin{center}
		\begin{tikzpicture}[scale=2]
			\draw (0,0) circle (1) node {$\bullet$};

			\draw (1,0) -- (0,0) -- ++ (60:1);
			\draw (0.3,0) arc (0:60:0.3);
			\draw (0.25,0.2) node [right] {$\theta$};

			\draw (0.5,0) node [below] {$R_0$};
		\end{tikzpicture}
	\end{center}

	Let $\psi_R(z) = e^{-i\theta} \psi_1(z)$, where $\theta$ is the angle between $R_0$ and $\psi_1(R_1)$. Then $\psi_{R_1}(R_1) = R_0$. Construct $R_2$ similarly, and then take $\psi = \psi_{R_2}^{-1} \circ \psi_{R_1}$. \'.e'
\end{proof*}

\begin{proposition}
	There's a unique line containing two distinct points $p_1,p_2 \iH$.
\end{proposition}

\begin{proof*}
	[In $D$] Choose $\psi \in G^D$ with $\psi(p_1) = \vec{0}$. There's a unique hyperbolic line $L$ containing $\vec{0}$ and $\psi(p_2)$, namely through the diameter of $D$ through $\psi(\overline{p_2})$. Thus $\psi^{-1}(L)$ is the unique line containing $p_1$ and $p_2$. \qed
\end{proof*}

\begin{proposition}
	Two lines $L_1$ and $L_2$ intersect in at most one point in $\HH$.
\end{proposition}

\begin{proof*}
	[In $H$] After applying an element $\psi\in G^H$, we may assume that
	\begin{itemize}
		\shortskip
		\item $\psi(L_1)$ is the positive imaginary axis.
		\item $\psi(L_2)$ is (i) a circle centred on the real axis or (ii) a vertical line
	\end{itemize}
	Consider the two cases for $\psi(L_2)$:
	\begin{enumerate}
		\shortskip
		\item At most one intersection with $\psi(L_1)$ in $H$ (other is in the lower half-plane);
		\item Has none. \qed
	\end{enumerate}
\end{proof*}

This final proposition motivates the following definition:

\begin{definition}
	We say that $L_1$ and $L_2$ are \emph{haroparallel} if they intersect in $\boundary{\HH}$ and are \emph{ultraparallel} if they do not intersect in $\boundary{\HH}$.

	\begin{center}
		\begin{minipage}{0.4\textwidth} % Geo 12/5
		\centering
		\begin{tikzpicture}
			\draw (0,0) -- (5,0);
			\draw [very thick] (1,0) -- (1,2);
			\draw [very thick] (3,0) arc (0:180:1);
		\end{tikzpicture}
	\end{minipage}
	\hspace{0.2cm}
		\begin{minipage}{0.4\textwidth}
		\centering
		\begin{tikzpicture}
			\draw (0,0) -- (5,0);
			\draw [very thick] (1,0) -- (1,2);
			\draw [very thick] (4,0) arc (0:180:1);
		\end{tikzpicture}
	\end{minipage}

	\begin{minipage}{0.4\textwidth}
		\centering
		haroparallel
	\end{minipage}
	\hspace{0.2cm}
	\begin{minipage}{0.4\textwidth}
		\centering
		ultraparallel
	\end{minipage}
	\end{center}

	If $L$ is a line, $p\not\in L$, then there are infinitely many ultraparallel lines to $L$ that pass through $p$.
\end{definition}

	\pagebreak

Now we consider distance; specifically, the shortest distance between two points.

\begin{proposition}
	If $p,q\iH$, then the line segment from $\pp$ to $\qq$ is the shortest path from $\pp$ to $\qq$ in $H$.
\end{proposition}

\begin{proof*}
	[In $H$] Let $L$ be the unique line segment from $\pp$ to $\qq$. After composing with $\psi\in G$, we may assume that $L$ is the positive real axis.

	So we have $p=ia$, $q=ib$, $a,b\iR$.  Let $\gamma(t)$ be a path from $\pp$ to $\qq$ in $H$. Then
	\begin{equation*}
		L_{g^H}(\gamma)
		= \int_0^1 \sqrt{\f{\gamma_1^2+\gamma_2^2}{\gamma_2^2}} \dif{t}
		\geq \int_0^1 \left\vert \f{\gamma_2\p(t)}{\gamma_2\p(t)} \right\vert \dif{t}
		\geq \left\vert \int_0^1 \f{\gamma\p(t)}{\gamma(t)} \dif{t} \right\vert
		= \left\vert \ln b - \ln a \right\vert
		= \left\vert \ln(b/a) \right\vert.
	\end{equation*}
	with equality if and only if $\gamma_1\p=0$ and $\gamma_2\p$ has constant sign; that is, if $\gamma$ is a vertical line segment. \qed
\end{proof*}

\begin{corollary}
	The distance from $ia$ to $ib$ in $H$ is $\left\vert \ln(b/a) \right\vert$.
\end{corollary}

\begin{corollary}
	The distance from $\vec{0}$ to $re^{i\theta}$ in $D$ is $\ln\left[ (1+r)/(1-r) \right] = 2\tanh^{-1} r$.
\end{corollary}

% subsection geometry_of_the_hyperbolic_plane (end)

\subsection{Isometries of the hyperbolic plane} % (fold)
\label{sub:isometries_of_the_hyperbolic_plane}

\lecturemarker{13}{4 Mar}
We extend complex conjugation to a map $c:\C_\infty \to \C_\infty$ by setting $c(\infty) = \infty$. Now, if $\phi_A$ is the Möbius transformation defined by the matrix $A\in\GL_2(\C)$, then we see that
\begin{equation*}
	\phi_A \circ c = c \circ \phi_{\bar{A}}.
\end{equation*}

\begin{definition}
	The \emph{extended Möbius group} is given by
	\begin{equation*}
		\overline{\Mobgp} = \left\{ \phi:\C_\infty \to \C_\infty : \phi\in\Mobgp \text{or } \phi\circ c\in\Mobgp \right\}.
	\end{equation*}
\end{definition}

We observe that $c^2=\iota$, so the second condition is equivalent to saying that $\phi = \psi\circ c$, where $\psi\in\Mobgp$. It follows from our first equation that the extended Möbius group is closed under composition, and thus it indeed forms a group, containing the Möbius group as an index two subgroup.

Then elements of $\Mobgp$ are \emph{orientation preserving}, while elements of $\overline{\Mobgp}$ not in $\Mobgp$ are \emph{orientation reversing}. We can compare this to rotations and reflections in Euclidean geometry. We already have our rotations (given by Möbius maps), so now let's consider reflections:

\begin{definition}
	If $C\subset\C_\infty$ is a Euclidean line or circle, the reflection in $C$ is the extended Möbius transformation defined by
	\begin{equation*}
		R_C = \psi^{-1} \circ c \circ \psi,
	\end{equation*}
	where $\psi\in\Mobgp$ satisfies $\phi(C) = \R\cup\{\infty\}$.
\end{definition}

Hopefully this definition is reasonably intuitive; now we just need to check that it makes sense. We need to check that our choice of $\psi$ doesn't matter. So suppose we have $\psi\p(C)=\R\cup\{\infty\}$. Then $\psi\p \circ \psi^{-1}(\R) = \R$, and so $\psi\p \circ \psi^{-1} = \phi_A$, for some $A\in\GL_2(\R)$. (As in the Euclidean plane, two reflections form a rotation.) Then
\begin{equation*}
	\psi^{\,\prime^{-1}} \circ c \circ \psi\p
	= \psi^{-1} \circ \phi^{-1}_A \circ c \circ \phi_A \circ \psi
	= \psi^{-1} \circ c \circ \psi,
\end{equation*}
and so $R_C$ is well-defined.

\begin{example}
	If $C$ is the unit circle, then $R_C=\psi_0 \circ c \circ \psi_0^{-1}$, and so
	\begin{equation*}
		R_C(z)
		= \psi_0\left( -i\,\f{\zbar+1}{\zbar-1} \right)
		= \f{-i\f{\zbar+1}{\zbar-1} - i}{-i\f{\zbar+1}{\zbar-1} + i}
		= \f{1}{\zbar}.
	\end{equation*}
	More generally, if $C_r$ is a circle of radius $r$ centred at $0$, then $R_{C_r} = \psi_r \circ R_{C_1} \circ \psi_{1/r}$, where $\psi_a(z) = az$. Thus $R_{C_r} = r^2/\zbar$.
\end{example}

\begin{proposition}
	$\overline{\Mobgp}$ is generated by reflections.
\end{proposition}

\begin{proof}
	It is enough to check that the maps
	\begin{enumerate}
		\shortskip
		\item $z\mapsto z+b$, $b\iC$;
		\item $z\mapsto az$, $a\iC$, $a\neq 0$;
		\item $z\mapsto 1/z$;
	\end{enumerate}
	are compositions of reflections, since these maps generate $\Mobgp$.
	
	Map (i) is generated by $R_{L_1} \circ R_{L_2}$, where $L_1$ and $L_2$ are two Euclidean lines perpendicular to $b$ and separated by a distance $b/2$.

	For map (ii), multiplication by $a\iR$ is $R_{C_1} \circ R_{C_2}$, where $C_2$ is the unit circle and $C_1$ is a circle of radius $\sqrt{a}\,$ centred at the origin, while multiplication by $e^{i\theta}$ is $R_{L_1} \circ R_{L_2}$, where $L_1$ and $L_2$ are two lines which intersect in an angle $\theta/2$ at the origin. Using these two maps, we can compose for any $a\iC\backslash\{0\}$.

	Finally, map (iii) is the composition of reflection in the unit circle with reflection in $\R$. This completes the proof.
\end{proof}

We can view the groups $\Isom(S^2)$, $\Isom(\R^2)$ and $\Isom(D)$ as subgroups of the extended Möbius group, corresponding to the extension of the following subgroups of $\Mobgp$ by $c$:
\begin{align*}
	\Isom^+(S^2) &= \left\{\phi_A : A = \mat{\alpha & \beta \\ -\overline{\beta} & \overline{\alpha}}, \det A = 1 \right\}, \\
	\Isom^+(\R^2) &= \left\{\phi_A : A = \mat{\alpha & \beta \\ 0 & \overline{\alpha}}, \det A = 1 \right\}, \\
	\Isom^+(D) &= \left\{\phi_A : A = \mat{\alpha & \beta \\ \overline{\beta} & \overline{\alpha}}, \det A = 1 \right\}, \\
\end{align*}

% subsection isometries_of_the_hyperbolic_plane (end)   % Hyperbolic geometry
%!TEX root = geometry.tex
\stepcounter{lecture}
\setcounter{lecture}{6}
\sektion{Geodesics}

\newcommand{\tgamma}{\tilde{\gamma}}

Let $g$ be a Riemannian metric on some open set $U\subset \R^2$, say
\begin{equation*}
	g = E \dif{x^2} + 2F \dif{x} \dif{y} + G \dif{y^2}.
\end{equation*}
The basis problem we want to answer is: given $p,q\in U$, how can we find the shortest path with respect to $g$ from $p$ to $q$, supposing it exists? These shortest paths are the analogues of lines in hyperbolic space.

\subsection{Energy functionals} % (fold)
\label{sub:energy_functionals}

Let $\gamma:[0,1] \to U$ be a smooth path. As we've seen before the length is
\begin{equation*}
	L_g(\gamma) = \int_0^1 \left.|\gamma\p(t)|\right._g \dif{t}.
\end{equation*}
This is invariant under reparameterisation:
\begin{equation*}
	L_g(\gamma\circ f) = L_g(\gamma),
\end{equation*}
where $f:[0,1] \to [0,1]$ is monotone and continuous.

Now we introduce a new function, which is not invariant under reparameterisation. This might seem like a bad thing, but actually it makes our lives a lot easier.

\begin{definition}
	The \emph{energy} of a smooth path $\gamma:[0,1] \to U$ is
	\begin{equation*}
		E_g(\gamma) = \int_0^1 \left.|\gamma\p(t)|\right._g^2 \dif{t}.
	\end{equation*}
\end{definition}

Recall the Cauchy-Schwarz inequality, which we've met in many different contexts:
\begin{equation*}
	\textstyle \int ab \leq \sqrt{\int a^2} \, \cdot \sqrt{\int b^2},
\end{equation*}
with equality if and only if $a=\lambda b$ or $b=\lambda a$, for some $\lambda$. Then
\begin{equation*}
	L_g(\gamma)
	= \int_0^1 \left.|\gamma\p(t)|\right._g \dif{t}
	\leq \sqrt{\int_0^1 \left.|\gamma\p(t)|\right._g^2 \dif{t}} \, \cdot \sqrt{\int_0^1 \dif{t}}
	= \sqrt{E_g(\gamma)},
\end{equation*}
with equality if and only if $\left.|\gamma_p(t)|\right._g = \lambda \cdot 1$ (a constant function); that is, if $\gamma$ has constant speed. Now, every $\gamma$ with $\gamma\p(t) \neq 0$ for all $t$ has a constant speed reparametrisation; consider
\begin{equation*}
	\tgamma = \gamma \circ f, \qquad
	f:[0,1] \to [0,1], \qquad
	f(s) = F^{-1}(s),
\end{equation*}
where $F:[0,1] \to [0,1]$ is given by 
\begin{equation*}
	F(s) = \f{\int_0^s \left.|\gamma_p(t)|\right._g \dif{t}}{\int_0^1 \left.|\gamma_p(t)|\right._g \dif{t}}.
\end{equation*}

\begin{definition}
	If $p,q \in U$, then the set of paths from $p$ to $q$ is given by $\Omega_{p,q}$; formally,
	\begin{equation*}
		\Omega_{p,q} = \left\{\text{$\gamma:[0,1]\to U$ smooth}: \gamma(0)=p, \gamma(1)=q\right\}.
	\end{equation*}
\end{definition}

\begin{proposition}
	The following two conditions are equivalent:
	\begin{enumerate}
	    \shortskip
		\item $E(\gamma_0) \leq E(\gamma)$ for all $\gamma\in\Omega_{p,q}$;
		\item $L(\gamma_0) \leq L(\gamma)$ for all $\gamma\in \Omega_{p,q}$, and $\gamma$ has constant speed.
	\end{enumerate}
\end{proposition}

\begin{proof}
	(i) $\implies$ (ii). If $\tgamma_0$ has constant then, then
	\begin{equation*}
		E(\tgamma_0)
		= \left[ L(\tgamma_0) \right]^2
		= \left[ L(\gamma_0) \right]^2
		\leq E(\gamma_0),
	\end{equation*}
	with equality if and only if $\gamma_0$ has constant speed; that is, $\gamma_0 = \tgamma_0$.

	(ii) $\implies$ (i). We have
	\begin{equation*}
		E(\gamma_0)
		= \left[ L(\gamma_0) \right]^2
		\leq \left[ L(\tgamma_0) \right]^2
		\leq E(\gamma),
	\end{equation*}
	which is what we require.
\end{proof}

\vspace{3pt}

\begin{note}
	If $\gamma\p(a) = 0$ for some $a\in[0,1]$, then can always find $F$ such that $E(\gamma \circ f) < E(\gamma)$.
\end{note}

% subsection energy_functionals (end)

\subsection{Calculus of variations} % (fold)
\label{sub:calculus_of_variations}

Given $H=H(x,y,z,w)$, suppose $\gamma\in\Omega_{p,q}$ minimises
\begin{equation*}
	\Phi(y) = \int_0^1 H(\gamma_1(t), \gamma_2(t), \gamma\p_1(t), \gamma\p_2(t)) \dif{t}
\end{equation*}
if, for example,
\begin{equation*}
	H(x,y,z,w) = E(x,y)\,z^2 + 2F(x,y) \, zw + G(x,y)\,w^2.
\end{equation*}
Then $\Phi(\gamma) = E_g(\gamma)$.

For any $\delta:[0,1] \to \R^2$ with $\delta(0)=\delta(1)=0$, if
\begin{equation*}
	(\gamma+\epsilon\delta)(t) = \gamma(t) + \epsilon\,\delta,
\end{equation*}
then $\gamma+\epsilon\delta \in \Omega_{p,q}$, when $\epsilon\ll1$. Thus $\epsilon=0$ minimises $\Phi(\gamma+\epsilon\delta)$:
\begin{align}
	0
	&= \dod{}{\epsilon}\Phi(\gamma+\epsilon\delta) \notag \\
	&= \int_0^1 \dod{}{\epsilon}\left[ (H(\gamma_1+\epsilon\delta_1, \gamma_2+\epsilon\delta_2, \gamma_1\p + \epsilon\delta_1\p, \gamma_2\p+\epsilon\delta_2\p) \right] \dif{t} \notag \\
	&= \int_0^1 \left[ H_x \delta_1 + H_y \delta_2 + H_z \delta_1\p + H_w \delta_2\p \right] \dif{t}. \label{eq:calc-variations}
\end{align}
Now we have
\begin{equation*}
	\int_0^1 H_z \delta_1\p \dif{t}
	= \left[ H_z \delta_1 \right]_0^1 - \int_0^1 \od{}{t}(H_z) \,\delta_1 \dif{t}
	= -\int_0^1 \od{}{t}(H_z)\,\delta_1 \dif{t},
\end{equation*}
as $\delta$ is a closed curve.

Returning to~\eqref{eq:calc-variations}, we see that
\begin{equation*}
	\int_0^1 \left[ \left( H_x - \dod{H_z}{t} \right)\delta_1 + \left( H_y - \dod{H_w}{t} \right)\delta_2 \right] \dif{t}
	\equiv 0,
\end{equation*}
for any $\delta_1,\delta_2$ with $\delta_i(0)=\delta_i(1) = 0$, $i=1,2$.

This gives us the \emph{Euler-Lagrange equations}:
\begin{equation*}
	H_x=\od{H_z}{t} \qqand
	H_y = \od{H_w}{t}.
\end{equation*}

% subsection calculus_of_variations (end)

\subsection{Geodesic equations} % (fold)
\label{sub:geodesic_equations}

\newcommand{\dgamma}{\dot{\gamma}}

In our case, we have
\begin{equation*}
	H(x,y,z,w) = E(x,y) \, z^2 + 2F(x,y)\,zw + G(x,y)\,w^2.
\end{equation*}
Simple differentiation gives us
\begin{equation*}
	H_x = E_x\,z^2 + 2F_x\,zw + G_x\,w^2 \qqand
	H_z = 2Ez + 2Fw.
\end{equation*}
Now we write $E(x,y) = E(\gamma_1(t), \gamma_2(t))$, and similar for $F$ and $G$. Letting a dot denote differentiation with respect to $t$, and substituting into the Euler-Lagrange equations, we obtain the \emph{geodesic equations}
\begin{align*}
	E_x \dot{\gamma}_1^2 + 2F_x \dgamma_1 \dgamma_2 + G_x \dgamma_2^2 &= \dod{}{t}(2E\dgamma_1 + 2F\dgamma_2), \\
	E_y \dgamma_1^2 + 2 F_y \dgamma_1 \dgamma_2 + G_y \dgamma_2^2 &= \dod{}{t}(2F\dgamma_1 + 2G\dgamma_2).
\end{align*}
This is a (nasty!) system of second-order differential equations.

\begin{definition}
	A path $\gamma:[a,b]\to U$ is a \emph{geodesic} if it satisfies the geodesic equations, or if is a critical points for the energy functional.

	A shortest length, constant speed path is a geodesic.
\end{definition}

\begin{theorem}
	Given $\pp\in U$ and $\xx\iR^2$, there is a unique geodesic $\gamma:(-\epsilon,\epsilon)\to U$ with $\gamma(0)=\pp$, $\gamma\p(0)=\xx$.
\end{theorem}

\begin{proof}
	This is an immediate consequence of the existence and uniqueness of solutions for ordinary differential equations.
\end{proof}

% subsection geodesic_equations (end)

	\pagebreak

\subsection{Exponential map} % (fold)
\label{sub:exponential_map}

\lecturemarker{14}{6 Mar}
\newcommand{\gvt}{\gamma_{\vv_\theta}}
\newcommand{\dGamma}{\dot{\Gamma}}

For $\pp\in U$, $\vv\iR^2$, there's a unique geodesic $\gamma_\vv:(-\epsilon,\epsilon) \to U$ with $\gamma(0)=\pp$, $\gamma\p(0)=\vv$. So why can't we extend this over all of $\R$? There are lots of reasons. Consider, for example, $U=\R^2\backslash\{0\}$. There is not geodesic linking the points $-x$ and $x$, $x\iR$, since we cannot go through the point $0$.

\begin{lemma}
	$\gamma_\vv(\lambda t) = \gamma_{\lambda\vv}(t)$, $\lambda\iR$.
\end{lemma}

\begin{proof}
	If $\gamma(t)$ satisfies the geodesic equations, so does $\gamma(\lambda t)$ (Both sides get multiplied by $\lambda^2$.) So $\overline{\gamma}=\gamma(\lambda t)$ is a geodesic with $\overline{\gamma}(0) = \gamma(0) = \pp$, $\overline{\gamma}\p(0) = \lambda\,\overline{\gamma}\p(0) = \lambda\vv$, and this gives us $\overline{\gamma} = \gamma_{\lambda\vv}$.
\end{proof}

\begin{definition}
	The \emph{exponential map} $\exp_\pp:B_\epsilon(0) \to U$ is given by
	\begin{equation*}
		\exp_\pp(\vv) = \gamma_\vv(1)
	\end{equation*}
\end{definition}

Note $\exp_\pp(\lambda \vv) = \gamma_{\lambda\vv}(1) = \gamma_\vv(\lambda)$, so $\exp_\pp(\vv)$ is defined for $|\vv|$ small.

\begin{proposition}
	$\eval[0]{\dif{\,\exp_\pp}}_0 = I$.
\end{proposition}

\begin{proof}
	Working from the definition, we have:
	\begin{align*}
		\eval[0]{\dif{\,\exp_\pp}}_0
		&= \lim_{\epsilon\to 0} \f{\exp_p(\epsilon\ww)-\exp_p(0)}{\epsilon} \\
		&= \lim_{\epsilon\to 0} \f{\gamma_{\epsilon\ww}(1) - \gamma_0(1)}{\epsilon} \\
		&= \lim_{\epsilon\to 0} \f{\gamma_\ww(\epsilon) - \gamma_\ww(0)}{\epsilon} \\
		&= \gamma\p_\ww(0) = \ww. \qedhere
	\end{align*}
\end{proof}

\begin{corollary}
	There are open sets $V_1 \subset \R^2$, $V_2\subset U$ with $0\in V_1, \pp\in V_2$, such that $\exp_p:V_1 \to V_2$ is a diffeomorphism; that is, differentiable, bijective and the inverse is differentiable.
\end{corollary}

\begin{proof}
	This follows from the inverse function theorem, since $I$ is invertible. Equivalently, we cause $\exp_p$ to define a new set of coords on $V_2$.
\end{proof}

% subsection exponential_map (end)

\subsection{Geodesic polar coordinates} % (fold)
\label{sub:geodesic_polar_coordinates}

Pick $\vv_1,\vv_2$ orthogonal with respect to the Riemannian metric $g_\pp$. Then we define
\begin{equation*}
	\vv_\theta = \vv_1 \cos\theta + \vv_2 \sin\theta.
\end{equation*}
This allows us to define the map
\begin{equation*}
	\fullfunction{T}{[0,\epsilon)}{[0,2\pi) \times U}{(r,\theta)}{\exp_p(r\vv_\theta)}.
\end{equation*}
These define a set of \emph{geodesic polar coordinates}.

Let $\overline{g} = \overline{E} \dif{r^2} + 2\overline{F} \dif{r}\dif{\theta} + \overline{G} \dif{\theta^2}$ be the metric induced from $g$ using $T$; that is,
\begin{equation*}
	\overline{E} = g(T_r,T_r),
	\qquad \overline{F} = g(T_r,T_\theta),
	\qquad \overline{G} = (T_\theta,T_\theta).
\end{equation*}

\begin{example}
	 Consider the Euclidean metric $g = g^E = \dif{x^2} + \dif{y^2}$, with exponential map $T(r,\theta) = \exp_0(r\vv_\theta) = r\vv_\theta$. Then
	 \begin{align*}
	 	x = r\cos\theta, & \qquad \dif{x} = \dif{r} - r\sin\theta \dif{\theta}, \\
	 	y = r\sin\theta, & \qquad \dif{y} = \dif{r} + r\cos\theta \dif{\theta}.
	 \end{align*}
	 Thus we have
	 \begin{equation*}
	 	\overline{g} = \dif{x^2} + \dif{y^2} = \dif{r^2} + r^2 \dif{\theta^2}.
	 \end{equation*}
\end{example}

We consider a similar problem on the third examples sheet:

\begin{example}
	We consider the metric on the disc:
	\begin{equation*}
		g = g^D = \f{4\left( \dif{x^2}+\dif{y^2} \right)}{\left( 1-x^2-y^2 \right)^2}.
	\end{equation*}
	Then our map is given by $T(r,\theta) = 2\tanh^{-1} r \vv_\theta$, and the induced metric is
	\begin{equation*}
		\overline{g} = \dif{r^2} + \sinh^2 r \dif{\theta}^2.
	\end{equation*}
\end{example}

There's a common pattern in both of these examples:

\vspace{3pt}

\begin{proposition}
	Under the notation established thus far, we have $\overline{E}=1$, $\overline{F}=0$, $\overline{G} = r^2 \tilde{G}(r,\theta)$ where $\lim_{r\to 0} \tilde{G}(r,\theta) = 1$.
\end{proposition}

\begin{proof}
	First we need a lemma:

	\vspace{8pt}

	\begin{lemma}
		Geodesics have constant speed:
		\begin{equation*}
			\left.|\gamma\p_\vv(t)|\right._g = \left.|\gamma\p_\vv(0)|\right._g = \left.|\vv|\right._{g_0}.
		\end{equation*}
	\end{lemma}

	\vspace{-3pt}

	Proof of the lemma is on the examples sheet. Now we can prove the proposition, tackling each function in turn:
	\begin{enumerate}
		\item From our previous work, we have
		\begin{equation*}
			T_r = \pd{}{r}(\gamma_{\vv_\theta}(r)) = \gamma\p_{\vv_\theta}(r).
		\end{equation*}
		Using our lemma, we thus have
		\begin{equation*}
			\overline{E}
			= g(T_r,T_r)
			= g(\gamma\p_{\vv_\theta}, \gamma\p_{\vv_\theta})
			= g(\gamma\p_{\vv_\theta}(0), \gamma\p_{\vv_\theta}(0))
			= g_0(\vv_\theta, \vv_\theta)
			= 1.
		\end{equation*}

		\item First consider the energy functional
		\begin{equation*}
			E_g^{[0,r]}(\gamma_{\vv_\theta})
			= \int_0^r g(\gamma\p_{\vv_\theta}, \gamma\p_{\vv_\theta}) \dif{t}
			= \int_0^r 1 \dif{t}
			= r.
		\end{equation*}
		Thus we have
		\begin{equation*}
			0
			= \pd{}{\theta}\left[ E_g^{[0,r]}(\gamma_{\vv_\theta}) \right]
			= \pd{}{\epsilon}\left[ E_g^{[0,r]}(\gvt + \epsilon\delta) \right],
		\end{equation*}
		where $\delta(t) = \pd{}{\theta}(\gvt(t)) = T_\theta(t,\theta)$.

		From our derivativation of the geodesic equations, we know
		\begin{align}
			&\dpd{}{\epsilon}\left[ E_g^{[0,r]}(\gvt + \epsilon\delta) \right] \notag \\
			&\qquad= \underbrace{\left[ H_z \delta_1 \right]_0^r + \left[ H_w \delta_2 \right]_0^r}_{\eqref{eq:calc-variations}} + \int_0^r \left( H_x - \dod{H_z}{t} \right) \delta_1 + \left( H_y - \dod{H_w}{t} \right) \delta_2 \dif{t}.
			\label{eq:induced-metric-geodesic}
		\end{align}
		The integral cancels to zero, since $\gvt$ is a geodesic. Hence,~\eqref{eq:calc-variations} and~\eqref{eq:induced-metric-geodesic} give
		\begin{equation*}
			2 \left[ \left.(E\dgamma_1 + F\dgamma_2)\right.\delta_1 + \left.(F\dgamma_1 + G\dgamma_2)\right. \delta_2 \right]
			= 2g(\gvt\p, \delta) = 0.
		\end{equation*}
		But $2g(T_r,T_\theta) = 2g(\gvt\p,\delta)$, so $\overline{F}=0$.

		\item First we consider
		\begin{equation*}
			T_\theta(0,\theta) = \pd{}{\theta}(\gvt(0)) = \pd{\pp}{\theta} = \vec{0}
		\end{equation*}
		Then we have
		\begin{align*}
			\dpd{T}{r}(T_\theta)(0,\theta) = T_{\theta r}(0,\theta) = T_{r\theta}(0,\theta)
			&= \dpd{}{\theta}T_r(0,\theta) \\
			&= \dpd{}{\theta}(\gvt\p(0)) \\
			&= \dpd{}{\theta}(\vv_\theta) \\
			&= -\vv_1 \sin\theta + \vv_2 \cos\theta = \vv_\theta^\perp
		\end{align*}
		Unpacking this, we deduce that
		\begin{equation*}
			T_\theta(r,\theta) = r\vv(r,\theta),
		\end{equation*}
		where $\lim_{r\to 0} \vv(r,\theta) = \vv_\theta^\perp$. Now
		\begin{equation*}
			\overline{G} = g(T_\theta, T_\theta) = r^2 g(\vv,\vv) = r^2 \tilde{G}(r,\theta),
		\end{equation*}
		where
		\begin{equation*}
			\lim_{r\to 0} \tilde{G}(r,\theta) = \lim_{r \to 0} g(\vv,\vv) = \eval[0]{g}_0(\vv_\theta^\perp, \vv_\theta^\perp) = 1. \qedhere
		\end{equation*}
	\end{enumerate}
\end{proof}

% subsection geodesic_polar_coordinates (end)

\subsection{Local Gauss-Bonnet} % (fold)
\label{sub:local_gauss_bonnet}

First we set up an open set $U\subset \R^2$ with geodesic polar coordinates $g=\dif{r^2} + G\dif{\theta^2}$, as discussed in the previous section. Let $\triangle OBC$ be a geodesic triangle.

\begin{center}
	\begin{tikzpicture}[scale=2.2]
		\draw (0,0) node {$\bullet$} node [below left] {$O$}
		-- ++ (0:2) node {$\bullet$} node [below right] {$B$}
		-- ++ (120:2) node {$\bullet$} node [above=2pt] {$C$} -- cycle;

		\draw [thick] (0,0)
			arc (140:90:1)
			arc (270:307.5:1.4) node {$\bullet$} 
			arc (307.5:340:1.4);

		\draw [thick] (0,0)
			arc (140:90:1)
			arc (270:307.5:1.4) ++ (0,-0.05) node [right=4pt] {$P_\theta$};

		\draw [dashed] (0,0) -- ++ (50:0.8);
		\draw (0.4,0) arc (0:50:0.4);
		\draw (0.35,0.15) node [right=2pt] {$\alpha$};

		\draw (1.6,0) arc (180:120:0.4);
		\draw (2,0) ++ (120:1.6) arc (300:240:0.4);

		\draw (1,1.4) node [below=4pt] {$\gamma$};
		\draw (1.76,0.22) node [left=5pt] {$\beta$};

		\draw [dashed] (0,0) arc (140:90:1) arc (270:307.5:1.4) -- ++ (37.5:0.8);
		\draw (0,0) arc (140:90:1) arc (270:307.5:1.4) -- ++ (37.5:0.3) arc (37.5:120:0.3);

		\draw (1.7,0.92) node [above] {$\phi(\theta)$};
	\end{tikzpicture}
\end{center}

Geodesic triangles are formed by the arcs of three geodesics on a curved surface; straight lines are used above only for illustrative purposes. Here we have
\begin{align*}
	OB &= \left\{\theta=0, r\in[0,b]\right\}, \\
	OC &= \left\{\theta=\alpha, r\in[0,c]\right\}, \\
	BC &= \left\{\Gamma(\theta) = (f(\theta,\theta)), \theta\in[0,1]\right\}.
\end{align*}
Now let $P_\theta=\Gamma(\theta)$. Then $\phi(\theta)$ is the angle between $OP_\theta$ and $BC$. Note that $\phi(0) = \pi-\beta$ and $\phi(\alpha)=\gamma$.

The length of this curve $BP_\theta$, given by
\begin{equation*}
	s(\theta) = \int_0^\theta |\Gamma\p(u)| \dif{u}.
\end{equation*}
We then define
\begin{equation*}
	h(\theta) := \od{s}{\theta} = \left.|\Gamma\p(\theta)|\right._g
	\qquad
	\text{which gives us}
	\qquad
	\od{f}{s} = \od{f}{\theta} \od{\theta}{s} = \f{f\p(\theta)}{h(\theta)}.
\end{equation*}

\begin{lemma}
	$\disp \dod{}{s}\left( \f{f\p}{h} \right) = \f{G_r}{2h}.$
\end{lemma}

\begin{proof}
	If we parametrise by arc length $\Gamma$, then it satisfies the geodesic equations. Letting a dot denote differentiation with respect to $s$:
	\begin{equation*}
		\od{}{s}\left[ 2E\,\dGamma_1 + 2F\,\dGamma_2 \right] = E_r\,\dGamma_1^2 + F_r\,\dGamma_1 \, \dGamma_2 + G_r\,\dGamma_2^2.
	\end{equation*}
	Most terms didsppaear, leaving
	\begin{equation*}
		\od{}{s}\left( 2\dod{f}{s} \right) = G_r \left( \dod{\theta}{s} \right)^2.
	\end{equation*}
	Finally, this gives us
	\begin{equation*}
		2\,\od{}{s}\left( \f{f\p}{h} \right) = \f{G_r}{h^2},
	\end{equation*}
	which is easily rearranged to give the result.
\end{proof}

\begin{lemma}
	$\od{\phi}{\theta} \equiv \phi\p = (-\sqrt{G}\,)r.$
\end{lemma}

\begin{proof}
	In the diagram above, $OP_\theta$ is a ray of constant $\theta$, parameterised by $\rho(u) = (u,\theta)$ and $\rho\p(u) = (1,0)$. Now $\Gamma\p=(f\p,1)$, and then
	\begin{equation*}
		\cos \phi
		= \f{g(\Gamma\p, \rho)}{|\Gamma\p|_g |\rho\p|_g}
		= \f{f\p}{h \cdot 1}
		= \f{f\p}{h}. \tag{$*$}
	\end{equation*}
	Now we also consider
	\begin{equation*}
		\sin^2\phi
		= 1-\cos^2\phi
		= 1 - \left( \f{f\p}{h} \right)^2
		= 1-\f{(f\p)^2}{(f\p)^2 + G}
		= \f{G}{(f\p)^2 + G}
		= \f{G}{h^2}. \tag{$**$}
	\end{equation*}
	We thus have $\sin \phi = \sqrt{G}/h$. Then we differentiate $(*)$:
	\begin{equation*}
		-\phi\p \sin\phi = \left( \f{f\p}{h} \right)\p = \f{G_r}{2h},
	\end{equation*}
	by the previous lemma. Then
	\begin{equation*}
		\phi\p
		= -\f{G_r}{2h \sin\theta}
		= -\f{G_r}{2h} \left( \f{h}{\sqrt{G}} \right)
		= -\f{G_r}{2\sqrt{G}}
		= -(\sqrt{G}\,) r. \qedhere
	\end{equation*}
\end{proof}

This leads us to one of the main theorems of this chapter:

\begin{theorem}
	[Local Gauss-Bonnet theorem] For a geodesic triangle as described previously,
	\begin{equation*}
		\defect(OBC)
		= \alpha+\beta+\gamma-\pi
		= \iint_{OBC} \f{-(\sqrt{G}\,)_{rr}}{\sqrt{G}} \dif{A_g}.
	\end{equation*}
\end{theorem}

\begin{proof}
	We have $\dif{A_g} = \sqrt{\deg g} = \sqrt{G}$\,, so
	\begin{align*}
		\iint_{OBC} \f{(-\sqrt{G}\,)_{rr}}{\sqrt{G}} \dif{A_g}
		&= \int_0^\alpha \int_0^{f(\theta)} (-\sqrt{G}\,)_{rr} \dif{r} \dif{\theta} \\
		&= \int_0^\alpha \left[ -(\sqrt{G}\,)_r \right]_0^{f(\theta)} \dif{\theta} \\
		&= \int_0^\alpha \left[ (\sqrt{G}\,)_r \right]_{r=0} - \left[ (\sqrt{G}\,)_r \right]_{r=f(\theta)} \dif{\theta}.
		\intertext{Note $G=r^2 \tilde{G}$, with $\tilde{G} \to 1$ as $r\to 0$. Thus $\sqrt{G} = r \sqrt{\tilde{G}}$\,, and so $(\sqrt{G}\,)_r = \sqrt{\tilde{G}} + r(\sqrt{\tilde{G}}\,)_r \to 1$ as $r\to 0$. Then}
		&= \int_0^\alpha (1+\phi\p) \dif{\theta} \\
		&= \left[ \theta+\phi(\theta) \right]_0^\alpha \\
		&= \alpha+\gamma-\left( \pi-\beta \right) \\
		&= \alpha+\gamma+\beta-\pi. \qedhere 
	\end{align*}
\end{proof}

\begin{corollary}
	For $\triangle ABC \subset U$ with geodesic sides, we have
	\begin{equation*}
		\defect(ABC) = \iint_{\triangle ABC} \f{-(\sqrt{G}\,)_{rr}}{\sqrt{G}} \dif{A_g}.
	\end{equation*}
\end{corollary}

\begin{proof}[Sketch proof]
	Introduce a point $O$ as follows:
	
	\begin{minipage}{0.85\textwidth}
		\centering
		\begin{tikzpicture}[scale=2]

			\coordinate (A) at (0,0);
			\coordinate (B) at (2,0);
			\coordinate (C) at ++(120:1.2);
			\coordinate (O) at ++(230:2);

			\draw (A) -- (B) -- (C) -- (A) -- (O) -- (B) -- (C) -- (O);

			\draw (0.15,0) arc (0:120:0.15);
			\draw (0,0) ++ (120:0.2) arc (120:230:0.2);
			\draw (0,0) ++ (230:0.25) arc (230:360:0.25);

			\draw (B) ++ (-0.6,0) arc (180:158.217:0.6);
			\draw (B) ++ (-0.7,0) arc (180:205:0.7);

			\draw (B) ++ (169.1085:0.8) node {$\beta_2$};
			\draw (B) ++ (192.5:0.85) node {$\beta_1$};

			\draw (A) ++ (230:1.6) arc (60:85.06:0.4);
			\draw (A) ++ (230:1.5) arc (60:35:0.5);

			\draw (C) ++ (-21.79:0.4) arc (-21.79:-60:0.4);
			\draw (C) ++ (-60:0.5) arc (-60:-104.92:0.5);

			\draw (A) ++ (60:0.3) node {$x_3$};
			\draw (A) ++ (175:0.35) node {$x_1$};
			\draw (A) ++ (295:0.4) node {$x_2$};

			\draw (O) ++ (37.5:0.75) node {$\alpha_2$};
			\draw (O) ++ (62.525:0.65) node {$\alpha_1$};

			\draw (C) ++ (-40.895:0.55) node {$\gamma_2$};
			\draw (C) ++ (-82.475:0.65) node {$\gamma_1$};

			\draw (B) node [right] {$B$};
			\draw (A) ++ (0.4,0.05) node [above] {$A$};
			\draw (O) node [below left] {$O$};
			\draw (C) node [above=2pt] {$C$};
		\end{tikzpicture}
	\end{minipage}
	\begin{minipage}{0.13\textwidth}
		\begin{align*}
			\triangle_1 &= \triangle OAC \\
			\triangle_2 &= \triangle BAO \\
			\triangle_3 &= \triangle CAB
		\end{align*}
	\end{minipage}

	Let $\psi=\triangle_1 \cup \triangle_2 \cup \triangle_3$. Then
	\begin{align*}
		\defect(\triangle_1) + \defect(\triangle_2) + \defect(\triangle_3)
		&= \gamma_1 + \gamma_2 + \alpha_1 \alpha_2 + \beta_1 + \beta_2 + x_1 + x_2 + x_3 - 3\pi \\
		&= \gamma_1 + \gamma_2 + \alpha_1 \alpha_2 + \beta_1 + \beta_2 - \pi
		 = \defect(\triangle_4).
	\end{align*}
	We can apply local Gauss-Bonnet to $\triangle_1, \triangle_2, \triangle_4$:
	\begin{align*}
		\defect(\triangle_3)
		&= \defect(\triangle_4)-\defect(\triangle_1)-\defect(\triangle_2) \\
		&= \iint_{\triangle_4} (-1)\dif{A_g} - \iint_{\triangle_1} (-1) \dif{A_g} - \iint_{\triangle_2} (-1)\dif{A_g} \\
		&= \iint_{\triangle_3} \f{-(\sqrt{G}\,)_{rr}}{\sqrt{G}} \dif{A_g},
	\end{align*}
	as required. This is only a sketch proof because we need to consider different configurations of points and triangles, but the other cases are very similar.
\end{proof}

\begin{corollary}
	\begin{equation*}
		\lim_{A,B,C\to p} \f{\defect(ABC)}{\Area(ABC)} = \eval[-1]{-\f{(\sqrt{G}\,)_{rr}}{\sqrt{G}}}_p.
	\end{equation*}
\end{corollary}

\begin{definition}
	If $g$ is a Riemannian metric on $U$, then the \emph{Gauss curvature} at $p$ is given by
	\begin{equation*}
		K_p(g) := \lim_{A,B,C\to p} \f{\defect(ABC)}{\Area(ABC)}.
	\end{equation*}
\end{definition}

Now, isometries preserve angles and areas, so if $\phi:(U_1,g_1) \to (U_2,g_2)$, then $K_p(g_1) = K_{\phi(p)}(g_2)$. The corollary shows that the limit in the definition exists (which is the hard part to prove), and is given by
\begin{equation*}
	-(\sqrt{G}\,)_{rr}/\sqrt{G} \qquad \text{if} \qquad g=\dif{r^2} + G\dif{\theta^2}.
\end{equation*}

\begin{example}
	Consider $g=g^D$, the hyperbolic metric on $D$.  In geodesic polar coordinates, this is equivalent to
	\begin{equation*}
		\overline{g} = \dif{r^2} + \sinh^2 r \dif{\theta} \qqand
		\sqrt{G} = \sinh r,
	\end{equation*}
	and the curvature is given by
	\begin{equation*}
		K = -\f{(\sqrt{G}\,)_{rr}}{\sqrt{G}} = -\f{\sinh r}{\sinh 1} \equiv -1.
	\end{equation*}
\end{example}

\begin{corollary}
	If $ABC$ is a triangle in $\HH$, then $\defect(ABC) = -\Area(ABC)$.
\end{corollary}

% subsection local_gauss_bonnet (end)   % Geodesics
%!TEX root = geometry.tex
\stepcounter{lecture}
\setcounter{lecture}{7}
\sektion{Surfaces}

\subsection{First fundamental form} % (fold)
\label{sub:first_fundamental_form}

Suppose $\sigma:U \to \R^3$ is a parameterised surface $S$ and let $g$ be the induced metric on $U$.

\begin{definition}
	The \emph{first fundamental form} at a point $p\in U$ is the bilinear form on $\R^2$ given by
	\begin{equation*}
		B_{I,p} := g_p(\vv,\ww).
	\end{equation*}
	This is represented by the matrix
	\begin{equation*}
		\mat{\sigma_x \\ \sigma_y} \mat{\sigma_x & \sigma_y} = \mat{\bsig_x \cdot \bsig_x & \bsig_x \cdot \bsig_y \\ \bsig_y \cdot \bsig_x & \bsig_y \cdot \bsig_y} = \mat{E & F \\ F & G}.
	\end{equation*}
\end{definition}

% subsection first_fundamental_form (end)

\subsection{Second fundamental form} % (fold)
\label{sub:second_fundamental_form}

The tangent space $T_{\sigma(p)} S$ is spanned by $\dif{\sigma_p}(1,0) = \bsig_x$ and $\dif{\sigma_p}(0,1) = \bsig_y$.

The unit normal to $T_{\sigma(p)} S$ at $\sigma(p)$ is
\begin{equation*}
	\nn(p) = \f{\bsig_x \cross \bsig_y}{\left\vert \bsig_x \cross \bsig_y \right\vert}.
\end{equation*}
The map $\nn:U\to S^2 \subset \R^3$ is called the \emph{Gauss map}.

\begin{definition}
	The \emph{second fundamental form} of $\sigma$ at $\pp$ is the bilinear form on $\R^2$ defined by
	\begin{equation*}
		B_{\text{\emph{II}},p}(\vv,\ww) = -\dif{\sigma_p(\vv)} \cdot \dif{\nn_p}(\ww)
	\end{equation*}
\end{definition}

There's a useful procedure for computing it. Let
\begin{itemize}
	\shortskip
	\item $S=\mat{\sigma_x & \sigma_y}$ be the $3\times 2$ matrix that represents $\dif{\sigma}$; and
	\item $N=\mat{\nn_x & \nn_y}$ be the matrix that represents $\dif{\nn}$.
\end{itemize}
Then we have
\begin{equation*}
	B_{\text{\emph{II}}}(\vv,\ww) = -(S\vv)^\Trans (N\ww)
	= - v^\Trans S^\Trans N \ww
	= -\vv^\Trans \mat{\sigma_x \\ \sigma_y} \mat{\nn_x & \nn_y} \ww.
\end{equation*}
Thus $B_{\text{\emph{II}}}$ is given by the matrix
\begin{equation*}
	-\mat{\sigma_x \\ \sigma_y} \mat{n_x & n_y}
	= -\mat{\sigma_x \cdot n_x & \sigma_x \cdot \nn_y \\ \sigma_y \cdot n_x & \sigma_y \cdot n_y}
	= \mat{L & M_1 \\ M_2 & N}.
\end{equation*}

\begin{lemma}
	\begin{equation*}
		\mat{L & M_1 \\ M_2 & N} = \mat{\sigma_{xx} \cdot \nn & \sigma_{xy} \cdot \nn \\ \sigma_{yx} \cdot \nn & \sigma_{yy} \cdot \nn}
	\end{equation*}
\end{lemma}

\begin{proof}
	If $\sigma_x \in T_{\sigma(p)} S$, then $\sigma_x \cdot \nn = 0$. Then $\sigma_{xx} \cdot \nn + \sigma_x \cdot \nn_x = 0$, so $-\sigma_x \cdot n_x = \sigma_{xx} \cdot \nn$. Thus $L=\sigma_{xx} \cdot \nn$. Other entries are similar.
\end{proof}

\begin{corollary}
	$B_{\text{\emph{II}}}$ is symmetric.
\end{corollary}

\begin{proof}
	We have $\sigma_{xy} = \sigma_{yx}$, so done.
\end{proof}

	\pagebreak

\begin{example}
	We have $\sigma(\theta,z) = (\cos \theta, \sin\theta, z)$; a cylinder of radius $1$. Thus
	\begin{equation*}
		\sigma_\theta=(-\sin \theta, \cos\theta, 0) \qqand
		\sigma_z= (0,0,1).
	\end{equation*}
	Then we have
	\begin{equation*}
		\mat{E & F \\ F & G} = \mat{1 & 0 \\ 0 & 1} \qqand
		g =\dif{z^2} + \dif{\theta^2},
	\end{equation*}
	so this is locally Euclidean. Thus the normal is
	\begin{equation*}
		\nn
		= \f{\sigma_\theta \cross \sigma_z}{\left\vert \sigma_{\theta} \cross \sigma_z \right\vert}
		= (\cos\theta, \sin\theta, 0)
	\end{equation*}
	Taking second derivatives gives
	\begin{equation*}
		\sigma_{\theta\theta} = (-\cos\theta, -\sin\theta,0) \qqand
		\sigma_{\theta z} = \sigma_{zz} = 0.
	\end{equation*}
	Thus our matrix is given by
	\begin{equation*}
		\mat{L & M \\ M & N} = \mat{-1 & 0 \\ 0 & 0} \qqand
		B_{\text{\emph{II}}} = \dif{\theta^2}.
	\end{equation*}
\end{example}

\begin{theorem}
	[Gauss' theorema egregium] If $g$ is the metric induced by $\sigma$, then
	\begin{equation*}
		K_p(g)
		= \f{\det(B_{\text{\emph{II}}}(\sigma))}{\det(B_I(\sigma))}
		= \f{LN-M^2}{EG-F^2}.
	\end{equation*}
\end{theorem}

\emph{See handout.}

% subsection second_fundamental_form (end)

% This was labelled section 7.4, but it's only 7.3 in LaTeX. Am I missing something?

\subsection{Closed surfaces and charts} % (fold)
\label{sub:closed_surfaces_and_charts}

We have the following basic problem:
\lecturemarker{16}{13 Mar}

\textbf{Problem.} A compact surface $S$ (such as the sphere $S^{\,2}$) cannot be written as the image of a single map $\sigma: U\to S$, where $U$ is open in $\R^2$.

This is actually a theorem, which can be proved using \emph{Algebraic Topology}.

\emph{Solution.} Cover $S$ with open sets, each of which is parameterised. This gives us something close to what we want. We require the following definition:

\begin{definition}
	If $S \subset \R^3$, a \emph{chart} for $S$ is an open set $V \subset S$ and a bijective map $f:V \xrightarrow{\text{open}} \R^2$ such that $\sigma = f^{-1}$ is a parametrisation.

	It might seem strange to think of the inverse of the map, but later it will be more convenient to think of charts in this way.

	If $f_i:V_i \to U_i$, $i=1,2$ are two charts on $S$, then the \emph{transition function} $\phi_{12}: f_1(V_1 \cap V_2) \to f_2 (V_1 \cap V_2)$ is given by $\phi_{12} = f_2 \circ f_1^{-1}$.

	We say that $f_1$ and $f_2$ are \emph{compatible} if $\phi_{12}$ and $\phi_{21} = \phi_{12}^{-1}$ are both differentiable.
	% \missingfigure{Geo 16/1}
\end{definition}

This might seem like a strange statement to make, because after some algebra we can prove that it always holds for embedded surfaces (the only surfaces that we've been considering). But later, when we consider abstract surfaces, this will turn out to be very useful.

\begin{definition}
	An \emph{atlas} for $S\subset \R^3$ is a set of compatible charts $f_i: V_i \to U_i$ such that the $V_i$ covers $S$. We say $S$ is an \emph{embedded surface} in $\R^3$ if it has an atlas.
\end{definition}

\begin{example}
	An atlas for $S^{\,2} \subset \R^3$ is
	\begin{itemize}
		\shortskip
		\item $\pi_1:S^2-\{N\} \to \R^2$ is stereographic projection from the north pole $N$;
		\item $\pi_2:S^2-\{S\} \to \R^2$ is stereographic projection from the north pole $S$.
	\end{itemize}
	We treat $\R^2-\{0\}$ as $\C^*$, and then our transition function is
	\begin{equation*}
		\fullfunction{\phi_{12}}{\C^*}{\C^*}{z}{1/\overline{z}}.
	\end{equation*}
\end{example}

\subsubsection*{Metrics} % (fold)
\label{ssub:metrics}

If $f_1,f_2$ are compatible charts on $S$, then $f_i^{-1}$ induces a Riemannian metric $g_i$ on $U_i$,, given by
\begin{equation*}
	g_i(\vv,\ww) = (\dif{f_i})^{-1}(\vv) \cdot (\dif{f_i^{-1}})(\ww).
\end{equation*}

\begin{lemma}
	$\phi_{12}:(f_1(V_1 \cap V_2), g_1)) \to (f_2(V_1 \cap V_2), g_2)$ is an isometry.
\end{lemma}

\begin{proof}
	Working through the algebra:
	\begin{align*}
		g_2(\dif{\phi_{12}(\vv)}, \dif{\phi_{12}(\ww)})
		&= (\dif{f_2})^{-1} (\dif{f_2} \circ (\dif{f_1})^{-1} (\vv)) \cdot \dif{f_2^{-1}}(\dif{f_2} \cdot (\dif{f_1})^{-1}(\ww)) \\
		&= \dif{f_1^{-1}}(\vv) \cdot \dif{f_1^{-1}}(\ww) \\
		&= g_1(\vv,\ww). \qedhere
	\end{align*}
\end{proof}

If $S\subset \R^3$ is a smoothly embedded surface, and $p\in S$, then the Gauss curvature is given by $K_p(S) := K_{f(p)}(g)$, where $f:V\to U$ is a chart defined in a neighbourhood of $p$ and $g$ is the metric induced on $f$.

The lemma implies that this is well-defined.

Similarly, $\gamma:(a,b) \to S$ is a geodesic if $f\circ \gamma$ is a geodesic with respect to the metric $g$ induced by $f$, where $f$ is any chart of $S$.

% subsubsection metrics (end)

% subsection closed_surfaces_and_charts (end)

\subsection{Abstract surfaces} % (fold)
\label{sub:abstract_surfaces}

Suppose $S$ is a Hausdorff, second-countable topological space. A chart on $S$ is an open set $V\subset S$ and a bijective map $f:V\to U \subset \R^2$, with $U$ open. (Don't worry if some of these terms are unfamiliar; they will be introduced formally in \emph{Metric \& Topological Spaces}. They are cited here merely for completeness.) Many definitions are the same as with closed surfaces:

If $f_1,f_2$ are charts on $S$, then the transition function $\phi_{12}:f_1(V_1 \cap V_2) \to f_2(V_1 \cap V_2)$ is given by $\phi_{12} = f_2 \circ f_1^{-1}$.

We say that $f_1$ and $f_2$ are compatible if $\phi_{12}$ are differentiable. This definition is exactly the same as before, but now it has teeth. In the embedded case, we merely need to ask that $f_1^{-1}$ be differentiable for $f_1$ and $f_2$ to be compatible. In this case, it doesn't make sense to ask that $f_1^{-1}$ be differentiable, so this is actually a useful distinction to make.

An atlas on $S$ is a set of compatible charts $f_i:V_i \to U_i$ such that the $V_i$ cover all of $S$.

\begin{definition}
	An \emph{abstract smooth surface} is a space $S$ as above together with an atlas on $S$.
\end{definition}

In some sense, there's nothing special about two dimensions in this definition. We could similarly define an abstract smooth $n$-manifold. Some other properties aren't so nice though. There are some four-manifolds which don't admit any structure as a smooth manifold, whereas $\R^4$ can be made into a smooth manifold in uncountably many ways.

In almost all cases, it is better to think about smooth manifolds, but these are not discussed in this course. We mention them here only for completeness, and will henceforth restrict our discussion to surfaces.

\begin{example}
	Consider the torus $T^2=\R^2/\Z^2$. There is a projection map $\pi:\R^2\to T^2$.
	Charts on $T^2$ are inverses of maps $\pi_U:U \to T^2$, the restriction of $\pi$ to an open set $U=B_\epsilon(p)$, $\epsilon<1/2$.
	% 
	% ensure that ball does not overlap itself when projected
	% inverse of map gives a triangle
	%
	Transition functions are translations by $(n,m)\iZ^2$.
\end{example}

\begin{definition}
	If $\left\{f_i:V_i \to U_i\right\}$ is an atlas on an abstract surface $S$, then a Riemannian metric on $S$ is a set of metrics $g_i$ on $U_i$ so that the transition functions $\phi_{ij}(f_i(V_i \cap V_j), g_i) \to (f_j(V_i \cap V_j), g_j)$ are all isometries.
\end{definition}

In an embedded surfaces, we get these as isometries for free. Here, we have to include it as part of the definition.

\begin{example}
	The flat metric on $T^2$ is defined by taking the atlas in the previous example, and equipping each $U$ with the Euclidean metric $\dif{x^2} + \dif{y^2}$. The transition functions are all translations, so isometries under $g^E$. The Gauss curvature of $g$ is identically zero.

	However, there is no way to embed $T^2$ into $\R^3$ such that the Gauss curvature is identically zero (see examples sheet).

	In some sense, it is better to think of this embedding as treating $T^2$ as the quotient of $(\R^2, g^E)$, by the action of a group of isometries.
\end{example}

Here the phrase \emph{flat metric} is used to describe a surface (or manifold) with identically zero Gauss curvature.

\begin{example}
	The Möbius strip also has a flat metric. The strip is given by $M=\R\times (-1,1)/G$, where $G\cong \Z$ and $K\cdot(X,Y) = (x+k, \left( -1 \right)^k + y)$. (Take the two sides of an infinite strip and glue them together with a strip, as we illustrated previously.) Again, this is an isometry of the Euclidean metric.
\end{example}

% subsection abstract_surfaces (end)

	\pagebreak

\subsection{Global Gauss-Bonnet} % (fold)
\label{sub:global_gauss_bonnet}

This leads us to the final theorem of the course, generalising the local Gauss-Bonnet theorem we saw previously:

\begin{theorem}
	If $(S,g)$ is a compact abstract surface equipped with a Riemannian metric $g$, then
	\begin{equation*}
		2\pi \,\chi(S) = \int_S K(g) \dif{A_g}.
	\end{equation*}
\end{theorem}

There are all sorts of beautiful theorems like this, which relate global topological information to local properties. This is not an isolated example, although it is the only such theorem we study in this course.

\begin{example}
	Take $S=S^{\,2}$ and let $g=g^{\,S}$ be the spherical metric. We know Gaussian curvature is $K \equiv 1$. Then
	\begin{equation*}
		2\pi \cdot 2 = \int_{S^2} 1 \dif{A} = 4\pi,
	\end{equation*}
	and everything is consistent.
\end{example}

The idea behind the theorem is quite easy. Technical details are needed to make it into a complete proof; here we present the main ideas.

\begin{proof}
	[Sketch proof] Find a geodesic triangulation of $S$ (that is, a triangulation where edges are geodesics), and so that each face is contained in a chart.
	% 
	The idea is to start with any triangulation, and subdivide the edges, replacing small edges by geodesics.
	
	\begin{center}
		\begin{tikzpicture}[scale=2.5]
			\draw [thick] plot [smooth] coordinates {(-1,0) (-0.9, 0.171) (-0.8, 0.288) (-0.7, 0.357) (-0.6, 0.384) (-0.5, 0.375) (-0.4, 0.336) (-0.3, 0.273) (-0.2, 0.192) (-0.1, 0.099) (0,0) (0.1, -0.099) (0.2, -0.192) (0.3, -0.273) (0.4, -0.336) (0.5, -0.375) (0.6, -0.384) (0.7, -0.357) (0.8, -0.288) (0.9, -0.171) (1,0)}; %
			
			\foreach \s/\t/\u/\v in {-1/0/-0.7/0.357, -0.7/0.357/-0.4/0.336, -0.4/0.336/0.1/-0.099, 0.1/-0.099/0.7/-0.357, 0.7/-0.357/1/0}
			{
				\draw (\s,\t) -- (\u,\v);
				\draw (\s,\t) node {$\bullet$};
			}
			
			\draw (1,0) node {$\bullet$};
			
		\end{tikzpicture}
		\end{center}

	Importantly, this does not change the topology of the triangulation.

	Now suppose triangulation has $V$ vertices, $E$ edges and $F$ faces. We know that $E=\f{3}{2}F$ (recall our discussion of the Euler characteristic for the sphere). Then
	\begin{align*}
		\iint_S K \dif{A_g}
		&= \sum_{i=1}^F \iint_{f_i} K \dif{A_g} \\
		\intertext{where $f_i$ is the $i$th face. Then we apply local Gauss-Bonnet, and letting $\alpha_{ij}$, $j=1,2,3$ be the angles in $f_i$:}
		&= \sum_{i=1}^F \defect(f_i) \\
		&= \sum_{i=1}^F \left( \alpha_{i1} + \alpha_{i2} + \alpha_{i3} - \pi \right) \\
		&= \sum_{i,j} \alpha_{ij} - \pi F
		 = 2\pi V - \pi F
		 = 2\pi\left( V-E+F \right). \qedhere
	\end{align*}
\end{proof}

% subsection global_gauss_bonnet (end)
   % Surfaces

%!TEX root = geometry.tex
\stepcounter{lecture}
\setcounter{lecture}{1}
\appendix
\sektion{Appendix: Review sheets}
\label{app:review-sheets}

\subsection{Euclidean geometry} % (fold)
\label{sub:euclidean_geometry}

Lines:
\begin{itemize}
	\shortskip
	\item A line is the shortest path between two points.
	\item Plane separation: the complement of a line is a disconnected topological space.
	\item There is a unique line passing through two distinct points.
	\item Two distinct lines intersect in at most one point.
	\item Given a point $\xx$ and a line $L$ not containing $\xx$, there is a unique line passing through $\xx$ and parallel to $L$. %
	\item Given a point $\xx$ and a line $L$ not containing $\xx$, there is a unique line passing through $\xx$ and perpendicular to $L$. %
\end{itemize}

Circles:
\begin{itemize}
	\shortskip
	\item A line and a circle intersect in at most two points.
	\item Two distinct circles intersect in at most two points.
	\item The perimeter of a circle of radius $R$ is $2\pi R$.
\end{itemize}

Isometries:
\begin{itemize}
	\shortskip
	\item If $F_1, F_2$ are orthogonal frames, then there is a unique isometry taking $F_1$ to $F_2$.
	\item Any isometry which fixes three non-colinear points is the identity.
	\item Any isometry can be written as the composition of at most three reflections.
\end{itemize}

Triangles:
\begin{itemize}
	\shortskip
	\item The sum of the interior angles in a triangle is $\pi$.
	\item If $A_1,A_2,A_3$ and $A_1\p,A_2\p,A_3\p$ are two sets of non-colinear points with $d(A_i,A_j) = d(A_i\p,A_j\p)$, then there is a unique $\phi\in\Isom(\R^2)$ with $\phi(A_i) = A_i\p$. %
	\item If instead we have $d(A_1,A_j) = d(A_1\p,A_j\p)$ and $\angle A_2 A_1 A_3 = \angle A_2\p A_1\p A_3\p$, then there is a unique $\phi\in\Isom(\R^2)$ with $\phi(A_i) = A_i\p$. %
\end{itemize}

Trigonometry:
\begin{itemize}
	\item If $\triangle ABC$ has sides $a,b,c$ and opposite angles $\alpha,\beta,\gamma$, then
	\begin{equation*}
		\f{\sin \alpha}{a} = \f{\sin \beta}{b} = \f{\sin\gamma}{c}, \qquad
		c^2 = a^2 + b^2 - 2ab \cos \gamma.
	\end{equation*}
\end{itemize}

% subsection euclidean_geometry (end)

	\pagebreak

\subsection{Spherical/projective geometry} % (fold)
\label{sub:spherical_p}

Spherical lines:
\begin{itemize}
	\shortskip
	\item A line is the shortest path between two points.
	\item Plane separation: the complement of a line is a disconnected topological space.
	\item There is a unique line passing through two distinct, non-antipodal points.
	\item Two distinct lines intersect in two points.
	\item Given a point $\xx$ and a line $L$ not containing $\xx$, there is a line passing through $\xx$ and perpendicular to $L$.
\end{itemize}

Projective lines:
\begin{itemize}
	\shortskip
	\item A line is the shortest path between two points.
	\item The complement of a line is connected.
	\item There is a unique line passing through two distinct, points.
	\item Two distinct lines intersect in exactly one point.
	\item Given a point $\xx$ and a line $L$ not containing $\xx$, there is a line passing through $\xx$ and perpendicular to $L$.
\end{itemize}

Circles:
\begin{itemize}
	\shortskip
	\item A line and a circle which is distinct from it intersect in at most two points.
	\item Two distinct circles intersect in at most two points.
	\item The perimeter of a circle of radius $R$ is $2\pi \sin R$.
\end{itemize}

Isometries:
\begin{itemize}
	\shortskip
	\item If $F_1, F_2$ are orthogonal frames, then there is a unique isometry taking $F_1$ to $F_2$.
	\item Any isometry which fixes three non-colinear points is the identity.
	\item Any isometry can be written as the composition of at most three reflections.
\end{itemize}

Triangles:
\begin{itemize}
	\shortskip
	\item The sum of the interior angles in a $\triangle ABC$ is $\pi+\Area(ABC)$.
	\item If $A_1,A_2,A_3$ and $A_1\p,A_2\p,A_3\p$ are two sets of non-colinear points with $d(A_i,A_j) = d(A_i\p,A_j\p)$, then there is a unique $\phi\in\Isom(\R^2)$ with $\phi(A_i) = A_i\p$. %
	\item If instead we have $d(A_1,A_j) = d(A_1\p,A_j\p)$ and $\angle A_2 A_1 A_3 = \angle A_2\p A_1\p A_3\p$, then there is a unique $\phi\in\Isom(\R^2)$ with $\phi(A_i) = A_i\p$. %
\end{itemize}

Trigonometry:
\begin{itemize}
	\item If $\triangle ABC$ has sides $a,b,c$ and opposite angles $\alpha,\beta,\gamma$, then
	\begin{equation*}
		\f{\sin \alpha}{\sin a} = \f{\sin \beta}{\sin b} = \f{\sin\gamma}{\sin c}, \qquad
		\begin{array}{l}
			\cos a = \ph \cos b \cos c + \sin \alpha \sin b \sin c, \\[3pt]
			\cos\alpha = -\cos\beta \cos \gamma + \sin a \sin \beta\sin\gamma.
		\end{array}
	\end{equation*}
\end{itemize}

% subsection spherical_p (end)

	\pagebreak

\subsection{Hyperbolic geometry} % (fold)
\label{sub:hyperbolic_geometry}

Models:
\begin{itemize}
	\shortskip
	\item Hyperboloid: % Originally "hyperboloid model"
	$S = \{(x,y,z): x^2+y^2-z^2=-1, z<0\}$, with the Minkowski metric $\dif{x^2} + \dif{y^2} - \dif{z^2}$ on $\R^3$. Lines are intersections of $S$ with planes through the origin.
	\item Unit disk model: $D = \left\{z \iC: \left\vert z \right\vert < 1\right\}$ with metric
	\begin{equation*}
		g^D = \f{4\left( \dif{x^2} + \dif{y^2} \right)}{\left( 1-x^2-y^2 \right)^2}.
	\end{equation*}
	Lines are Euclidean lines/circles perpendicular to $\boundary{D}$.
	\item Upper half plane model: $H = \left\{z \iC: \Re (z)>0\right\}$ with metric
	\begin{equation*}
		g^H = \f{\dif{x^2} + \dif{y^2}}{y^2}.
	\end{equation*}
	Lines are Euclidean lines/circles perpendicular to $\boundary{H}$.
\end{itemize}

Lines:
\begin{itemize}
	\shortskip
	\item A line is the shortest path between two points.
	\item Plane separation: the complement of a line is a disconnected topogical space.
	\item There is a unique line passing through two distinct points.
	\item Two distinct lines intersect in at most one point.
	\item Given $\xx$ and $L$ as previously, there is a unique line passing through $\xx$ perpendicular to $L$, and infinitely many lines passing through $\xx$ which do not intersect $L$.
\end{itemize}

Circles:
\begin{itemize}
	\shortskip
	\item In either the upper half-plane or the unit disk models, circles are Euclidean circles (but their centres are not the Euclidean centres.)
	\item A line and a circle, or two distinct circles, intersect in at most two points.
	\item The perimeter of a circle of radius $R$ is $2\pi\sinh R$.
\end{itemize}

Isometries:
\begin{itemize}
	\shortskip
	\item If $F_1, F_2$ are orthogonal frames, then there is a unique isometry taking $F_1$ to $F_2$.
	\item Any isometry which fixes three non-colinear points is the identity.
	\item Any isometry can be written as the composition of at most three reflections.
\end{itemize}

Triangles:
\begin{itemize}
	\shortskip
	\item The sum of the interior angles in a $\triangle ABC$ is $\pi-\Area(ABC)$.
	\item If $A_1,A_2,A_3$ and $A_1\p,A_2\p,A_3\p$ are two sets of non-colinear points with $d(A_i,A_j) = d(A_i\p,A_j\p)$, then there is a unique $\phi\in\Isom(\R^2)$ with $\phi(A_i) = A_i\p$. %
	\item If instead we have $d(A_1,A_j) = d(A_1\p,A_j\p)$ and $\angle A_2 A_1 A_3 = \angle A_2\p A_1\p A_3\p$, then there is a unique $\phi\in\Isom(\R^2)$ with $\phi(A_i) = A_i\p$. %
\end{itemize}

Trigonometry:
\begin{itemize}
	\item If $\triangle ABC$ has sides $a,b,c$ and opposite angles $\alpha,\beta,\gamma$, then
	\begin{equation*}
		\f{\sin \alpha}{\sinh a} = \f{\sin \beta}{\sinh b} = \f{\sin\gamma}{\sinh c}, \qquad
		\begin{array}{l}
			\cosh a = \cosh b \cosh c - \cos \alpha \sinh b \sinh c, \\[3pt]
			\cos\alpha = -\cos\beta \cos \gamma + \cosh a \sinh \beta\sinh\gamma.
		\end{array}
	\end{equation*}
\end{itemize}

\vfill

% subsection hyperbolic_geometry (end)   % Review sheets

\lecturenotesendshort

\end{document}